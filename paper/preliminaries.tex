\subsection{Groupoid Quotient}
Let us start by formally introducing the groupoid quotient \cite{sojakovaPhD}.
The groupoid quotient is a higher dimensional version of the set quotient,
so let us quickly recall the set quotient.
Given a setoid $(X,R)$ (a set $X$ with an equivalence relation $R$ valued in propositions on $X$),
the set quotient gives a type $\quot{X}{R}$, which is $X$ with the points identified according to $R$.
Note that $\quot{X}{R}$ always is a set since $R$ becomes the equality.

Instead of a setoid, the groupoid quotient takes a groupoid as input.
Recall that a groupoid is a category in which every morphism is invertible.
In particular, each groupoid has identity morphisms, denoted by $\idgrpd{x}$, and a composition operation.
The composition of $f$ and $g$ is denoted by $\compgrpd{f}{g}$.
In addition, the type of morphisms from $x$ to $y$ is required to be a set.
We write $\grpd$ for the type of groupoids.

Given $G : \grpd$, the groupoid quotient gives a 1-type $\gquotType(G)$.
In this type, the points are objects of $G$
and these are identified according to the morphisms in $G$.
In addition, the groupoid structure must be preserved.
Informally, we define the groupoid quotient as the following HIT.

\begin{lstlisting}[mathescape=true]
Inductive $\gquotType$ $(G : \grpd)$ :=
| $\gcl$ : $\functions{G}{\gquotType(G)}$
| $\gcleq$ : $\depprod{(x, y : G) (f : \mor{G}{x}{y})}{\gcl(x) = \gcl(y)}$
| $\ge$ : $\depprod{(x : G)}{\gcleq(\idgrpd{x}) = \refl{\gcl(x)}}$
| $\gconcat$ : $\depprod{(x, y, z : G)(f : \mor{G}{x}{y})(g : \mor{G}{y}{z})}{\gcleq(\compgrpd{f}{g}) = \concat{\gcleq(f)}{\gcleq(g)}}$
| $\gtrunc$ : $\depprod{(x, y : \gquotType(G))(p, q: x = y) (r, s : p = q)}{r = s}$
\end{lstlisting}

To formally add this type to our theory, we need to provide
introduction, elimination, and computation rules for $\gquotType(G)$.
Formulating the elimination principle requires two preliminary notions.
These are inspired by the work of Licata and Brunerie \cite{licata2015cubical}.
The first of these gives paths in a dependent type over a path in the base.

\begin{definition}
\label{def:path_over}
Given a type $X : \Type$,
a type family $Y : \functions{X}{\Type}$,
points $x_1, x_2 : X$,
a path $p : x_1 = x_2$,
and points $\overline{x_1} : Y(x_1)$ and $\overline{x_2} : Y(x_2)$ over $x_1$ and $x_2$ respectively,
we define the type $\pathover[Y]{p}{\overline{x_1}}{\overline{x_2}}$ of \fat{paths over} $p$ from $\overline{x_1}$ to $\overline{x_2}$ by path induction on $p$
by saying that the paths over the identity path $\refl{x}$ from $\overline{x_1}$ to $\overline{x_2}$ are just paths $\overline{x_1} = \overline{x_2}$.
\end{definition}

Note that the groupoid quotient also has constructors for paths between paths.
This means that we also need a dependent version of 2-paths,
and inspired by the terminology of globular sets, we call these \emph{globes} over a given 2-path.
We define them as follows.

\begin{definition}
\label{def:globe_over}
Let $X$, $Y$, and $x_1, x_2$ be as in Definition \ref{def:path_over}.
Suppose that we have paths $p, q : x_1 = x_2$,
a 2-path $g : p = q$,
and paths $\pover{p} : \pathover{p}{\overline{x_1}}{\overline{x_2}}$ and $\pover{q} : \pathover{q}{\overline{x_1}}{\overline{x_2}}$ over $p$ and $q$ respectively.
We define the type $\globeover{g}{\pover{p}}{\pover{q}}$ of \fat{globes over} $g$ from $\pover{p}$ to $\pover{q}$ by path induction on $g$
by saying that the paths over the identity path $\refl{p}$ are just paths $\pover{p} = \pover{q}$.
\end{definition}

From this point on, we assume that our type theory has the groupoid quotient.
More specifically, we assume the following axiom.

\begin{figure*}[t]
Introduction rules: 

\vspace{2pt}

\begin{center}
\begin{bprooftree}
\AxiomC{$x : G$}
\UnaryInfC{$\gcl(x) : \gquotType(G)$}
\end{bprooftree}
\begin{bprooftree}
\AxiomC{$x, y : G$}
\AxiomC{$f : \mor{G}{x}{y}$}
\BinaryInfC{$\gcleq(f) : \gcl(x) = \gcl(y)$}
\end{bprooftree}
\begin{bprooftree}
\AxiomC{$x : G$}
\UnaryInfC{$\ge(x) : \gcleq(\idgrpd{x}) = \refl{\gcl(x)}$}
\end{bprooftree}
\end{center}

\vspace{5pt}


\begin{center}
\begin{bprooftree}
\AxiomC{$x, y, z : G$}
\AxiomC{$f : \mor{G}{x}{y}$}
\AxiomC{$g : \mor{G}{y}{z}$}
\TrinaryInfC{$\gconcat(f, g) : \gcleq(\compgrpd{f}{g}) = \concat{\gcleq(f)}{\gcleq(g)}$}
\end{bprooftree}
\end{center}

\vspace{5pt}

\begin{center}
\begin{bprooftree}
\AxiomC{$x, y : \gquotType(G)$}
\AxiomC{$p, q : x = y$}
\AxiomC{$r, s : p = q$}
\TrinaryInfC{$\gtrunc(r, s) : r = s$}
\end{bprooftree}
\end{center}

\vspace{5pt}

Elimination rule:

\vspace{2pt}

\begin{center}
\begin{bprooftree}
\AxiomC{$Y : \gquotType(G) \rightarrow \onetypes$}
\AxiomC{$\gclY : \depprod{(x : G)}{Y(\gcl(x))}$}
\noLine
\BinaryInfC{$\gcleqY : \depprod{(x, y : G) (f : \mor{G}{x}{y})}{\pathover{\gcleq(f)}{\gclY(x)}{\gclY(y)}}$}
\noLine
\UnaryInfC{$\geY : \depprod{(x : G)}{\globeover{\ge(x)}{\gcleqY(\idgrpd{x})}{\refl{\gclY(x)}}}$}
\noLine
\UnaryInfC{$\gconcatY : \depprod{(x, y, z : G) (f : \mor{G}{x}{y}) (g : \mor{G}{y}{z})}{}$}
\noLine
\UnaryInfC{$\qquad\qquad\qquad\qquad\globeover{\gconcat(f, g)}{\gcleqY(\compgrpd{f}{g})}{\concat{\gcleqY(f)}{\gcleqY(g)}}$}
\UnaryInfC{$\gind(\gclY, \gcleqY, \geY, \gconcatY) : \depprod{(x : \gquotType(G))}{Y(x)}$}
\end{bprooftree}
\end{center}

\vspace{5pt}

Computation rules:

For $\gcl$: $\gind(\gclY, \gcleqY, \geY, \gconcatY)(\gcl(x)) \defeq \gclY(x)$

For $\gcleq$: $\apd{(\gind(\gclY, \gcleqY, \geY, \gconcatY))}{(\gcleq(f))} = \gcleqY(f)$
\caption{Introduction, elimination, and computation rules for the groupoid quotient \cite{sojakovaPhD}.}
\label{fig:gquot}
\end{figure*}

\begin{axiom}
For each groupoid $G$ there is a type $\gquotType(G)$ which satisfies the rules in Figure \ref{fig:gquot}.
\end{axiom}

Note that there are no computation rules for $\gconcat$, $\ge$, and $\gtrunc$,
because such rules follow from the fact that $Y$ is a family of 1-types.

\subsection{Bicategory Theory}
The upcoming constructions make heavy use of notions from bicategory theory \cite{10.1007/BFb0074299,leinster:basic-bicats}
and in particular, the displayed machinery introduced by Ahrens \etal \ \cite{bicatjournal}.
Here we recall some examples of bicategories and the basics of displayed bicategories.

A bicategory consists of objects, 1-cells between objects, and 2-cells between 1-cells.
The type of 1-cells from $x$ to $y$ is denoted by $x \onecell y$ and the type of 2-cells from $f$ to $g$ is denoted by $f \twocell g$.
Note that the type $f \twocell g$ is required to be a set.
There are identity 1-cells and 2-cells denoted by $\id_1$ and $\id_2$ respectively, composition of 1-cells $f$ and $g$ is denoted by $f \cdot g$,
and the vertical composition of 2-cells $\tc$ and $\tc'$ is denoted by $\tc \vcomp \tc'$.
The left whiskering of a 2-cell $\tc$ with 1-cell $f$ is denoted by $f \whiskerl \tc$ and right whiskering of $\tc$ with a 1-cell $g$ is denoted by $\tc \whiskerr g$. 
Unitality and associativity of vertical composition of 2-cells hold strictly, while For 1-cells these laws hold only up to invertible 2-cell. Given 1-cell $f : A \onecell B$, there are invertible 2-cells $\lunitor{f} : \id_1(A) \cdot f \twocell f$ and $\runitor{f} : f \cdot \id_1(B) \twocell f$.
Given three composable 1-cells $f,g$ and $h$, there is an invertible 2-cell $\lassoc{f}{g}{h} : f \cdot (g \cdot h) \twocell (f \cdot g) \cdot h$.

Let us fix some notation before continuing.  Given bicategories $\B_1$
and $\B_2$, we write $\pseudo(\B_1, \B_2)$ for the type of
pseudofunctors from $\B_1$ to $\B_2$. Pseudofunctors preserve identity
and composition of 1-cells up to invertible 2-cell. Given a
pseudofunctor $F : \pseudo(\B_1, \B_2)$ and an object $A$ of $\B_1$,
there are invertible 2-cells $\identitor{F} :\id_1(F(A)) \twocell
F(\id_1(A))$ and $\compositor{F} : F(f) \cdot F(g) \twocell F(f \cdot
g)$, with $f$ and $g$ two composable 1-cells. Preservation of identity
and composition of 2-cells is strict.
The type of pseudotransformations from $F$ to $G$ is denoted by
$\pstrans{F}{G}$.
The naturality square of a pseudotransformation $\theta : \pstrans{F}{G}$ commutes only up to invertible 2-cell. Given a 1-cell $f$ in $\B_1$, there is an invertible 2-cell $\theta_1(f) : \theta(A) \cdot G(f) \twocell F(f) \cdot \theta(B)$.
The type of modifications from $\theta$ to
$\theta'$ is denoted by $\modif{\theta}{\theta'}$
\cite{leinster:basic-bicats}.  Next we discuss \emph {biadjunctions}
\cite{gurski2012biequivalences,LACK2000179}.

\begin{definition}
Let $\B_1$ and $\B_2$ be bicategories.
A \fat{biadjunction} from $\B_1$ to $\B_2$ consists of
\begin{itemize}
	\item pseudofunctors $L : \pseudo(\B_1, \B_2)$ and $R : \pseudo(\B_2, \B_1)$;
	\item pseudotransformations $\eta : \pstrans{\id(\B_1)}{R \circ L}$ and $\varepsilon : \pstrans{L \circ R}{\id(\B_2)}$;
	\item invertible modifications
	\[
	\tau_1 : \modif{\rho(L)^{-1} \vcomp L \whiskerl \eta \vcomp \alpha(L, R, L) \vcomp \varepsilon \whiskerr L \vcomp \lambda(L)}{\id(L)}
	\]
	\[
	\tau_2 : \modif{\lambda(R)^{-1} \vcomp \eta \whiskerr R \vcomp \alpha(R, L, R)^{-1} \vcomp R \whiskerl \varepsilon \vcomp \rho(R)}{\id(R)}.
	\]
\end{itemize}
The type of biadjunctions from $\B_1$ to $\B_2$ is denoted by $L \dashv R$
where $L : \pseudo(\B_1, \B_2)$ and $R : \pseudo(\B_2, \B_1)$.
If we have $L \dashv R$, we say that $L$ is \fat{left biadjoint} to $R$.
\end{definition}

Before presenting the definition of coherent biadjunction, let us introduce some notation.
Suppose, that we have a biadjunction $L \dashv R$ where $L : \pseudo(\B_1, \B_2)$.
For each $x : \B_1$ we get a 1-cell $\eta(x) : x \onecell R(L(x))$ and for $x : \B_2$, we get a 1-cell $\varepsilon(x) : L(R(x)) \onecell x$.
Given $x : \B_1$, we get an invertible 2-cell $\widehat{\tau_2(x)} : L(\eta(x)) \cdot \varepsilon(L(x)) \twocell \id_1(L(x))$.
Similarly, we define an invertible 2-cell $\widehat{\tau_1(x)} : \eta(R(x)) \cdot R(\varepsilon(x)) \twocell \id_1(R(x))$ for $x : \B_2$.

\begin{definition}
Given bicategories $\B_1$ and $\B_2$ and a biadjunction $L \dashv R$ with $L : \pseudo(\B_1, \B_2)$, we say that $L \dashv R$ is \fat{coherent} if the following 2-cells are equal to identity 2-cells
\begin{equation*}
\begin{split}
(\identitor{L} \vcomp L(\widehat{\tau_1(x)}^{-1}) \vcomp \compositor{L}^{-1}) \whiskerr \varepsilon(x)
\vcomp \alpha^{-1}
\vcomp L(\eta(R(x))) \whiskerl \varepsilon_1(\varepsilon(x))^{-1}
\vcomp \alpha
\vcomp \widehat{\tau_2(R(x))} \whiskerr \varepsilon(x)
\end{split}
\end{equation*}
\begin{equation*}
\begin{split}
\eta(x) \whiskerl \widehat{\tau_1(L(x))}^{-1}
\vcomp \alpha
\vcomp (\eta_1(\eta(x)))^{-1} \whiskerr R(\varepsilon(L(x)))
\vcomp \alpha^{-1}
\vcomp \eta(x) \whiskerl (\compositor{R} \vcomp R(\widehat{\tau_2(x)}) \vcomp \identitor{R}^{-1})
\end{split}
\end{equation*}
\end{definition}

The following bicategories are important for subsequent constructions:
%Beside these standard notions, we use two bicategories:
$\onetypes$ and $\grpd$.

\begin{example}
We have
\begin{itemize}
	\item a bicategory $\onetypes$ whose objects are 1-types, 1-cells are functions, and 2-cells are homotopies;
	\item a bicategory $\grpd$ of groupoids
	whose objects are groupoids, 1-cells are functors, and 2-cells are natural transformations.
\end{itemize}
\end{example}
In homotopy type theory, groupoids (and more generally categories) are
usually required to be univalent, meaning that the type of isomorphisms between objects
in a groupoid is equivalent to to the type of equalities between objects. In this paper we
do not use this requirement, so the groupoids in $\grpd$ are not
necessarily univalent.


Next we discuss \emph{displayed bicategories}, which is our main tool to define bicategories of algebras for a signature.
Intuitively, a displayed bicategory $\D$ over $\B$ represents structure and properties to be added to $\B$.
Displayed bicategories generalize displayed categories to the bicategorical setting \cite{AhrensL19}.
Each such $\D$ gives rise to a total bicategory $\total{\D}$.
The full definition can be found in the paper by Ahrens \etal \ \cite{bicatjournal}, and here, we only show a part.

\begin{definition}
Let $\B$ be a bicategory.
A \fat{displayed bicategory} $\D$ over $\B$ consists of
\begin{itemize}
	\item For each $x : \B$ a type $\dob{\D}{x}$ of \fat{objects over $x$};
	\item For each $f : x \onecell y$, $\xx : \dob{\D}{x}$ and $\yy : \dob{\D}{y}$,
	a type $\dmor{\xx}{\yy}{f}$ of \fat{1-cells over $f$};
	\item For each $\tc : f \twocell g$, $\ff : \dmor{\xx}{\yy}{f}$, and $\gg : \dmor{\xx}{\yy}{g}$, a \emph{set} $\dtwo{\ff}{\gg}{\tc}$ of \fat{2-cells over $\tc$}.
\end{itemize}
In addition, there are identity cells and there are composition and whiskering operations.
The composition of displayed 1-cells $f$ and $g$ is denoted by $f \cdot g$, the displayed identity 1-cell is denoted by $\id_1(x)$.
The vertical composition of 2-cells $\tc$ and $\tc'$ is denoted by $\tc \vcomp \tc'$, the left and right whiskering is denoted by $f \whiskerl \tc$ and $\tc \whiskerr f$ respectively,
and the identity 2-cell is denoted by $\id_2(f)$. 
\end{definition}

\begin{definition}
\label{def:totalbicat}
Let $\B$ be a bicategory and let $\D$ be a displayed bicategory over $\B$.
We define the \fat{total bicategory} $\total{D}$ as the bicategory whose objects are dependent pairs $(x, \xx)$ with $x$ in $\B$ and $\xx$ in $\dob{\D}{x}$.
The 1-cells and 2-cells in $\total{D}$ are defined similarly.
In addition, we define the \fat{projection} $\dproj{D} : \pseudo(\total{D}, \B)$ to be the pseudofunctor which takes the first component of each pair.
\end{definition}

Let us finish this section by defining the displayed bicategories we need in the remainder of this paper.
Motivation and explanation of Examples \ref{ex:DFAlg} and \ref{ex:DCell} is given by Ahrens \etal ~ \cite{bicatjournal}.

\begin{example}
\label{ex:DFAlg}
Given a bicategory $\B$ and a pseudofunctor $F : \pseudo(\B, \B)$,
we define a displayed bicategory $\DFAlg(F)$ over $\B$ such that
\begin{itemize}
	\item the objects over $x : \B$ are 1-cells $h_x : F(x) \onecell x$;
	\item the 1-cells over $f : x \onecell y$ from $h_x$ to $h_y$ are invertible 2-cells $\tcC_f : h_x \cdot f \twocell F(f) \cdot h_y$;
	\item the 2-cells over $\tc : f \twocell g$ from $\tcC_f$ to $\tcC_g$ are equalities
	\begin{equation*}\label{eq:twocell_alg}
	h_x \whiskerl \tc \vcomp \tcC_g = \tcC_f \vcomp F(\tc) \whiskerr h_y.
	\end{equation*}
\end{itemize}
\end{example}

\begin{example}
\label{ex:disp_depprod}
Given a bicategory $\B$, a type $I$, and for each $i : I$ a displayed bicategory $\D_i$ over $\B$,
we define a displayed bicategory $\depprod{(i : I)}{\D_i}$ over $\B$ such that
\begin{itemize}
	\item the objects over $x : \B$ are functions $\depprod{(i : I)}{\D_i(x)}$;
	\item the 1-cells over $f : x \onecell y$ from $\overline{x}$ to $\overline{y}$ are functions $\depprod{(i : I)}{\dmor{\overline{x}(i)}{\overline{y}(i)}{f}}$;
	\item the 2-cells over $\tc : f \twocell g$ from $\overline{f}$ to $\overline{g}$ are functions $\depprod{(i : I)}{\dtwo{\overline{f}(i)}{\overline{g}(i)}{\tc}}$.
\end{itemize}
\end{example}

\begin{example}
\label{ex:DCell}
Let $\B$ be a bicategory with a displayed bicategory $\D$ over it.
Now suppose that we have pseudofunctors $S, T : \pseudo(\B, \B)$ and two pseudotransformations $l, r : \pstrans{\dproj{D} \cdot S}{\dproj{D} \cdot T}$.
Then we define a displayed bicategory $\DCell(l,r)$ over $\total{D}$ such that
\begin{itemize}
	\item the objects over $x$ are 2-cells $\tcB_x : l(x) \twocell r(x)$;
	\item the 1-cells over $f : x \onecell y$ from $\tcB_x$ to $\tcB_y$ are equalities
	\[
	(\tcB_x \whiskerr T(\dproj{D}(f))) \vcomp r(f)
	=
	l(f) \vcomp (S(\dproj{D}(f)) \whiskerl \tcB_y);
	\]
	\item the 2-cells over $\tc : f \twocell g$ are inhabitants of the unit type.
\end{itemize}
\end{example}

\begin{example}
\label{ex:fullsub}
Let $\B$ be a bicategory and let $P$ be a family of propositions on the objects of $\B$.
We define a displayed bicategory $\FSub(P)$ over $\B$ whose objects over $x$ are proofs of $P(x)$
and whose displayed 1-cells and 2-cells are inhabitants of the unit type.
The total bicategory $\total{\FSub(P)}$ is the \fat{full subbicategory} of $\B$ whose objects satisfy $P$.
\end{example}
