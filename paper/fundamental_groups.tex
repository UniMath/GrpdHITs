Let us finish by using Construction \ref{constr:hit_exist} to determine the fundamental group of some HITs.
Such results are often proven by encode-decode method \cite{LicataS13,LicataF14}, but we take a different approach.
We only use encode-decode to determine the path space of the groupoid quotient.
For other higher inductive types, we just find a simpler description of the initial groupoid algebra.
We start by determining the path space of the groupoid quotient.

\begin{proposition}
\label{prop:groupoid_quot_encode_decode}
Let $G$ be a groupoid and let $x$ and $y$ be objects in $G$.
Then the types $\gcl(x) = \gcl(y)$ and $\mor{G}{x}{y}$ are equivalent.
\end{proposition}

As a result, the type of paths between two points in a HIT is the type of morphisms in the initial groupoid algebra.
Since this algebra is unique, we can determine the fundamental group by finding a simpler description for the initial groupoid algebra.

\subsection{Circle}
\label{sec:circle_fund_group}
Let us start by taking a close look at the circle.
Recall that the circle is defined as the following HIT

\begin{lstlisting}[mathescape=true]
Inductive $\circleS$ :=
| $\baseS$ : $\circleS$
| $\SLoop$ : $\baseS = \baseS$
\end{lstlisting}

Note that we can define a signature $\circleS$ that represents this HIT.
Next we construct a groupoid algebra $\circlegrpd$ of this signature and we prove $\circlegrpd$ is biinitial.

\begin{definition}
We define a groupoid $\circlegrpd$ as follows
\begin{itemize}
	\item the type of objects is the unit type
	\item the type of morphisms from $\unitt$ to $\unitt$ is the integers
\end{itemize}
\end{definition}

\begin{problem}
\label{prob:initial_grpd_alg_circle}
To construct an $\circleS$-algebra structure on $\circlegrpd$.
\end{problem}

\begin{construction}{prob:initial_grpd_alg_circle}

\end{construction}

\begin{proposition}
The groupoid $\circlegrpd$ is biinitial in $\algG{\circleS}$.
\end{proposition}

From Construction \ref{constr:hit_exist} and the fact that biinitial objects are unique up to equivalence,
we can deduce that the circle is the groupoid quotient of $\circlegrpd$ and that its base point $\baseS$ is $\gcl(\unitt)$.
Since the morphisms of $\circlegrpd$ are just the integers, we immediately get the following from Proposition \ref{prop:groupoid_quot_encode_decode}

\begin{corollary}
The type $\baseS = \baseS$ is equivalent to the integers.
\end{corollary}

\subsection{Torus}
\label{sec:torus_fund_group}

We construct an algebra of the torus in groupoid and we prove it
equivalent to initial algebra of the torus in groupoid.

\begin{definition}
  Definition of the groupoid
\end{definition}

\begin{problem}
  \label{prob:initial_grpd_alg_torus}
  Algebra structure
\end{problem}

\begin{construction}{prob:initial_grpd_alg_torus}

\end{construction}

\begin{proposition}
  Initiality
\end{proposition}

\begin{corollary}
  The fundamental group of the torus is $\mathbb{Z} \times \mathbb{Z}$.
\end{corollary}

\subsection{Group Quotient}
\label{sec:group_quotient_fund_group}

Let $G$ be a group.  We construct an algebra of the group quotient of $G$ in
groupoid and we prove it equivalent to initial algebra of the group
quotient of $G$ in groupoid.

\begin{definition}
  Definition of the groupoid
\end{definition}

\begin{problem}
  \label{prob:initial_grpd_alg_group_quot}
  Algebra structure
\end{problem}

\begin{construction}{prob:initial_grpd_alg_group_quot}

\end{construction}

\begin{proposition}
  Initiality
\end{proposition}

\begin{corollary}
  The fundamental group of the group quotient of $G$ is $G$ itself.
\end{corollary}
