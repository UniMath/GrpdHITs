\begin{figure*}
\begin{center}
\begin{bprooftree}
\AxiomC{$P : \poly$}
\UnaryInfC{$\idE{A} : \pathendpoint{A}{P}{P}$}
\end{bprooftree}
\begin{bprooftree}
\AxiomC{$P, Q, R : \poly$}
\AxiomC{$e_1 : \pathendpoint{A}{P}{Q}$}
\AxiomC{$e_2 : \pathendpoint{A}{Q}{R}$}
\TrinaryInfC{$\comp{e_1}{e_2} : \pathendpoint{A}{P}{R}$}
\end{bprooftree}
\begin{bprooftree}
\AxiomC{$\constr : \pathendpoint{A}{A}{\idP}$}
\end{bprooftree}
\end{center}

\vspace{5pt}

\begin{center}
\begin{bprooftree}
\AxiomC{$P, Q : \poly$}
\UnaryInfC{$\inle : \pathendpoint{A}{P}{\sumP{P}{Q}}$}
\end{bprooftree}
\begin{bprooftree}
\AxiomC{$P, Q : \poly$}
\UnaryInfC{$\inre : \pathendpoint{A}{Q}{\sumP{P}{Q}}$}
\end{bprooftree}
\end{center}

\vspace{5pt}

\begin{center}
\begin{bprooftree}
\AxiomC{$P, Q : \poly$}
\UnaryInfC{$\prle : \pathendpoint{A}{\prodP{P}{Q}}{P}$}
\end{bprooftree}
\begin{bprooftree}
\AxiomC{$P, Q : \poly$}
\UnaryInfC{$\prre : \pathendpoint{A}{\prodP{P}{Q}}{Q}$}
\end{bprooftree}
\begin{bprooftree}
\AxiomC{$P, Q, R: \poly$}
\AxiomC{$e_1 : \pathendpoint{A}{P}{Q}$}
\AxiomC{$e_2 : \pathendpoint{A}{P}{R}$}
\TrinaryInfC{$\pair{e_1}{e_2} : \pathendpoint{A}{P}{\prodP{Q}{R}}$}
\end{bprooftree}
\end{center}

\vspace{5pt}

\begin{center}
\begin{bprooftree}
\AxiomC{$P : \poly$}
\AxiomC{$X : \onetypes$}
\AxiomC{$x : X$}
\TrinaryInfC{$\Ce(x) : \pathendpoint{A}{P}{\constantP{X}}$}
\end{bprooftree}
\begin{bprooftree}
\AxiomC{$X , Y : \onetypes$}
\AxiomC{$f : X \rightarrow Y$}
\BinaryInfC{$\fmap(f) : \pathendpoint{A}{\constantP{X}}{\constantP{Y}}$}
\end{bprooftree}
\end{center}
\caption{Rules for the path endpoints.}
\label{fig:path_ep}
\end{figure*}

\begin{figure*}
\begin{center}
\begin{bprooftree}
\AxiomC{$T : \poly$}
\AxiomC{$e : \pathendpoint{A}{R}{T}$}
\BinaryInfC{$\hrefl{e} : \homotendpoint{l}{r}{a_1}{a_2}{e}{e}$}
\end{bprooftree}
\begin{bprooftree}
\AxiomC{$T : \poly$}
\AxiomC{$e_1, e_2 : \pathendpoint{A}{R}{T}$}
\AxiomC{$h : \homotendpoint{l}{r}{a_1}{a_2}{e_1}{e_2}$}
\TrinaryInfC{$\hinv{h} : \homotendpoint{l}{r}{a_1}{a_2}{e_2}{e_1}$}
\end{bprooftree}
\end{center}

\vspace{5pt}

\begin{center}
\begin{bprooftree}
\AxiomC{$T : \poly$}
\AxiomC{$e_1, e_2, e_3 : \pathendpoint{A}{R}{T}$}
\AxiomC{$h_1 : \homotendpoint{l}{r}{a_1}{a_2}{e_1}{e_2}$}
\AxiomC{$h_2 : \homotendpoint{l}{r}{a_1}{a_2}{e_2}{e_3}$}
\QuaternaryInfC{$\hconcat{h_1}{h_2} : \homotendpoint{l}{r}{a_1}{a_2}{e_1}{e_3}$}
\end{bprooftree}
\end{center}

\vspace{5pt}

\begin{center}
\begin{bprooftree}
\AxiomC{$T_1, T_2 : \poly$}
\AxiomC{$e_1, e_2 : \pathendpoint{A}{Q}{T_1}$}
\AxiomC{$e : \pathendpoint{A}{T_1}{T_2}$}
\AxiomC{$h : \homotendpoint{l}{r}{a_1}{a_2}{e_1}{e_2}$}
\QuaternaryInfC{$\hap e h : \homotendpoint{l}{r}{a_1}{a_2}{\comp{e_1}{e}}{\comp{e_2}{e}}$}
\end{bprooftree}
\end{center}

\vspace{5pt}

\begin{center}
\begin{bprooftree}
\AxiomC{$T_1, T_2, T_3 : \poly$}
\AxiomC{$e_1 : \pathendpoint{A}{R}{T_1}$}
\AxiomC{$e_2 : \pathendpoint{A}{T_1}{T_2}$}
\AxiomC{$e_3 : \pathendpoint{A}{T_2}{T_3}$}
\QuaternaryInfC{$\hassoc{e_1}{e_2}{e_3} : \homotendpoint{l}{r}{a_1}{a_2}{\comp{e_1}{(\comp{e_2}{e_3})}}{\comp{(\comp{e_1}{e_2})}{e_3}}$}
\end{bprooftree}
\end{center}

\vspace{5pt}

\begin{center}
\begin{bprooftree}
\AxiomC{$T: \poly$}
\AxiomC{$e : \pathendpoint{A}{R}{T}$}
\BinaryInfC{$\hlunit{e} : \homotendpoint{l}{r}{a_1}{a_2}{\comp{\idE{R}}{e}}{e}$}
\end{bprooftree}
\begin{bprooftree}
\AxiomC{$T: \poly$}
\AxiomC{$e : \pathendpoint{A}{R}{T}$}
\BinaryInfC{$\hrunit{e} : \homotendpoint{l}{r}{a_1}{a_2}{\comp{e}{\idE{T}}}{e}$}
\end{bprooftree}
\end{center}

\vspace{5pt}

\begin{center}
\begin{bprooftree}
\AxiomC{$T_1, T_2 : \poly$}
\AxiomC{$e_1 : \pathendpoint{A}{R}{T_1}$}
\AxiomC{$e_2 : \pathendpoint{A}{R}{T_2}$}
\TrinaryInfC{$\hpairprle : \homotendpoint{l}{r}{a_1}{a_2}{\comp{\pair{e_1}{e_2}}{\prle}}{e_1}, \quad \hpairprre : \homotendpoint{l}{r}{a_1}{a_2}{\comp{\pair{e_1}{e_2}}{\prre}}{e_2}$}
\end{bprooftree}
\end{center}

\vspace{5pt}

\begin{center}
\begin{bprooftree}
\AxiomC{$T_1, T_2 : \poly$}
\AxiomC{$e_1, e_2 : \pathendpoint{A}{R}{T_1}$}
\AxiomC{$e_3, e_4 : \pathendpoint{A}{R}{T_2}$}
\AxiomC{$h_1 : \homotendpoint{l}{r}{a_1}{a_2}{e_1}{e_2}$}
\AxiomC{$h_2 : \homotendpoint{l}{r}{a_1}{a_2}{e_3}{e_4}$}
\QuinaryInfC{$\hpair{h_1}{h_2} : \homotendpoint{l}{r}{a_1}{a_2}{\pair{e_1}{e_3}}{\pair{e_2}{e_4}}$}
\end{bprooftree}
\end{center}

\vspace{5pt}

\begin{center}
\begin{bprooftree}
\AxiomC{$T_1, T_2 , T_3: \poly$}
\AxiomC{$e_1 : \pathendpoint{A}{R}{T_1}$}
\AxiomC{$e_2 : \pathendpoint{A}{T_1}{T_2}$}
\AxiomC{$e_3 : \pathendpoint{A}{T_1}{T_3}$}
\QuaternaryInfC{$\hcomppair : \homotendpoint{l}{r}{a_1}{a_2}{\comp{e_1}{\pair{e_2}{e_3}}}{\pair{\comp{e_1}{e_2}}{\comp{e_1}{e_3}}}$}
\end{bprooftree}
\end{center}

\vspace{5pt}

\begin{center}
\begin{bprooftree}
\AxiomC{$T: \poly$}
\AxiomC{$X: \onetypes$}
\AxiomC{$x: X$}
\AxiomC{$e : \pathendpoint{A}{R}{T}$}
\QuaternaryInfC{$\hcompconst x : \homotendpoint{l}{r}{a_1}{a_2}{\comp{e}{\Ce x}}{\Ce x}$}
\end{bprooftree}
\end{center}

\vspace{5pt}

\begin{center}
%% \begin{bprooftree}
%% \AxiomC{$T_1, T_2 : \poly$}
%% \AxiomC{$e_1, e_2 : \pathendpoint{A}{R}{T_1}$}
%% \AxiomC{$h : \homotendpoint{l}{r}{a_1}{a_2}{e_1}{e_2}$}
%% \TrinaryInfC{$\hinl{h} : \homotendpoint{l}{r}{a_1}{a_2}{\comp{e_1}{\inle}}{\comp{e_2}{\inle}}$}
%% \end{bprooftree}
\begin{bprooftree}
\AxiomC{$j : J$}
\AxiomC{$e : \pathendpoint{A}{R}{Q_j}$}
\BinaryInfC{$\hconstr{j}{e} : \homotendpoint{l}{r}{a_1}{a_2}{\comp{e}{l(j)}}{\comp{e}{r(j)}}$}
\end{bprooftree}
%% \vspace{5pt}
%% \begin{bprooftree}
%% \AxiomC{$T_1, T_2 : \poly$}
%% \AxiomC{$e_1, e_2 : \pathendpoint{A}{R}{T_2}$}
%% \AxiomC{$h : \homotendpoint{l}{r}{a_1}{a_2}{e_1}{e_2}$}
%% \TrinaryInfC{$\hinr{h} : \homotendpoint{l}{r}{a_1}{a_2}{\comp{e_1}{\inre}}{\comp{e_2}{\inre}}$}
%% \end{bprooftree}
\begin{bprooftree}
\AxiomC{$\harg : \homotendpoint{l}{r}{a_1}{a_2}{a_1}{a_2}$}
\end{bprooftree}
\end{center}


\caption{Rules for the homotopy endpoints.}
\label{fig:homot_ep}
\end{figure*}

Before we can discuss how to construct 1-truncated higher inductive types,
we need to define signatures for those.
Our notion of signature is similar to the one by Dybjer and Moenclaey \cite{DBLP:journals/entcs/DybjerM18}.
However, instead of defining them externally, we define a type of signatures within type theory
just like what was done for inductive-recursive and inductive-inductive
definitions \cite{Dybjer1999AFA,forsberg2012finite}. 
In addition, we show that each signature $\sign$ gives rise to a bicategory of algebras for $\sign$.

In this section, we study HITs of the following shape
\begin{lstlisting}[mathescape=true]
Inductive $H$ :=
| $c$ : $\functions{\polyAct{P}{H}}{H}$
| $p$ : $\depprod{(j : I) (x : \polyAct{Q}{H})}{l(x) = r(x)}$
| $s$ : $\depprod{(j : J) (x : \polyAct{R}{H}) (r : a_1(x) = a_2(x))}$
      ${\homotendpointAct{q_1}{x, r} = \homotendpointAct{q_2}{x, r}}$
| $t$ : $\depprod{(x, y : H) (q_1, q_2 : x = y) (r, s : q_1 = q_2)}{r = s}$
\end{lstlisting}
To see what the challenges are when defining such HITs, let us take a closer look at the torus.

\begin{lstlisting}[mathescape=true]
Inductive $\torus$ :=
| $\base$ : $\torus$
| $\leftLoop, \rightLoop$ : $\base = \base$
| $\surface$ : $\concat{\leftLoop}{\rightLoop} = \concat{\rightLoop}{\leftLoop}$
\end{lstlisting}

There is a point constructor $\base$, two paths constructors $\leftLoop, \rightLoop : \base = \base$,
and a homotopy constructor $\surface : \concat{\leftLoop}{\rightLoop} = \concat{\rightLoop}{\leftLoop}$.
Note that $\leftLoop$ and $\rightLoop$ refer to $\base$ and that $\surface$ refers to all other constructors.
Hence, the signatures we define, must be flexible enough to allow such dependencies.

A similar challenge comes up when defining the bicategory of algebras for a signature.
For the torus, an algebra would consist of a type $X$, a point $b$, paths $p, q : b = b$, and a 2-path $s : \concat{p}{q} = \concat{q}{p}$.
Again there are dependencies: $p$ and $q$ depend on $b$ while $s$ depends on both $p$ and $q$.
To overcome this challenge, we use displayed bicategories to construct the bicategory of algebras in a stratified way.

\subsection{Signatures}
Now let us define signatures, and to do so, we must specify data which describes the constructors for points, paths, and homotopies.
To specify the point constructors, we use \emph{polynomial codes}.
Given a type $X$ and a polynomial code $P$, we get another type $\polyAct{P}{X}$.
Such a code $P$ describes an operation of the form $\polyAct{P}{X} \rightarrow X$.

\begin{definition}
The type of \fat{codes for polynomials} is inductively generated by the following constructors
\[ 
\constantP{A} : \poly, \quad \idP : \poly, \quad \sumP{P_1}{P_2} : \poly, \quad \prodP{P_1}{P_2} : \poly
\]
where $A$ is a 1-type and $P_1$ and $P_2$ are elements of $\poly$.
\end{definition}

The constructor $\constantP{A}$ represents the constant polynomial, $\idP$ represents the identity,
and $\sumP{P_1}{P_2}$ and $\prodP{P_1}{P_2}$ represent the sum and product respectively.
Note that we restrict ourselves to finitary polynomials since we do not have a constructor which represents the function space.

The second part of the signature describes the path constructors, which represent universally quantified equations.
To describe them, we must give two \emph{path endpoints}.
These endpoints can refer to the point constructor, which we represent by a polynomial $A$.
In addition, they have a source (the type of the quantified variable) and a target (the type of the term).
The source and the target are represented by polynomials $S$ and $T$ respectively.

\begin{definition}
Let $A$, $S$, and $T$ be codes for polynomials.
The type $\pathendpoint{A}{S}{T}$ of \fat{path endpoints} with arguments $A$, source  $S$, and target $T$ is inductively generated by the constructors given in Figure \ref{fig:path_ep}.
\end{definition}

Note that the parameter $A$ is only used in the path endpoint $\constr$, which represents the point constructor.
If we have a type $X$ with a function $c : \polyAct{A}{X} \rightarrow X$,
then each endpoint $e$ gives for every $x : \polyAct{S}{X}$ a point $\pathendpointFun{e}(x) : \polyAct{T}{X}$.
Note that $\pathendpointFun{e}(x)$ depends on $c$ while we do not write $c$ in the notation.
Hence, two endpoints $e_l, e_r$ represent the equation
\[
\depprod{(x : \polyAct{S}{X})}{\pathendpointAct{e_l}{x} = \pathendpointAct{e_r}{x}}.
\]
Note that a HIT could have arbitrarily many path constructors and we index them by the type $J$.

The last part of the signature describes the homotopy constructors
and these depend on both the point and path constructors.
A homotopy constructor represents an equation of paths, which is universally quantified over both points and paths of
the HIT being defined.
The point argument is represented by a polynomial $R$, and the path argument is represented by a polynomial $T$ and endpoints $a_1, a_2 : \pathendpoint{A}{R}{T}$.
Lastly, the type of the paths in the equation is described by two endpoints $s_l, s_r : \pathendpoint{A}{R}{W}$ with $W : \poly$.

\begin{definition}
Suppose that we have
\begin{itemize}
	\item A polynomial $A$;
	\item A type $J$ together with for each $j : J$ a polynomial $Q_j$ and endpoints $l_j, r_j : \pathendpoint{A}{Q_j}{\idP}$;
	\item A polynomial $R$;
	\item A polynomial $T$ with endpoints $a_1, a_2 : \pathendpoint{A}{R}{T}$;
	\item A polynomial $W$ with endpoints $s_1, s_2 : \pathendpoint{A}{R}{W}$.
\end{itemize}
Then we define the type $\homotendpoint{l}{r}{a_1}{a_2}{s_1}{s_2}$ of \fat{homotopy endpoint} inductively by the constructors in Figure \ref{fig:homot_ep}.
\end{definition}

There are three homotopy endpoints of particular importance.
The first one is $\constructor{path}$, which represents the path constructor and it makes use of $l_j$ and $r_j$.
The second one, $\harg$, represents the path argument and it uses $a_1$ and $a_2$.
The last one is $\constructor{ap}$ and it represents the action of the point constructor on a homotopy endpoint.

The way we represent path arguments allows us to represent equations with any finite number of path arguments by only two path endpoints.
For example, two path arguments $p : x_1 = y_1$ and $q : x_2 = y_2$ is represented by one path argument of type $(x_1, x_2) = (y_1, y_2)$.

From the grammar in Figure \ref{fig:homot_ep}, we can derive
the following additional homotopy endpoints, which be employed in forthcoming
examples. 

\vspace{5pt}

\begin{center}
\begin{bprooftree}
\AxiomC{$T_1, T_2 : \poly$}
\AxiomC{$e_1, e_2 : \pathendpoint{A}{R}{T_1}$}
\AxiomC{$e_3, e_4 : \pathendpoint{A}{R}{T_2}$}
\AxiomC{$h : \homotendpoint{l}{r}{a_1}{a_2}{\pair{e_1}{e_3}}{\pair{e_2}{e_4}}$}
\QuaternaryInfC{$\hprl{h} \eqdef \hconcat{\hinv{\hpairprle}}{\hconcat{\hap {\prle} h}{\hpairprle}} : \homotendpoint{l}{r}{a_1}{a_2}{e_1}{e_2}$}
\alwaysNoLine
\UnaryInfC{$\hprr{h} \eqdef \hconcat{\hinv{\hpairprre}}{\hconcat{\hap {\prre} h}{\hpairprre}} : \homotendpoint{l}{r}{a_1}{a_2}{e_3}{e_4}$}
\end{bprooftree}
\end{center}

\vspace{5pt}
Moreover, we have a function $\idtoH$ sending a path between endpoints $e_1 = e_2$ into an homotopy endpoint $\homotendpoint{l}{r}{a_1}{a_2}{e_1}{e_2}$, readily definable by path induction.

Given a type $X$ with a function $c : \polyAct{A}{X} \rightarrow X$ and for each $x : Q_j(X)$ a path $l_j(x) = r_j(x)$,
a homotopy endpoint $p : \homotendpoint{l}{r}{a_1}{a_2}{s_1}{s_2}$ gives rise for each point $x : \polyAct{R}{X}$ and path $w : a_1(x) = a_2(x)$
to another path $\homotendpointAct{p}{x, w} : s_1(x) = s_2(x)$.
Hence, two homotopy endpoints $p, q : \homotendpoint{l}{r}{a_1}{a_2}{s_1}{s_2}$ represent the equation
\[
\depprod{(x : \polyAct{R}{X}) (w : a_1(x) = a_2(x))}{\homotendpointAct{p}{x, w} = \homotendpointAct{q}{x, w}}
\]
Now let us put this all together and define what signatures for higher inductive types are.

\begin{definition}
\label{def:signature}
A \fat{HIT-signature} $\sign$ consists of
\begin{itemize}
	\item A polynomial $\pointconstr[\sign]$;
	\item A type $\pathlabel[\sign]$ together with for each $j : \pathlabel[\sign]$ a polynomial $\patharg[\sign]_j$ and endpoints $\pathleft[\sign]_j, \pathright[\sign]_j : \pathendpoint{\pointconstr[\sign]}{\patharg[\sign]_j}{\idP}$;
	\item A type $\homotlabel[\sign]$ together with for each $j : \homotlabel[\sign]$ polynomials $\homotpointarg[\sign]_j$ and $\homotpathtarg[\sign]_j$,
	endpoints $\pathargleft[\sign]_j, \pathargright[\sign]_j : \pathendpoint{\pointconstr[\sign]}{\homotpointarg[\sign]_j}{\homotpathtarg[\sign]_j}$
	and $\homotpathleft[\sign]_j, \homotpathright[\sign]_j : \pathendpoint{\pointconstr[\sign]}{\homotpointarg[\sign]_j}{\idP}$,
	and homotopy endpoints $\homotleft[\sign]_j, \homotright[\sign]_j : \homotendpoint{\pathleft[\sign]_j}{\pathright[\sign]_j}{\pathargleft[\sign]_j}{\pathargright[\sign]_j}{\homotpathleft[\sign]_j}{\homotpathright[\sign]_j}$.
\end{itemize}
\end{definition}

If $\sign$ is clear from the context, we do not write the superscript.
In the remainder, we show how to interpret the following HIT given a signature $\sign$:

\begin{lstlisting}[mathescape=true]
Inductive $H$ :=
| $c$ : $\functions{\polyAct{\pointconstr}{H}}{H}$
| $p$ : $\depprod{(j : \pathlabel) (x : \polyAct{\patharg_j}{H})}{\pathendpointAct{\pathleft_j}{x} = \pathendpointAct{\pathright_j}{x}}$
| $s$ : $\depprod{(j : \homotlabel) (x : \polyAct{\homotpointarg_j}{H}) (r : \pathargleft_j(x) = \pathargright_j(x))}$
      ${\homotendpointAct{\homotleft_j}{x, r} = \homotendpointAct{\homotright_j}{x, r}}$
| $t$ : $\depprod{(x, y : H) (p, q : x = y) (r, s : p = q)}{r = s}$
\end{lstlisting}

Next we consider three examples of HITs we can express with these signatures.

\begin{example}
\label{ex:torus}
The torus is described by the signature $\torus$.
\begin{itemize}
	\item Take $\pointconstr[\torus] = \constantP{\unit}$;
	\item Take $\pathlabel[\torus] = \bool$ and for both inhabitants we take $\patharg[\torus] = \constantP{\unit}$ and $\pathleft[\torus] = \pathright[\torus] = \constr$;
	\item Take $\homotlabel[\torus] = \unit$.
	Since there no arguments for this path constructor, we take $\homotpointarg[\torus] = \constantP{\unit}$ and $\pathargleft[\torus] = \pathargright[\torus] = \Ce(\unitt)$.
	Now for the left-hand side and right-hand side of this equation, we take $\hconcat{\hconstr{\booltrue}{\idENA}}{\hconstr{\boolfalse}{\idENA}}$ and $\hconcat{\hconstr{\boolfalse}{\idENA}}{\hconstr{\booltrue}{\idENA}}$ respectively. 
\end{itemize}
\end{example}

\begin{example}
\label{ex:mod}
We represent the integers modulo 2 as the following HIT:
\begin{lstlisting}[mathescape=true]
Inductive $\ZT$ :=
| $\ZZ$ : $\ZT$
| $\ZS$ : $\ZT \rightarrow \ZT$
| $\ZM$ : $\depprod{(x : \ZT)}{\ZS(\ZS(x)) = x}$
| $\ZC$ : $\depprod{(x : \ZT)}{\ZM(\ZS(x)) = \ap{\ZS}{(\ZM(x))}}$
\end{lstlisting}
Note that all constructors except $\ZZ$ are recursive.
We define a signature $\ZT$.
\begin{itemize}
	\item Take $\pointconstr[\ZT] = \sumP{\constantP{\unit}}{\idP}$;
	\item Take $\pathlabel[\ZT] = \constantP{\unit}$ and for its unique inhabitant we take $\patharg[\ZT] = \idP$ and
	\[
	\pathleft[\ZT] = \comp{(\comp{\inre}{\constr})}{(\comp{\inre}{\constr})}, \quad
	\pathright[\ZT] = \idENA;
	\]
	\item Take $\homotlabel[\ZT] = \constantP{\unit}$.
	Furthermore, we take $\homotpointarg[\ZT] = \idP$ and $\pathargleft[\ZT] = \pathargright[\ZT] = \Ce(\unitt)$.
	The endpoints $\homotpathleft$ and $\homotpathright$ encode $\ZS(\ZS(\ZS(x)))$ and $\ZS(x)$ respectively,
	and for the left-hand side and right-hand side of this equation, we take
	\[
	\hap{\constr}{(\hconcat{\hinv{\hlunitN}}{\hconcat{\hassocN}{\hconcat{\hap{\inre}{(\hconstr{\unit}{\idENA})}}{\hconcat{\hinv{\hassocN}}{\hconcat{\hlunitN}{\hlunitN}}}}})}
	\]
	\[
	\hconcat{\hinv{\hassocN}}{\hconcat{\hinv{\hassocN}}{\hconcat{\hconstr{\unitt}{\comp{\inre}{\constr}}}{\hrunitN}}}.
	\]
	respectively.
	Note that we use $\hassocN$, $\hlunitN$, and $\hrunitN$ to make the equations type check.
	If we would interpret the left-hand side and right-hand side of  the homotopy constructor in 1-types,
	then all occurrences of $\hassocN$, $\hlunitN$, and $\hrunitN$ become the identity path.
	We thus get the right homotopy constructor.
\end{itemize}
\end{example}

\begin{example}
\label{ex:settrunc}
Given a 1-type $A$,  the set truncation of $A$ is defined by the following HIT:
\begin{lstlisting}[mathescape=true]
Inductive $\ST{A}$ :=
| $\SC$ : $A \rightarrow \ST{A}$
| $\Strunc$ : $\depprod{(x, y : \ST{A}) (p, q : x = y)}{p = q}$
\end{lstlisting}
Note that this higher inductive type has a parameter $A$,
so the signature we define, depends on a 1-type $A$ as well.
To encode the path arguments of $\Strunc$, we use that two paths $p, q : x = y$ is the same as a path $r : (x, x) = (y, y)$.
Define a signature $\ST{A}$ such that
\begin{itemize}
	\item $\pointconstr[\ST{A}] = \constantP{A}$;
	\item $\pathlabel[\ST{A}]$ is the empty type;
	\item $\homotlabel[\ST{A}] = \unit$.
	In addition, there are two point arguments $\homotpointarg[\ST{A}] = \prodP{\idP}{\idP}$
	and a path argument with left-hand side $\pair{\prle}{\prle}$ and right-hand side $\pair{\prre}{\prre}$.
	For the left-hand side and right-hand side of the homotopy, we take $\hprl{\harg}$ and $\hprr{\harg}$ respectively.
\end{itemize}
\end{example}


More examples of HIT signatures are discussed in Section \ref{sec:examples}.

\subsection{Algebras in 1-types and groupoids}
With the signatures in place, our next goal is to study the introduction rules of HITs and for that, we define bicategories of algebras for a signature.
Since we ultimately want to construct HITs via the groupoid quotient, we look at both algebras in 1-types and groupoids.

In both cases, we use a stratified approach with displayed bicategories.
Let us illustrate this by briefly describing the construction for 1-types.
On $\onetypes$, we define a displayed bicategory and we denote its total bicategory by $\prealg{\sign}$.
The objects of $\prealg{\sign}$ consist of a 1-type $X$ together with an operation $\polyAct{P}{X} \rightarrow X$.
Concretely, the objects satisfy the point introduction rules specified by $\sign$.
On top of $\prealg{\sign}$, we define another displayed bicategory whose total bicategory is denoted by $\pathalg{\sign}$.
Objects of $\pathalg{\sign}$ satisfy the introduction rules for both the points and the paths.
Lastly, we take a full subbicategory of $\pathalg{\sign}$ obtaining another bicategory $\alg(\sign)$ whose objects satisfy the introduction rules for the points, paths and homotopies.

To define $\prealg{\sign}$, we use Example \ref{ex:DFAlg}.

\begin{problem}
\label{prob:sem_poly}
Given $P : \poly$, to construct pseudofunctors
\[
\semP{P} : \pseudo(\onetypes, \onetypes), 
\semPG{P} : \pseudo(\grpd, \grpd).
\]
\end{problem}

\begin{construction}{prob:sem_poly}
We only discuss the case for 1-types since the case for groupoids is similar.
Given a polynomial $P$ and a type $X$, we get a type $\polyAct{P}{X}$ by induction.
The verification that this gives rise to a pseudofunctor can be found in the formalization.
\end{construction}

\begin{definition}
\label{def:prealg}
Let $\sign$ be a signature.
Then we define the bicategories $\prealg{\sign}$ and $\prealgG{\sign}$
to be the total bicategories of $\DFAlg(\semP{\pointconstr[\sign]})$
and $\DFAlg(\semPG{\pointconstr[\sign]})$ respectively.
Objects of these bicategories are called \fat{prealgebras} for $\sign$.
\end{definition}

Note that prealgebras only have structure witnessing the introduction rule for the points.
Next we look at the introduction rule for the paths.
In this case, the desired structure is added via Example \ref{ex:DCell} and to apply this construction,
we interpret path endpoints as pseudotransformations.

\begin{problem}
\label{prob:sem_endpoint}
Given $e : \pathendpoint{A}{P}{Q}$, to construct pseudotransformations
\[
\semE{e} : \pstrans{\semP{P} \cdot \dproj{\DFAlg(\semP{A})}}{\semP{Q} \cdot \dproj{\DFAlg(\semP{A})}},
\]
\[
\semEG{e} : \pstrans{\semPG{P} \cdot \dproj{\DFAlg(\semPG{A})}}{\semPG{Q} \cdot \dproj{\DFAlg(\semPG{A})}}.
\]
\end{problem}

\begin{construction}{prob:sem_endpoint}
We only discuss $\semE{e}$ since $\semEG{e}$ is defined similarly.
Given a 1-type $X$ and $c : \polyAct{A}{X} \rightarrow X$, we define the function
$\semE{e} : \polyAct{P}{X} \rightarrow \polyAct{Q}{X}$ by induction.
The verification that this gives rise to a pseudotransformation can be found in the formalization.
\end{construction}

\begin{definition}
\label{def:pathalg}
Let $\sign$ be a signature.
We use Examples \ref{ex:disp_depprod} and \ref{ex:DCell} to define displayed bicategories over $\prealg{\sign}$ and $\prealgG{\sign}$.
\[
\pathalgMD(\sign) = \depprod{(i : \pathlabel[\sign])}{\DCell(\semE{\pathleft[\sign](i)},\semE{\pathright[\sign](i)})}
\]
\[
\pathalgGD{\sign} = \depprod{(i : \pathlabel[\sign])}{\DCell(\semEG{\pathleft[\sign](i)},\semEG{\pathright[\sign](i)})}
\]
We define $\pathalgM(\sign)$ and $\pathalgG{\sign}$ to be the total bicategories of $\pathalgMD(\sign)$ and $\pathalgGD{\sign}$ respectively.
Objects of $\pathalgM(\sign)$ and $\pathalgG{\sign}$ are called \fat{path algebras} for $\sign$.
\end{definition}

\begin{problem}
\label{prob:sem_homendpoint}
Suppose that we have a homotopy endpoint $h : \homotendpoint{l}{r}{a_1}{a_2}{s_1}{s_2}$.
Given a 1-type $X$ with $c : \polyAct{A}{X} \rightarrow X$ and $p : \depprod{(j : J) (x : \polyAct{Q_j}{X})}{l_j(x) = r_j(x)}$,
to construct for each $x : \polyAct{Q}{X}$ and $w : \semE{a_1}(x) = \semE{a_2}(x)$
an equality $\semH{h}(x, w) : \semE{s_1}(x) = \semE{s_2}(x)$.

In addition, given a groupoid $G$ together with a functor $c : \semPG{A}{(G)} \rightarrow G$ and for each $j : J$ a natural transformation $\semEG{l_j}(G) \twocell \semEG{r_j}(G)$,
to construct for each object $x : \semPG{Q}{(G)}$ and morphism $w : \semEG{a_1}(G)(x) \rightarrow \semEG{a_2}(G)(x)$
a morphism $\semHG{h}(x, w) : \semEG{s_1}(G)(x) \rightarrow \semEG{s_2}(G)(x)$.
\end{problem}

\begin{construction}{prob:sem_homendpoint}
By induction.
\end{construction}

\begin{definition}
\label{def:bicat_grpd}
Let $\sign$ be a HIT signature.
We define $\algM(\sign)$ to be the full subbicategory of $\pathalgM(\sign)$
in which every object $X$ satisfies
\begin{equation*}
\begin{split}
\prod
(j : \homotlabel[\sign]) &(x : \polyAct{\homotpointarg[\sign]_j}{X}) (w : \semE{\pathargleft[\sign]_j}(x) = \semE{\pathargright[\sign]_j}(x)),\\
&\semH{\homotleft[\sign]_j}(x, w) = \semH{\homotright[\sign]_j}(x, w)
\end{split}
\end{equation*}
In addition, we define $\algG{\sign}$ to be the full subbicategory of $\pathalgG{\sign} $
in which every object $X$ satisfies
\begin{equation*}
\begin{split}
\prod
(j : \homotlabel[\sign]) &(x : \semPG{\homotpointarg[\sign]_j}{X}) (w : \semEG{\pathargleft[\sign]_j}(x) = \semEG{\pathargright[\sign]_j}(x)),\\
&\semHG{\homotleft[\sign]_j}(x, w) = \semHG{\homotright[\sign]_j}(x, w)
\end{split}
\end{equation*}
Objects of $\algM(\sign)$ and $\algG{\sign}$ are called \fat{algebras} for $\sign$.
\end{definition}

The bicategory $\algM(\sign)$ is constructed by repeatedly using Definition \ref{def:totalbicat}.
By unpacking the definition, we see that an algebra $X : \algM(\sign)$ consists of
\begin{itemize}
	\item A 1-type $X$;
	\item A function $\AlgPoint{X} : \polyAct{\pointconstr}{X} \rightarrow X$;
	\item For each $j : \pathlabel$ and point $x : \polyAct{\patharg_j}{X}$ a path
	$
	\AlgPath{X}{j}(x) : \semE{\pathleft_j}(x) = \semE{\pathright_j}(x);
	$
	\item For each $j : \homotlabel$, $x : \polyAct{\homotpointarg}{X}$ and $w : \semE{a_1}(x) = \semE{a_2}(x)$, a homotopy
	$
	\AlgHomot{X}{j} : \semH{\homotleft}(x, w) = \semH{\homotright}(x, w)
	$
\end{itemize}
