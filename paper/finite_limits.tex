This section is dedicated to the construction of finite limits in the
bicategory $\alg(\sign)$ of algebras in 1-types for the signature
$\sign$.  Finite limits in a bicategory include the terminal object,
binary products, inserters and equifiers
\cite{power1991characterization}. In this section we discuss the three
latter kinds of limits.

\subsection{Products}

Binary products in a bicategory generalize the notion of binary
product in a category.

\begin{definition}
Let $\B$ be a bicategory and let $A$ and $B$ be two objects of
$\B$. The \emph{product} of $A$ and $B$ is given by an object $\prodB
A B$ together with 1-cells $\projlB : \prodB A B \onecell A$ and
$\projrB : \prodB A B \onecell B$.

The triple $(\prodB A B,\projlB, \projrB)$ must satisfy the following
universal property. Assume given an object $X$ and 1-cells $f : X
\onecell A$ and $g : X \onecell B$. Then there exist a 1-cell
$\mappair f g : X \onecell \prodB A B$ and two 2-cells $\theta :
\mappair f g \cdot \projlB \twocell f$ and $\theta' : \mappair f g
\cdot \projrB \twocell g$.

Moreover, given two 1-cells $h_1,h_2 : X \onecell \prodB A B$ with
2-cells $\theta_i : h_i \cdot \projlB \twocell f$ and $\theta'_i : h_i
\cdot \projrB \twocell g$, for $i= 1,2$, there exists a unique
2-cell $\tau : h_1 \twocell h_2$ such that $\tau \whiskerr
\projlB \vcomp \theta_2 = \theta_1$ and $\tau \whiskerr \projrB \vcomp
\theta'_2 = \theta'_1$.
\end{definition} 

Here is how to construct the binary product of algebras in
$\alg(\sign)$.

\begin{definition}
Let $A$ and $B$ be two algebras for $\sign$. The product of $A$ and
$B$ consists of the following data:
\begin{itemize}
\item the carrier is $A \times B$, the product of the carriers of $A$ and $B$;
\item the function $\AlgPoint{A \times B} : \polyAct{\pointconstr}{A \times B} \rightarrow A \times B$ is
\[
\AlgPoint{A \times B}(x) \eqdef (\AlgPoint{A}(\polyAct{\pointconstr}{\projl}(x)),\AlgPoint{B}(\polyAct{\pointconstr}{\projr}(x)))
\]
\end{itemize}
\end{definition}

\begin{proposition}
Product satisfies the required universal property
\end{proposition}

\subsection{Inserter}

\begin{definition}
Inserter in a bicategory
\end{definition}

\begin{definition}
The inserter of algebras
\end{definition}

\begin{proposition}
Inserter satisfies the required universal property
\end{proposition}

\subsection{Equifier}

\begin{definition}
Equifier in a bicategory
\end{definition}

\begin{definition}
The equifier of algebras
\end{definition}

\begin{proposition}
Equifier satisfies the required universal property
\end{proposition}
