From Theorem \ref{thm:initial_alg_sem} we know that initiality implies the induction principle.
Hence, it suffices to construct a biinitial object in the bicategory of algebras in 1-types.
By Proposition \ref{prop:biinitial_in_grpd}, it suffices to construct a biinitial object in $\algG{\sign}$.
To do so, we adapt the semantics by Dybjer and Moeneclaey to our setting \cite{DBLP:journals/entcs/DybjerM18}.

\begin{problem}
\label{prob:initial_grpd_alg}
Given a signature $\sign$, to construct a biinitial object  $\constfont{G}$ in $\algG{\sign}$.
\end{problem}

\begin{construction}{prob:initial_grpd_alg}
\label{constr:initial_grpd_alg}
We only discuss how the carrier $G$ of $\constfont{G}$ is defined.
\begin{itemize}
	\item Note  that each polynomial $P$ gives rise to a container $\hat{P}$.
	Note that each container induces a W-type \cite{abbott2003categories},
	and we define the type of objects of $G$ to be the W-type induced by $\hat{\pointconstr}$.
	Denote this type by $\initob$ and let $\constfont{c}^{\initob} : \pointconstr(\initob) \to \initob$ its algebra map.
	\item The morphisms of $G$ are constructed as a set quotient.
	First, we define an inductive type $\initmorgen{P}{x}{y}$, for each polynomial $P :\poly$ and elements $x, y : P (\initob)$. Its constructors are given in Figure \ref{fig:initmor}.
        When $P$ is $\idP$ and $x,y : \initob$, we write $\initmor{x}{y}$ instead of $\initmorgen{\idP}{x}{y}$. Afterwards, for each $x, y : \initob$ and $f, g : \initmor{x}{y}$, we define a type $\initeq{f}{g}$.
	Both of these are defined as an inductive type and for the constructors, we refer the reader to the formalization.
	Basically, the constructors for these types are chosen in such a way that the groupoid being defined here, has the desired structure.
	This means we add constructors witnessing the path constructors, identity, composition, and all other laws.
	We use the same idea to define $\initeq{f}{g}$.
	
	Note that the input of the quotient is an equivalence relation, which is valued in propositions.
	For this reason, we define $\initeqprop{f}{g}$ to be the propositional truncation of $\initeq{f}{g}$.
	All in all, we define the morphisms from $x$ to $y$ to be the set quotient of $\initmor{x}{y}$ by $\approx_p$.
	\qedhere	
\end{itemize}
\end{construction}

\begin{figure*}[t]
\begin{center}
\begin{bprooftree}
\AxiomC{$P : \poly$}
\AxiomC{$x : P(\initob)$}
\BinaryInfC{$\initmorid{x} : \initmorgen{P}{x}{x}$}
\end{bprooftree}
\begin{bprooftree}
\AxiomC{$P : \poly$}
\AxiomC{$x,y : P(\initob)$}
\AxiomC{$f : \initmorgen{P}{x}{y}$}
\TrinaryInfC{$\initmorinv{f} : \initmorgen{P}{y}{x}$}
\end{bprooftree}
\end{center}

\vspace{5pt}

\begin{center}
\begin{bprooftree}
\AxiomC{$P : \poly$}
\AxiomC{$x,y,z : P(\initob)$}
\AxiomC{$f : \initmorgen{P}{x}{y}$}
\AxiomC{$g : \initmorgen{P}{y}{z}$}
\QuaternaryInfC{$\initmorcomp{f}{g} : \initmorgen{P}{x}{z}$}
\end{bprooftree}
\end{center}

\vspace{5pt}

\begin{center}
\begin{bprooftree}
\AxiomC{$P,Q:\poly$}
\AxiomC{$x,y : P(\initob)$}
\AxiomC{$f : \initmorgen{P}{x}{y}$}
\TrinaryInfC{$\initmorinl{f} : \initmorgen{P+Q}{\inl(x)}{\inl(y)}$}
\end{bprooftree}
\begin{bprooftree}
\AxiomC{$P,Q:\poly$}
\AxiomC{$x,y : Q(\initob)$}
\AxiomC{$f : \initmorgen{Q}{x}{y}$}
\TrinaryInfC{$\initmorinr{f} : \initmorgen{P+Q}{\inr(x)}{\inr(y)}$}
\end{bprooftree}
\end{center}

\vspace{5pt}

\begin{center}
\begin{bprooftree}
\AxiomC{$P,Q : \poly$}
\AxiomC{$x,y : P(\initob)$}
\AxiomC{$w,z : Q(\initob)$}
\AxiomC{$f : \initmorgen{P}{x}{y}$}
\AxiomC{$g : \initmorgen{Q}{w}{z}$}
\QuinaryInfC{$\initmorpair{f}{g} : \initmorgen{P\times Q}{\pair{x}{w}}{\pair{y}{z}}$}
\end{bprooftree}
\end{center}

\vspace{5pt}

\begin{center}
\begin{bprooftree}
\AxiomC{$j : \pathlabel$}
\AxiomC{$x : \patharg_j(\initob)$}
\BinaryInfC{$\initmorpath{j,x} : \initmorgen{\idP}{\semE{\pathleft_j}(x)}{\semE{\pathright_j}(x)}$}
\end{bprooftree}
\begin{bprooftree}
\AxiomC{$x,y : \pointconstr(\initob)$}
\AxiomC{$f : \initmorgen{\pointconstr}{x}{y}$}
\BinaryInfC{$\initmorap{f} : \initmorgen{\idP}{\constfont{c}^{\initob}(x)}{\constfont{c}^{\initob}(y)}$}
\end{bprooftree}
\end{center}
\caption{Rules for the type $\initmorgen{P}{x}{y}$.}
\label{fig:initmor}
\end{figure*}

\begin{figure*}[t]
\begin{center}
\begin{bprooftree}
\AxiomC{$f : \initmorgen{P}{x}{y}$}
\UnaryInfC{$\initeqgen{P}{f}{f}$}
\end{bprooftree}
\begin{bprooftree}
\AxiomC{$f,g : \initmorgen{P}{x}{y}$}
\AxiomC{$\initeqgen{P}{f}{g}$}
\BinaryInfC{$\initeqgen{P}{g}{f}$}
\end{bprooftree}
\begin{bprooftree}
\AxiomC{$f,g,h : \initmorgen{P}{x}{y}$}
\AxiomC{$\initeqgen{P}{f}{g}$}
\AxiomC{$\initeqgen{P}{g}{h}$}
\TrinaryInfC{$\initeqgen{P}{f}{h}$}
\end{bprooftree}
\end{center}

\vspace{5pt}

\begin{center}
\begin{bprooftree}
\AxiomC{$\initeqgen{\idP}{\initmorap{\initmorid{x}}}{\initmorid{\constfont{c}^{\initob}(x)}}$}
\end{bprooftree}
\begin{bprooftree}
\AxiomC{$f : \initmorgen{\pointconstr}{x}{y}$}
\AxiomC{$g : \initmorgen{\pointconstr}{y}{z}$}
\BinaryInfC{$\initeqgen{\idP}{\initmorap{\initmorcomp{f}{g}}}{\initmorcomp{\initmorap{f}}{\initmorap{g}}}$}
\end{bprooftree}
\end{center}

\vspace{5pt}

\begin{center}
\begin{bprooftree}
\AxiomC{$\initeqgen{P+Q}{\initmorinl{\initmorid{x}}}{\initmorid{\inl{x}}}$}
\end{bprooftree}
\begin{bprooftree}
\AxiomC{$\initeqgen{P+Q}{\initmorinr{\initmorid{x}}}{\initmorid{\inr{x}}}$}
\end{bprooftree}
\end{center}

\vspace{5pt}

\begin{center}
\begin{bprooftree}
\AxiomC{$\initeqgen{P}{f}{g}$}
\UnaryInfC{$\initeqgen{P+Q}{\initmorinl{f}}{\initmorinl{g}}$}
\end{bprooftree}
\begin{bprooftree}
\AxiomC{$f : \initmorgen{P}{x}{y}$}
\AxiomC{$g : \initmorgen{P}{y}{z}$}
\BinaryInfC{$\initeqgen{P + Q}{\initmorinl{\initmorcomp{f}{g}}}{\initmorcomp{\initmorinl{f}}{\initmorinl{g}}}$}
\end{bprooftree}
\end{center}

\vspace{5pt}

\begin{center}
\begin{bprooftree}
\AxiomC{$\initeqgen{Q}{f}{g}$}
\UnaryInfC{$\initeqgen{P+Q}{\initmorinr{f}}{\initmorinr{g}}$}
\end{bprooftree}
\begin{bprooftree}
\AxiomC{$f : \initmorgen{Q}{x}{y}$}
\AxiomC{$g : \initmorgen{Q}{y}{z}$}
\BinaryInfC{$\initeqgen{P + Q}{\initmorinr{\initmorcomp{f}{g}}}{\initmorcomp{\initmorinr{f}}{\initmorinr{g}}}$}
\end{bprooftree}
\end{center}


\caption{Rules for the type $\initeqgen{P}{f}{g}$.}
\label{fig:initmor}
\end{figure*}


\begin{problem}
\label{prob:hit_exist}
Each signature has a HIT.
\end{problem}

\begin{construction}{prob:hit_exist}
\label{constr:hit_exist}
By Propositions \ref{thm:initial_alg_sem} and \ref{prop:biinitial_in_grpd}, it suffices to find a biinitial object in $\algG{\sign}$.
The desired object is given in Construction \ref{constr:initial_grpd_alg}.
\end{construction}
