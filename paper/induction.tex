The algebra structure only represents the introduction rule
and the next step is to define the elimination and computation rules for higher inductive types.
Before we can formulate these principles, we need to define dependent actions of polynomials, path endpoints, and homotopy endpoints.
All of these constructions are done by induction and details can be found in the literature \cite{DBLP:journals/entcs/DybjerM18,hermida1998structural,van2019construction}.

\begin{problem}
\label{prob:poly_dact}
Given a type $X$, a type family $Y$ on $X$, and a polynomial $P$,
to construct a type family $\polyDact{P}{Y}$ on $\polyAct{P}{X}$.
\end{problem}

\begin{problem}
\label{prob:poly_dmap}
Given a type $X$, a type family $Y$ on $X$, a polynomial $P$, and a map $f : \depprod{(x : X)}{Y(x)}$,
to construct a map
$
\polyDmap{P}{f} : \depprod{(x : \polyAct{P}{X})}{\polyDact{P}{Y}(x)}.
$
\end{problem}

\begin{problem}
\label{prob:poly_pr}
Given a type $X$, a type family $Y$ on $X$, and a polynomial $P$, to construct a map
$
\polyoplax P : \polyAct{P}{\Sum (x : X). Y(x)} \rightarrow \Sum (x : \polyAct{P}{X}). \polyDact{P}{Y}(x)
$
such that for all $z :  \polyAct{P}{\Sum (x : X). Y(x)}$, we have $\projl(\polyoplax{P}(z)) = \polyAct{P}{\projl}(z)$.
\end{problem}

\begin{problem}
\label{prob:endpoint_dact}
Given a type $X$, a type family $Y$ on $X$, an endpoint $e : \pathendpoint{A}{P}{Q}$, and a map $c : \polyAct{A}{X} \rightarrow X$,
to construct for each $x : \polyAct{P}{X}$ and $y : \polyDact{P}{Y}(x)$ an inhabitant $\pathendpointDact{e}{y} : \polyDact{Q}{Y}(\pathendpointAct{e}{x})$.
\end{problem}

\begin{problem}
\label{prob:endpoint_dact_natural}
Suppose, that we have polynomials $A, P, Q$, a type $X$ with a map $c_X : \polyAct{A}{X} \rightarrow X$,
and a type family $Y$ on $X$ with a map $c_Y : \depprod{(x : X)}{\polyDact{A}{Y}(x) \rightarrow Y(c_X(x))}$
and a map $f : \depprod{(x : X)}{Y(x)}$.
Given an endpoint $e : \pathendpoint{A}{P}{Q}$,
to construct an equality
\[
\pathendpointDnat{e}{f} : \polyDmap{Q}{f}(\semE{e}(x)) = \pathendpointDact{e}{\polyDmap{P}{f}(x)}.
\]
\end{problem}

\begin{problem}
\label{prob:homot_dact}
Let $\sign$ be a signature.
Let $X$ be a type with a function $c_X : \polyAct{\pointconstr}{X} \rightarrow X$
and for each $j : \pathlabel$ and $x : \polyAct{\patharg_j}{X}$ a path $p_X(j, x) : \semE{\pathleft_j}(x) = \semE{\pathright_j}(x)$.
In addition, suppose that $Y$ is a type family on $X$,
that we have a function $c_Y : \depprod{(x : \polyAct{\pointconstr}{X})}{\polyDact{\pointconstr}{Y}(x) \rightarrow Y(c_X(x))}$,
and that for all $j  : \pathlabel$ and points $x : \polyAct{\patharg_j}{X}$ and $\xx : \polyDact{\patharg_j}{Y}(x)$,
we have a path $p_Y : \pathover{p_X(j, x)}{\pathendpointDact{\pathleft_j}{\xx}}{\pathendpointDact{\pathright_j}{\xx}}$.
Furthermore, let $j : \homotlabel$, let $x : \homotpointarg_j(X)$ and $\xx : \polyDact{\homotpointarg_j}{Y}(x)$ be points
and let $w : \semE{\pathargleft_j}(x) = \semE{\pathargright_j}(x)$
and $\disp{w} : \pathover{p}{\pathendpointDact{\pathargleft_j}{\xx}}{\pathendpointDact{\pathargright_j}{\xx}}$
be paths.
Then for each homotopy endpoint $h : \homotendpoint{\pathleft}{\pathright}{\pathargleft_j}{\pathargright_j}{\homotpathleft_j}{\homotpathright_j}$, 
to construct a path
\[
\homotendpointDact{h}{\xx, \disp{w}} : \pathover{\homotendpointAct{h}{x, w}}{\pathendpointDact{\homotpathleft_j}{\xx}}{\pathendpointDact{\homotpathright_j}{\xx}}.
\]
\end{problem}

With these notions in place, we define \emph{displayed algebras}.
These represent the input of the elimination rule.
Furthermore, we show that each displayed algebra gives rise to a total algebra and a projection.

\begin{definition}\label{def:disp_alg}
Given a signature $\sign$ and an algebra $X$ for $\sign$,
a \fat{displayed algebra} $Y$ over $X$ consists of
\begin{itemize}
	\item A family $Y$ of 1-types over $X$;
	\item For each $x : \polyAct{\pointconstr}{X}$ a map $\DispAlgPoint{Y} : \polyDact{\pointconstr}{Y}(x) \rightarrow Y(\AlgPoint{X}(x))$;
	\item For each $j : \pathlabel$, $x : \polyAct{\patharg_j}{X}$, and $\pover{x} : \polyDact{\patharg_j}{Y}(x)$, a path
	$
	\DispAlgPath{Y}{j} : \pathover{\AlgPath{X}{j}{x}}{\pathendpointDact{\pathleft_j}{\pover{x}}}{\pathendpointDact{\pathright_j}{\pover{x}}}
	$;
	\item For each $j : \homotlabel$, points $x : \polyAct{\homotpointarg_j}{X}$ and $\pover{x} : \polyDact{\homotpointarg_j}{Y}(x)$,
	and paths $w : \pathargleft_j(x) = \pathargright_j(x)$ and $\pover{w} : \pathover{w}{\pathendpointDact{\pathargleft_j}{\pover{x}}}{\pathendpointDact{\pathargright_j}{\pover{x}}}$,
	a globe
	$
	\DispAlgHomot{Y}{j} : \globeover{\AlgHomot{X}{j}{(x, w)}}{\homotendpointDact{\homotleft_j}{\xx}}{\homotendpointDact{\homotright_j}{\xx}}
	$
	over $\AlgHomot{X}{j}{(x, w)}$.
\end{itemize}
\end{definition}

\begin{remark}
The type family of a displayed algebra is required to be 1-truncated.
This means that the HITs we construct, can only be eliminated into
1-types, and as a consequence, these HITs only have the right
elimination principle with respect to 1-types.
\end{remark}

\begin{problem}
\label{prob:total_alg}
Given an algebra $X$ for a signature $\sign$ and a displayed algebra $Y$ over $X$,
to construct an algebra $\TotalAlg{Y}$ and a morphism $\pi_1 : \TotalAlg{Y} \onecell X$.
\end{problem}

\begin{construction}{prob:total_alg}\label{constr:total_alg}
We only discuss the carrier and the point constructor of $\TotalAlg{Y}$.
The carrier of $\TotalAlg{Y}$ is defined to be $\sum (x : X), Y(x)$
and the function $\AlgPoint{\TotalAlg Y}$ acts on elements $z : \polyAct{\pointconstr}{\sum (x : X), Y(x)}$ as follows:
\[
\AlgPoint{\TotalAlg Y}(z) \eqdef
(\AlgPoint{X}(\projl(\polyoplax{\pointconstr}(z)))
,
\DispAlgPoint{Y}(\projr(\polyoplax{\pointconstr}(z)))).
\]
The underlying map of the morphism $\pi_1 : \TotalAlg{Y} \onecell X$ takes the first projection of a pair.
\end{construction}

We call $\TotalAlg{Y}$ the \emph{total algebra} of $Y$ and the morphism $\pi_1$ is called the \emph{first projection}.
The output of the elimination rule and the computation rules are given by a \emph{section} to be defined in Definition \ref{def:section} below.
One might expect that, just like for the groupoid quotient, the computation rules for the paths
are given as globes over some 2-path in the base (Definition \ref{def:globe_over}).
However, this is not the case.

This is because there is a slight discrepancy between the rules for the groupoid quotient and the HITs we discuss,
namely for the former the computation rules for the points are definitional equalities
while for the latter, these rules only hold propositionally.
This affects how we need to formulate the computation rules for the paths.

Let us illustrate this via the torus (Example \ref{ex:torus}).
The input for the elimination rule consists, among others, of a type family $Y$, a point $b : Y(\base)$,
and a path $p_l : \pathover{\leftLoop}{b}{b}$.
The elimination rule gives a map $f : \depprod{(x : \torus)}{Y(x)}$.
By the point computation rule, we have a propositional equality between $f(\base)$ and $b$.
Now the computation rule for $\leftLoop$ ought to equate $\apd{f}{\leftLoop}$ and $p_l$.
However, such an equation does not type check since $\apd{f}{\leftLoop}$ has type $\pathover{\leftLoop}{f(b)}{f(b)}$ while $p_l$ has type $\pathover{\leftLoop}{b}{b}$.
In conclusion, we cannot formulate the computation rules the same way as we did for the groupoid quotient.

Our solution to this problem is to define a type of \emph{squares} over a given 2-path similarly to Definition \ref{def:globe_over}.

\begin{definition}
Let $X$ be a type and let $Y$ be a type family on $X$.
Suppose that we are given points $x_1, x_2 : X$ and $\overline{x_1}, \overline{x_1}' : Y(x_1)$ and $\overline{x_2}, \overline{x_2}' : Y(x_2)$, paths $p, q : x_1 = x_2$ together with paths $\pover{p} : \pathover{p}{\overline{x_1}}{\overline{x_2}}$ and $\pover{q} : \pathover{q}{\overline{x_1}'}{\overline{x_2}'}$ over $p$ and $q$ respectively.
If we also have two paths $h_1 : \overline{x_1} = \overline{x_1}'$ and $h_2 : \overline{x_2} = \overline{x_2}'$ and a 2-path $g : p = q$,
then we define the type of \fat{squares} over $g$ from $\pover{p}$ to $\pover{q}$ with sides $h_1$ and $h_2$ by path induction on $g$.
\end{definition}

\begin{definition}
\label{def:section}
Let $X$ be an algebra for a given signature $\sign$ and let $Y$ be a displayed algebra over $X$.
Then a \fat{section} of $Y$ consists of
\begin{itemize}
	\item A map $f : \depprod{(x : X)}{Y(x)}$;
	\item For all $x : \polyAct{\pointconstr}{X}$, an equality $f(\AlgPoint{X}(x)) = \DispAlgPoint{Y}(\polyDmap{\pointconstr}{f}(x))$;
	\item For all $j : \pathlabel$ and $x : \polyAct{\patharg}{X}$, a square from
	$\apd{f}{(\AlgPath{X}{j}(x))}$
	to
	$\DispAlgPath{Y}{j}(\polyDmap{\patharg}{f}(x))$
	with sides
	$\pathendpointDnat{\pathleft_j}{f}(x)$
	and
	$\pathendpointDnat{\pathright_j}{f}(x)$.
\end{itemize}
\end{definition}

\begin{definition}
Let $\sign$ be a signature and let $X$ be an algebra for $\sign$.
Then we say that $X$ is a \fat{1-truncated higher inductive type} for $\sign$ if each displayed algebra $Y$ over $X$ has a section.
\end{definition}

Often we just say that $X$ is a HIT for $\sign$ instead of saying that $X$ is 1-truncated HIT.
With this in place, we can check whether our rules for higher inductive types
agree for the usual examples with the rules given in the literature \cite{hottbook}.
We illustrate this with the torus (Example \ref{ex:torus}) and the set truncation (Example \ref{ex:settrunc}).
In the next example, we write $p \vcomp q$ for the concatenation of dependent paths.

\begin{example}[Example \ref{ex:torus} cont'd]
Recall the signature $\torus$ for the torus.
Let $X$ be a HIT for $\torus$.
Since $X$ is an algebra, we have a point $\base : X$, two paths $\leftLoop, \rightLoop : \base = \base$,
and a 2-path $\surface : \concat{\leftLoop}{\rightLoop} = \concat{\rightLoop}{\leftLoop}$.
This corresponds precisely to the usual introduction rules of the torus.

A family $Y$ of 1-types on $X$ together with a point $b : Y(\base)$,
two paths $l : \pathover{\leftLoop}{b}{b}$ and $r : \pathover{\rightLoop}{b}{b}$
and a globe $h : \globeover{\surface}{l \vcomp r}{r \vcomp l}$
over $\surface$ gives rise to a displayed algebra over $X$.
This corresponds to the usual input of the elimination rule of the torus.
If we have a section $s$ of $Y$, then in particular, we get a map $f_s : \depprod{(x : X)}{Y(x)}$.
We also get a path $p_s : f(\base) = b$, a square from $\apd{f}{\leftLoop}$ to $l$ and one from $\apd{f}{\rightLoop}$ to $r$.
Both squares have sides $p_s$ and $p_s$.
These are the computation rules for the points and paths of the torus.
Note that since we are looking at $1$-truncated HITs, this only gives the 1-truncation of the torus.
\end{example}

\begin{example}[Example \ref{ex:settrunc} cont'd]
Let $A$ be a 1-type and recall the signature $\ST{A}$.
Now let $X$ be a HIT on $\ST{A}$.
Note that an algebra for $\ST{A}$ consists of a type $Z$ together with a map $A \rightarrow Z$
and a proof that $Z$ is a set.
This means in particular, that we have a map $\SC : A \rightarrow X$ and a proof $\Strunc$
that $X$ is a set.

A family $Y$ of sets on $X$ together with a map $i : \depprod{(a  : A)}{Y(\SC(A))}$
give rise to a displayed algebra over $X$.
A section $s$ of that displayed algebra consists of a map $f_s : \depprod{(x : X)}{Y(x)}$
such that $f_s(\SC(a)) = i(a)$ for all $a : A$.
This corresponds to the usual elimination and computation rules for the
set truncation.
\end{example}

To verify that an algebra satisfies the elimination rule, we use \emph{initial algebra semantics} \cite{hermida1998structural}.
However, this technique is usually applied in a categorical setting and it uses initial objects in categories.
Since we work in a bicategorical setting, we need to use the corresponding notion in bicategory theory: \emph{biinitiality}.

\begin{definition}
Let $\B$ be a bicategory and let $x$ be an object in $\B$.
Then we say $x$ is \fat{biinitial} if
\begin{itemize}
	\item For each object $y$ there is a 1-cell $x \onecell y$;
	\item Given 1-cells $f, g : x \onecell y$, there is a 2-cell $f \twocell g$;
	\item Given 2-cells $\tc, \tc' : f \twocell g$, there is an equality $\tc = \tc'$.
\end{itemize}
\end{definition}

Briefly, an object $x$ is biinitial if for each $y$ there is a 1-cell from $x$ to $y$, which is unique up to a unique 2-cell.
Now we can formulate initial algebra semantics for our signatures.

\begin{proposition}
\label{thm:initial_alg_sem}
Let $\sign$ be a signature and let $X$ be an algebra for $\sign$.
Then
\begin{itemize}
	\item If $X$ is a 1-truncated HIT for $\sign$, then $X$ is biinitial in $\algM(\sign)$.
	\item If $X$ is biinitial, then $X$ is a 1-truncated HIT for $\sign$.
\end{itemize}
\end{proposition}

One consequence of initial algebra semantics, is that HITs are unique up to path equality if the univalence axiom holds.
This result is a consequence of the fact that the bicategory of algebras is \emph{univalent}.
Recall that a bicategory is univalent if
equality between objects $X$ and $Y$ is equivalent to adjoint equivalences between $X$ and $Y$
and equality of 1-cells $f$ and $g$ is equivalent to invertible 2-cells between $f$ and $g$
\cite{bicatjournal}.
Using the methods employed by Ahrens \etal \ one can show that the bicategory of algebras
is univalent.
Since biinitial objects are unique up to adjoint equivalence, one can conclude that HITs are unique up
to path equality.

\begin{proposition}
Let $\sign$ be a signature and let $H_1$ and $H_2$ be HITs for $\sign$.
Denote the underlying algebras of $H_1$ and $H_2$ by $X_1$ and $X_2$.
Then $X_1 = X_2$.
\end{proposition}
