\documentclass[screen]{acmart}
\settopmatter{}

\copyrightyear{2020} 
\acmYear{2020} 
\setcopyright{rightsretained} 
\acmISBN{978-1-4503-7104-9/20/07}
\acmDOI{10.1145/3373718.3394803}
\acmConference[LICS '20]{Proceedings of the 35th Annual ACM/IEEE Symposium on Logic
  in Computer Science (LICS)}{July 8--11, 2020}{Saarbr\"ucken, Germany}
\acmBooktitle{Proceedings of the 35th Annual ACM/IEEE Symposium on Logic in
  Computer Science (LICS '20), July 8--11, 2020, Saarbr\"ucken, Germany}
\startPage{1}

%% Copyright information
%% Supplied to authors (based on authors' rights management selection;
%% see authors.acm.org) by publisher for camera-ready submission;
%% use 'none' for review submission.
%\setcopyright{none}
%\setcopyright{acmcopyright}
%\setcopyright{acmlicensed}
\setcopyright{rightsretained}
%\copyrightyear{2018}           %% If different from \acmYear

%% Bibliography style
\bibliographystyle{ACM-Reference-Format}
%% Citation style
%\citestyle{acmauthoryear}  %% For author/year citations
%\citestyle{acmnumeric}     %% For numeric citations
%\setcitestyle{nosort}      %% With 'acmnumeric', to disable automatic
                            %% sorting of references within a single citation;
                            %% e.g., \cite{Smith99,Carpenter05,Baker12}
                            %% rendered as [14,5,2] rather than [2,5,14].
%\setcitesyle{nocompress}   %% With 'acmnumeric', to disable automatic
                            %% compression of sequential references within a
                            %% single citation;
                            %% e.g., \cite{Baker12,Baker14,Baker16}
                            %% rendered as [2,3,4] rather than [2-4].


%%%%%%%%%%%%%%%%%%%%%%%%%%%%%%%%%%%%%%%%%%%%%%%%%%%%%%%%%%%%%%%%%%%%%%
%% Note: Authors migrating a paper from traditional SIGPLAN
%% proceedings format to PACMPL format must update the
%% '\documentclass' and topmatter commands above; see
%% 'acmart-pacmpl-template.tex'.
%%%%%%%%%%%%%%%%%%%%%%%%%%%%%%%%%%%%%%%%%%%%%%%%%%%%%%%%%%%%%%%%%%%%%%


%% Some recommended packages.
\usepackage{booktabs}   %% For formal tables:
                        %% http://ctan.org/pkg/booktabs
\usepackage{subcaption} %% For complex figures with subfigures/subcaptions
                        %% http://ctan.org/pkg/subcaption
\usepackage{hyperref}
\usepackage{amsmath}
\usepackage{amsthm}
\usepackage{amssymb}
\usepackage{mathtools}
\usepackage[utf8]{inputenc}
\usepackage{bussproofs}
\usepackage{listings}
\usepackage{coq}
\usepackage{color}
\usepackage{xcolor}
\usepackage{stmaryrd}

\usepackage{xspace}
\usepackage{xifthen}

\usepackage[all,cmtip]{xy}

\theoremstyle{plain}
\newtheorem{theorem}{Theorem}[section]
\newtheorem{problem}[theorem]{Problem}
\newtheorem{proposition}[theorem]{Proposition}

\theoremstyle{definition}
\newtheorem{constrInternal}[theorem]{Construction}

\newenvironment{construction}[2][]
{\pushQED{\qed}\begin{constrInternal}[{for Problem~\ref{#2}\ifthenelse{\isempty{#1}}{}{; #1}}]}
	{\popQED\end{constrInternal}}

\newtheorem{definition}[theorem]{Definition}
\newtheorem{example}[theorem]{Example}
\newtheorem{axiom}[theorem]{Axiom}
\newtheorem{remark}[theorem]{Remark}

\newcommand{\cL}{{\cal L}}

\newenvironment{bprooftree}
{\leavevmode\hbox\bgroup}
{\DisplayProof\egroup}

% general
\newcommand{\ie}{\emph{i.e.,} }
\newcommand{\cf}{\emph{c.f.} }
\newcommand{\eg}{\emph{e.g.,} }
\newcommand{\etal}{\emph{et~al.}}
\newcommand{\etc}{\emph{etc.}}
\newcommand{\wrt}{w.r.t.~}
\newcommand{\resp}{resp.~}
\newcommand{\loccit}{\emph{loc. cit.} }

\newcommand{\fat}[1]{\textbf{#1}}

% math
\newcommand{\eqdef}{:\equiv}

% for bicategories
\newcommand{\constfont}[1]{\ensuremath{\mathsf{#1}}}
\newcommand{\cat}[1]{\ensuremath{\constfont{#1}}\xspace}

\newcommand{\tc}{\theta}
\newcommand{\tcB}{\gamma}
\newcommand{\tcC}{\tau}

\newcommand{\lunitor}[1]{\lambda}
\newcommand{\runitor}[1]{\rho}
\newcommand{\linvunitor}[1]{\lambda^{-1}}
\newcommand{\rinvunitor}[1]{\rho^{-1}}
\newcommand{\lassoc}[3]{\alpha}
\newcommand{\rassoc}[3]{\alpha^{-1}}

%\newcommand{\lunitor}[1]{\lambda(#1)}
%\newcommand{\runitor}[1]{\rho(#1)}
%\newcommand{\linvunitor}[1]{\lambda(#1)^{-1}}
%\newcommand{\rinvunitor}[1]{\rho(#1)^{-1}}
%\newcommand{\lassoc}[3]{\alpha(#1, #2, #3)}
%\newcommand{\rassoc}[3]{\alpha(#1, #2, #3)^{-1}}
\newcommand{\modvar}{\Gamma}
\newcommand{\identitor}[1]{{#1}_i}
\newcommand{\compositor}[1]{{#1}_c}

\newcommand{\iso}{\cong}
\newcommand{\adjequiv}{\simeq}
\newcommand{\invtwocell}{\constfont{inv2cell}}
\newcommand{\id}{\operatorname{id}}
\newcommand{\vcomp}{\bullet}
\newcommand{\whiskerl}{\vartriangleleft}
\newcommand{\whiskerr}{\vartriangleright}
\newcommand{\onecell}{\rightarrow}
\newcommand{\twocell}{\Rightarrow}

\newcommand{\B}{\cat{B}}
\renewcommand{\C}{\cat{C}}
\newcommand{\D}{\cat{D}}
\newcommand{\E}{\cat{E}}
\newcommand{\onetypes}{1\mbox{-}\cat{Type}}
\newcommand{\grpd}{\cat{Grpd}}

\newcommand{\ap}[2]{\constfont{ap} \ #1 \ #2}
\newcommand{\apd}[2]{\constfont{apd} \ #1 \ #2}

\newcommand{\function}{\rightarrow}

% displayed stuff
\newcommand{\disp}[1]{\overline{#1}}
\newcommand{\widedisp}[1]{\overline{#1}}
\renewcommand{\aa}{\disp{a}}
\newcommand{\aaa}{\widedisp{a'}}
\newcommand{\bb}{\disp{b}}
\newcommand{\cc}{\disp{c}}
\newcommand{\dd}{\disp{d}}
\newcommand{\ff}{\disp{f}}
\newcommand{\fff}{\widedisp{f'}}
\renewcommand{\gg}{\disp{g}}
\newcommand{\hh}{\disp{h}}
\newcommand{\rr}{\disp{r}}
\newcommand{\xx}{\disp{x}}
\newcommand{\yy}{\disp{y}}
\newcommand{\zz}{\disp{z}}
\newcommand{\dtc}{\disp{\tc}}
\newcommand{\dtcB}{\disp{\tcB}}
\newcommand{\dtcC}{\disp{\tcC}}

\newcommand{\FF}{\disp{F}}
\newcommand{\FFo}{\disp{F'}}
\newcommand{\FFt}{\disp{F''}}
\newcommand{\GG}{\disp{G}}
\newcommand{\HH}{\disp{H}}
\newcommand{\LL}{\disp{L}}
\newcommand{\RR}{\disp{R}}
\newcommand{\etaeta}{\disp{\eta}}
\newcommand{\thetatheta}{\disp{\theta}}
\newcommand{\epseps}{\disp{\epsilon}}
\newcommand{\tautaul}{\disp{\tau_1}}
\newcommand{\tautaur}{\disp{\tau_2}}
\newcommand{\mm}{\disp{m}}

\newcommand{\dob}[2]{\ensuremath{{#1}(#2)}} % displayed objects
\newcommand{\dmor}[3]{#1 \xrightarrow{#3} #2} % displayed morphisms
\newcommand{\dtwo}[3]{#1 \xRightarrow{#3} #2} % displayed 2-cells
\newcommand{\total}[2][]{\ensuremath{\textstyle \int_{#1}{#2}}} % total bicategory
\newcommand{\dsigma}[2][]{\ensuremath{\sum_{#1}{#2}}} % displayed sigma
\newcommand{\diso}[3]{\ensuremath{ {#1} \iso_{#2} {#3} }}
\newcommand{\dadjequiv}[3]{\ensuremath{ {#1} \adjequiv_{#2} {#3} }}
\newcommand{\did}{\id}
\newcommand{\dproj}[1]{\pi_{#1}}

% pseudofunctors and so on
\newcommand{\pseudo}{\cat{Pseudo}}
\newcommand{\pstrans}[2]{#1 \twocell #2}
\newcommand{\modif}[2]{#1 \Rrightarrow #2}
\newcommand{\disppsfun}[3]{\dmor{#1}{#2}{#3}}
\newcommand{\disppstrans}[3]{\dtwo{#1}{#2}{#3}}
\newcommand{\dispmodif}[3]{\xymatrix{#1 \ar@3[r]^-{#3} & #2}}

% algebras
\newcommand{\algd}{\cat{Alg_D}}
\newcommand{\alg}{\cat{Alg}}
\newcommand{\addcell}{\cat{Add2Cell}}
\newcommand{\var}{\cat{var}}

% macros with proper spacing
\newcommand{\spac}{\hskip 0.2em plus 0.1em}
\def\Lam #1.{\lambda\,#1.\spac}%
\def\Sum #1.{\sum_{#1}\spac}%
\def\Prod #1.{\prod_{#1}\spac}%

% HoTT
\newcommand{\Type}[0]{\textsc{Type}}
\newcommand{\type}[1]{\operatorname{\textsf{#1}}}
\newcommand{\constructor}[1]{\operatorname{\mathbf{#1}}}

\newcommand{\transport}[0]{\constfont{transport}}
\renewcommand{\transp}[1]{#1_*}
\newcommand{\depeq}[1][*]{=_{#1}}
\newcommand{\defeq}{\equiv}                                                                

\newcommand{\Empty}[0]{\type{Empty}}
\newcommand{\Unit}[0]{\type{Unit}}
\newcommand{\bool}[0]{\type{2}}
\renewcommand{\booltrue}{\constructor{true}}
\renewcommand{\boolfalse}{\constructor{false}}
\newcommand{\hProp}[0]{\type{hProp}}
\newcommand{\hSet}[0]{\type{hSet}}
\renewcommand{\U}[0]{\type{U}}
\newcommand{\unit}[0]{\type{1}}
\newcommand{\unitt}{\constructor{tt}}
\newcommand{\idpath}{\constructor{refl}}
\newcommand{\J}{\constfont{J}}

\newcommand{\functions}[2]{#1 \rightarrow #2}
\newcommand{\depprod}[2]{\prod #1 , #2}

\newcommand{\refl}[1]{\constructor{idpath}(#1)}
\newcommand{\concat}[2]{#1 \bullet #2}
\newcommand{\inverse}[1]{#1^{-1}}

\newcommand{\pathover}[4][]{#3 =_{#2}^{#1} #4}
\newcommand{\globeover}[4][]{#3 =_{#2}^{#1} #4}

\newcommand{\op}[1]{#1^{\constructor{op}}}

\newcommand{\projl}{\pi_1}
\newcommand{\projr}{\pi_2}

\newcommand{\inl}{\constructor{inl}}
\newcommand{\inr}{\constructor{inr}}

\newcommand{\funextsec}{\constfont{funext}}

% variables
\newcommand{\sign}{\Sigma}
\newcommand{\pover}[1]{\overline{#1}}

% Displayed bicategories needed for algebras
\newcommand{\DFAlg}{\constfont{DFalg}}
\newcommand{\DCell}{\constfont{DFcell}}
\newcommand{\FSub}{\constfont{FSub}}

% Signatures
\newcommand{\pointconstr}[1][]{\constfont{A}^{#1}}
\newcommand{\pathlabel}[1][]{\constfont{J}^{#1}_{\constfont{P}}}
\newcommand{\patharg}[1][]{\constfont{S}^{#1}}
\newcommand{\pathleft}[1][]{\constfont{l}^{#1}}
\newcommand{\pathright}[1][]{\constfont{r}^{#1}}
\newcommand{\homotlabel}[1][]{\constfont{J}^{#1}_{\constfont{H}}}
\newcommand{\homotpointarg}[1][]{\constfont{R}^{#1}}
\newcommand{\homotpathtarg}[1][]{\constfont{T}^{#1}}
\newcommand{\pathargleft}[1][]{\constfont{a}^{#1}}
\newcommand{\pathargright}[1][]{\constfont{b}^{#1}}
\newcommand{\homotpathleft}[1][]{\constfont{s}^{#1}}
\newcommand{\homotpathright}[1][]{\constfont{t}^{#1}}
\newcommand{\homotleft}[1][]{\constfont{p}^{#1}}
\newcommand{\homotright}[1][]{\constfont{q}^{#1}}

% Algebras
\newcommand{\prealgM}{\constfont{PreAlg}}
\newcommand{\pathalgM}{\constfont{PathAlg}}
\newcommand{\pathalgMD}{\constfont{DPathAlg}}
\newcommand{\algM}{\constfont{Alg}}
\newcommand{\prealg}[1]{\prealgM(#1)}
\newcommand{\pathalg}[1]{\pathalgM(#1)}
\newcommand{\algebra}[1]{\algM(#1)}
\newcommand{\prealgG}[1]{\prealgM_{\grpd}(#1)}
\newcommand{\pathalgG}[1]{\pathalgM_{\grpd}(#1)}
\newcommand{\pathalgGD}[1]{\pathalgMD_{\grpd}(#1)}
\newcommand{\algG}[1]{\algM_{\grpd}(#1)}

\newcommand{\AlgPoint}[1]{\constfont{c}^{#1}}
\newcommand{\AlgPath}[2]{\constfont{p}^{#1}_{#2}}
\newcommand{\AlgHomot}[2]{\constfont{h}^{#1}_{#2}}

\newcommand{\DispAlgPoint}[1]{\overline{\constfont{c}}^{#1}}
\newcommand{\DispAlgPath}[2]{\overline{\constfont{p}}^{#1}_{#2}}
\newcommand{\DispAlgHomot}[2]{\overline{\constfont{h}}^{#1}_{#2}}

% Groupoids
\newcommand{\mor}[3]{#1(#2, #3)}
\newcommand{\idgrpd}[1]{\id(#1)}
\newcommand{\compgrpd}[2]{#1 \cdot #2}

% Quotient
\newcommand{\quot}[2]{{#1 / #2 }}

% Groupoid quotient
\newcommand{\gquotType}{\type{GQuot}}
\newcommand{\gcl}{\constructor{gcl}}
\newcommand{\gcleq}{\constructor{gcleq}}
\newcommand{\gconcat}{\constructor{gconcat}}
\renewcommand{\ge}{\constructor{ge}}
\newcommand{\gtrunc}{\constructor{gtrunc}}

\newcommand{\gclY}{\constfont{gcl}_Y}
\newcommand{\gcleqY}{\constfont{gcleq}_Y}
\newcommand{\gconcatY}{\constfont{gconcat}_Y}
\newcommand{\geY}{\constfont{ge}_Y}
\newcommand{\gtruncY}{\constfont{gtrunc}_Y}

\newcommand{\gind}{\constructor{gind}}

% Biequivalence between 1-types and groupoids

\newcommand{\pgrpd}{\constfont{PathGrpd}}
\newcommand{\gquot}{\constfont{GQuot}}
\newcommand{\prepgrpd}{\constfont{PathGrpd}_{\prealgM}}
\newcommand{\pregquot}{\constfont{GQuot}_{\prealgM}}
\newcommand{\pathpgrpd}{\constfont{PathGrpd}_{\pathalgM}}
\newcommand{\pathgquot}{\constfont{GQuot}_{\pathalgM}}
\newcommand{\algpgrpd}{\constfont{PathGrpd}_{\algM}}
\newcommand{\alggquot}{\constfont{GQuot}_{\algM}}

% Polynomials
\newcommand{\poly}{\constfont{P}}
\newcommand{\constantP}[1]{\constructor{C}(#1)}
\newcommand{\idP}{\constfont{Id}}
\newcommand{\sumP}[2]{#1 + #2}
\newcommand{\prodP}[2]{#1 \times #2}
\newcommand{\polyAct}[2]{#1(#2)}
\newcommand{\polyDact}[2]{\overline{#1}(#2)}
\newcommand{\polyDmap}[2]{\overline{#1}(#2)}

% Path Endpoints
\newcommand{\pathendpoint}[3]{\constfont{E}_{#1}(#2,#3)}
\newcommand{\idENA}{\constructor{id}} % identity endpoints
\newcommand{\idE}[1]{\idENA_{#1}} % identity endpoints
\newcommand{\comp}[2]{#1 \cdot #2} % composition of endpoints
\newcommand{\inle}{\constructor{inl}} %left inclusion of endpoints
\newcommand{\inre}{\constructor{inr}} % right inclusion of endpoints
\newcommand{\prle}{\constructor{pr}_1} % first projection of endpoints
\newcommand{\prre}{\constructor{pr}_2} % second projection of endpoints
\newcommand{\pair}[2]{(#1 , #2)} % pairing of endpoints
\newcommand{\Ce}{\constructor{c}} % constant endpoint
\newcommand{\constr}{\constructor{constr}} % constructor endpoint
\newcommand{\fmap}{\constructor{fmap}} % function endpoint

\newcommand{\pathendpointFun}[1]{\semE{#1}}
\newcommand{\pathendpointAct}[2]{\pathendpointFun{#1}(#2)}
\newcommand{\pathendpointDact}[2]{\pathendpointFun{\overline{#1}}(#2)}
\newcommand{\pathendpointDnat}[2]{\pathendpointFun{\overline{#1}}(#2)}

% Homotopy Endpoints
\newcommand{\homotendpoint}[6]{\constfont{H}_{#1,#2,#3,#4}(#5, #6)}

\newcommand{\hrefl}[1]{\constructor{idpath}(#1)}
\newcommand{\hinv}[1]{#1^{-1}}
\newcommand{\hconcatsymb}[0]{@}
\newcommand{\hconcat}[2]{#1 \, \hconcatsymb \, #2}
\newcommand{\hassocN}{\boldsymbol\alpha}
\newcommand{\hassoc}[3]{\hassocN(#1, #2, #3)}
\newcommand{\hlunitN}{\boldsymbol\lambda}
\newcommand{\hlunit}[1]{\hlunitN(#1)}
\newcommand{\hrunitN}{\boldsymbol\rho}
\newcommand{\hrunit}[1]{\hrunitN(#1)}
\newcommand{\hprl}[1]{\constructor{pr}_1(#1)}
\newcommand{\hprr}[1]{\constructor{pr}_2(#1)}
\newcommand{\hpairprle}{\constructor{pairpr}_1}
\newcommand{\hpairprre}{\constructor{pairpr}_2}
\newcommand{\hpair}[2]{(#1 , #2)}
\newcommand{\hinl}[1]{\constructor{inl}(#1)}
\newcommand{\hinr}[1]{\constructor{inr}(#1)}
\newcommand{\harg}[0]{\constructor{p}_{\constfont{arg}}}
\newcommand{\hconstr}[2]{\constructor{path}_{#1}(#2)}
\newcommand{\hcomppair}{\constructor{comppair}}
\newcommand{\hap}[2]{\constructor{ap} \ #1 \ #2}
\newcommand{\hcompconst}[1]{\constructor{cmap}(#1)}

\newcommand{\homotendpointAct}[2]{#1(#2)}
\newcommand{\homotendpointDact}[2]{\overline{#1}(#2)}

\newcommand{\idtoH}{\constfont{idtoH}}


% Interpretation
\newcommand{\semP}[1]{\llbracket #1 \rrbracket} % interpretation of poylnomials on 1-types
\newcommand{\semPG}[1]{\langle #1 \rangle} % interpretation of poylnomials on groupoids
\newcommand{\semE}[1]{\llbracket #1 \rrbracket} % interpretation of path endpoints on 1-types
\newcommand{\semEG}[1]{\langle #1 \rangle} % interpretation of path endpoints on groupoids
\newcommand{\semH}[1]{\llbracket #1 \rrbracket} % interpretation of homotopy endpoints on 1-types
\newcommand{\semHG}[1]{\langle #1 \rangle} % interpretation of homotopy endpoints on groupoids

% initial groupoid
\newcommand{\initob}{G_0}
\newcommand{\initmor}[2]{{#1 \sim #2}}
\newcommand{\initeq}[2]{{#1 \approx #2}}
\newcommand{\initeqprop}[2]{{#1 \approx_p #2}}

% Circle
\newcommand{\circleS}{S^1}
\newcommand{\baseS}{\constructor{base}_{S^1}}
\newcommand{\SLoop}{\constructor{loop}_{S^1}}

% Torus
\newcommand{\torus}{\mathcal{T}^2}
\newcommand{\base}{\constructor{base}}
\newcommand{\leftLoop}{\constructor{loop_l}}
\newcommand{\rightLoop}{\constructor{loop_r}}
\newcommand{\surface}{\constructor{surf}}

% Integers modulo 2
\newcommand{\ZT}{\mathbb{Z}_2}
\newcommand{\ZZ}{\constructor{Z}}
\newcommand{\ZS}{\constructor{S}}
\newcommand{\ZM}{\constructor{m}}
\newcommand{\ZC}{\constructor{c}}

% Set Truncation
\newcommand{\ST}[1]{|| #1 ||}
\newcommand{\SC}{\constructor{inc}}
\newcommand{\Strunc}{\constructor{trunc}}

% Free algebra
\newcommand{\freesign}[2]{\constfont{FreeSig}_{#1}(#2)}
\newcommand{\freealg}[2]{\constfont{FreeAlg}_{#1}(#2)}
\newcommand{\freealginc}[1]{\constfont{inc}_{#1}}
\newcommand{\freealgpsfun}[1]{\constfont{F}_{#1}}
\newcommand{\underlying}{\constfont{U}}
\newcommand{\freepath}[1]{\widehat{#1}}
\newcommand{\freehomot}[1]{\widehat{#1}}


%Coequalizer
\newcommand{\coequalizer}[2]{\mathsf{Coeqz}(#1,#2)}
\newcommand{\coequalizerbase}{\constructor{inc}}
\newcommand{\coequalizerglue}{\constructor{glue}}

%Coequifier
\newcommand{\coequifier}[2]{\mathsf{Coeqf}(#1,#2)}
\newcommand{\coequifierbase}{\constructor{inc}}
\newcommand{\coequifierglue}{\constructor{glue}}

%Group quotient
\newcommand{\groupquot}[1]{\mathsf{GroupQuot(#1)}}
\newcommand{\groupquotbase}{\constructor{base}}
\newcommand{\groupquotloop}{\constructor{loop}}
\newcommand{\groupquotloope}{\constructor{loope}}
\newcommand{\groupquotloopm}{\constructor{loopm}}

%Monoidal Object
\newcommand{\monobj}{\mathsf{MonObj}}
\newcommand{\monobjunit}{\constructor{u}}
\newcommand{\monobjtensor}{\constructor{m}}
\newcommand{\monobjlambda}{\constructor{lam}}
\newcommand{\monobjrho}{\constructor{rho}}
\newcommand{\monobjalpha}{\constructor{al}}
\newcommand{\monobjtr}{\constructor{tr}}
\newcommand{\monobjpent}{\constructor{pent}}

%Coherent Group
\newcommand{\cohgroup}{\mathsf{CohGrp}}
\newcommand{\cohgroupunit}{\constructor{u}}
\newcommand{\cohgrouptensor}{\constructor{m}}
\newcommand{\cohgroupinv}{\constructor{i}}
\newcommand{\cohgrouplambda}{\constructor{lam}}
\newcommand{\cohgrouprho}{\constructor{rho}}
\newcommand{\cohgroupalpha}{\constructor{al}}
\newcommand{\cohgrouplinv}{\constructor{linv}}
\newcommand{\cohgrouprinv}{\constructor{rinv}}
\newcommand{\cohgrouptr}{\constructor{tr}}
\newcommand{\cohgrouppent}{\constructor{pent}}
\newcommand{\cohgroupinvadjtrl}{\constructor{invadjtrl}}
\newcommand{\cohgroupinvadjtrr}{\constructor{invadjtrr}}


% Fundamental groups
\newcommand{\circlegrpd}{\mathcal{S}}
\newcommand{\torusgrpd}{\mathcal{T}}
\newcommand{\grquotgrpd}{\mathcal{G}}

\newcommand{\circlegrpdbase}{\mathfrak{b}}
\newcommand{\circlegrpdloop}{\mathfrak{l}}

\newcommand{\torusgrpdbase}{\mathfrak{b}}
\newcommand{\torusgrpdloopl}{\mathfrak{l}}
\newcommand{\torusgrpdloopr}{\mathfrak{r}}
\newcommand{\torusgrpdsurf}{\mathfrak{s}}


\newcommand{\remove}[1]{}


\begin{document}
	
\lstset{language=Coq}

%% Title information
\title{Constructing Higher Inductive Types as Groupoid Quotients}

%% Author information
%% Contents and number of authors suppressed with 'anonymous'.
%% Each author should be introduced by \author, followed by
%% \authornote (optional), \orcid (optional), \affiliation, and
%% \email.
%% An author may have multiple affiliations and/or emails; repeat the
%% appropriate command.
%% Many elements are not rendered, but should be provided for metadata
%% extraction tools.

%% Author with single affiliation.
\author{Niels van der Weide}
\orcid{0000-0003-1146-4161}             %% \orcid is optional
\affiliation{
  \department{Institute for Computation and Information Sciences}
  \institution{Radboud Universiteit}
  \city{Nijmegen}
  \country{The Netherlands}
}
\email{nweide@cs.ru.nl}

\begin{abstract}
	In this paper, we show that all finitary 1-truncated higher inductive types (HITs) can be constructed from the groupoid quotient.
	We start by defining internally a notion of signatures for HITs,
	and for each signature, we construct a bicategory of algebras in 1-types and in groupoids.
	We continue by proving initial algebra semantics for our signatures.
	After that, we show that the groupoid quotient induces a biadjunction between the bicategories of algebras in 1-types and in groupoids.
	We finish by constructing a biinitial object in the bicategory of algebras in groupoids.
	From all this, we conclude that all finitary 1-truncated HITs can be constructed from the groupoid quotient. 
	All the results are formalized over the UniMath library of univalent mathematics in Coq.
\end{abstract}

% 2012 ACM Computing Classification System (CSS) concepts
% Generate at 'http://dl.acm.org/ccs/ccs.cfm'.
\begin{CCSXML}
	<ccs2012>
	<concept>
	<concept_id>10003752.10003790.10011740</concept_id>
	<concept_desc>Theory of computation~Type theory</concept_desc>
	<concept_significance>500</concept_significance>
	</concept>
	<concept>
	<concept_id>10003752.10003790.10003796</concept_id>
	<concept_desc>Theory of computation~Constructive mathematics</concept_desc>
	<concept_significance>500</concept_significance>
	</concept>
	</ccs2012>
\end{CCSXML}

\ccsdesc[500]{Theory of computation~Type theory}
\ccsdesc[500]{Theory of computation~Constructive mathematics}
% end generated code

% Legacy 1998 ACM Computing Classification System categories are also
% supported, but not recommended.
%\category{CR-number}{subcategory}{third-level}[fourth-level]
%\category{D.3.0}{Programming Languages}{General}
%\category{F.3.2}{Logics and Meanings of Programs}{Semantics of Programming Languages}[Program analysis]

\keywords{
higher inductive types,
homotopy type theory,
Coq,
bicategories}

\maketitle

\section{Introduction}
The Martin-Löf identity type, also known as \emph{propositional equality}, represents provable equality in type theory \cite{martin1975intuitionistic}.
This type is defined polymorphically over all types and has a single introduction rule representing reflexivity.
The eliminator, often called the J-rule or path induction, is used to prove symmetry and transitivity.
Note that in particular, we can talk about the identity type of an already established identity type.
This can be iterated to obtain an infinite tower of types, which has the structure of an $\infty$-groupoid \cite{van2011types,lumsdaine2009weak}.

The J-rule is also the starting point of \emph{homotopy type theory} \cite{hottbook}.
In that setting, types are seen as spaces, terms are seen as points, proofs of identity of terms are seen as paths,
and proofs of identity between identities are seen as homotopies.
In mathematical terms, type theory can be interpreted in many Quillen model categories \cite{awodey2009homotopy,LumsdaineW15}, as for example simplicial sets \cite{simpset}.
In the simplicial model, not every two inhabitants of the identity type are equal,
which is also the case in the groupoid model \cite{HofmannS94,MR1686862} and the cubical sets model \cite{BezemCH13}.

If we assume enough axioms, then we can construct types for which we can prove that not every two inhabitants of the identity type are equal.
One example is the universe if one assumes the univalence axiom \cite{hottbook}.
Other examples can be obtained by using \emph{higher inductive types} (HITs).

Higher inductive types generalize inductive types by allowing constructors for paths, paths between paths, and so on.
While inductive types are specified by giving the arities of the operations \cite{dybjer1994inductive},
for higher inductive types one must also specify the arities of the paths, paths between paths, and so on.
The resulting higher inductive type is freely generated by the provided constructors.
To make this concrete, let us look at some examples \cite{hottbook}:

\begin{center}
\begin{lstlisting}[mathescape=true]
Inductive $\circleS$ :=
| $\baseS$ : $\circleS$
| $\SLoop$ : $\baseS = \baseS$
\end{lstlisting}

\begin{lstlisting}[mathescape=true]
Inductive $\torus$ :=
| $\base$ : $\torus$
| $\leftLoop, \rightLoop$ : $\base = \base$
| $\surface$ : $\concat{\leftLoop}{\rightLoop} = \concat{\rightLoop}{\leftLoop}$
\end{lstlisting}
\end{center}

The first one, $\circleS$, represents the circle.
It is generated by a point constructor $\baseS : \circleS$ and a path constructor $\SLoop : \baseS = \baseS$.
The second one, $\torus$, represents the torus.
This type is generated by a point constructor $\base$, two path constructors $\leftLoop$ and $\rightLoop$ of type $\base = \base$,
and a homotopy constructor $\surface : \concat{\leftLoop}{\rightLoop} = \concat{\rightLoop}{\leftLoop}$
where $p \vcomp q$ denotes the concatenation of $p$ and $q$.
Note that path and homotopy constructors depend on previously given constructors in the specification.
For both types, introduction, elimination, and computation rules can be given \cite{hottbook}.

In this paper, we study a schema of higher inductive types that allows defining types by
giving constructors for the points, paths, and homotopies.
All of these constructors can be recursive, but they can only have a finite number of recursive arguments.
Concretely, this means that every inhabitant can be constructed as a finitely branching tree.
Note that recursion is necessary to cover examples such as the set truncation, algebraic theories, and the integers.
Such HITs are called \emph{finitary}.
A similar scheme was studied by Dybjer and Moeneclaey
and they interpret HITs on this scheme in the groupoid model \cite{DBLP:journals/entcs/DybjerM18}.

Say that a type $X$ is \emph{1-truncated} if for all $x, y : X$, $p, q : x = y$, and $r, s : p = q$ we have $r = s$,
and a \emph{1-type} is a type which is 1-truncated.
In terms of the $\infty$-groupoid structure mentioned before, such types are 1-groupoids.
An example of a 1-type is $\circleS$ \cite{LicataS13}, which we mentioned before,
and another one is the classifying space of a group \cite{LicataF14}.
Groupoids are related to 1-types via the \emph{groupoid quotient} \cite{sojakovaPhD},
which takes a groupoid $G$ and returns a 1-type
whose points are objects of $G$ identified up to isomorphism.

The main goal of this paper is to show that finitary 1-truncated higher inductive types can be derived from simpler principles.
More specifically, every finitary 1-truncated HIT can be constructed in a type theory
with propositional truncations, set quotients, and groupoid quotients.
Note that the set quotient is a special instance of the groupoid quotient.
The result of this paper can be used to simplify the semantic study of finitary 1-truncated HITs.
Instead of verifying the existence of a wide class of HITs, one only needs to check the existence
of propositional truncations and groupoid quotients.


Moreover, we employ our framework for HITs for the development of
2-dimensional universal algebra. Each HIT discussed in the paper comes
with a notion of algebra: a 1-type (or a groupoid) which is closed
under the introduction rules of the HIT.  Algebras for a HIT form a
bicategory. We prove that this bicategory has PIE limits. Moreover,
all its morphisms admit a factorization analogous to the one given by the
first isomorphism theorem. Our framework also allows the construction
of the free algebra for a signature, generalizing the notion of term
algebra in (1-dimensional) universal algebra.

Lastly, we show how to exploit our construction of HITs via the
groupoid quotient to calculate the fundamental group of some HITs.

The contributions of this paper are summarized as follows
\begin{itemize}
	\item An internal definition of signatures for HITs which allow path and homotopy constructors (Definition \ref{def:signature});
	\item Bicategories of algebras in both 1-types and groupoids (Definition \ref{def:bicat_grpd});
	\item A proof that biinitial algebras in 1-types satisfy the induction principle (Proposition \ref{thm:initial_alg_sem});
	\item A biadjunction between the bicategories of algebras in 1-types and algebras in groupoids (Construction \ref{constr:alg_biadj});
	\item A construction of 1-truncated HITs from the groupoid quotient (Construction \ref{constr:hit_exist}),
	which shows that such HITs exist.
	This is the main contribution of this paper;
        \item PIE limits in the bicategories of algebras in 1-types (Section \ref{sec:finite_limits});
        \item The definition of the free algebra for a signature as a HIT (Definition \ref{def:free_alg});
        %The free algebra construction underlies a coherent biadjunction between the bicategory of algebras and the bicategory of 1-types;
        \item A proof of the first isomorphism theorem for 1-types (Theorem \ref{thm:iso_thm});
        \item An alternative approach for calculating the fundamental groups of some HITs (Section~\ref{sec:fundamental_groups}).
\end{itemize}
\emph{\textbf{Related Work.}}
Various schemes of higher inductive types have been defined and studied.
Awodey \etal \ studied inductive types in homotopy type theory and prove initial
algebra semantics \cite{AwodeyGS12}.
Sojakova extended their result to various higher inductive types, among which
are the groupoid quotient, W-suspensions, and the torus \cite{Sojakova15,sojakovaPhD}.
For these types, Sojakova proves that homotopy initiality is equivalent to the induction principle while we only show that the induction principle follows from homotopy initiality.
Note that we consider a general types that have higher path constructors but are truncated.
On the other hand, Sojakova looks at arbitrary types which are not necessarily 1-truncated, but she does not consider a general class of HITs with higher path constructor.
In particular, homotopy initiality for the groupoid quotient was proven \cite[Theorem 196]{sojakovaPhD} while we prove
Basold \etal \ \cite{BasoldGW17} defined a scheme for HITs allowing for both point and path constructors,
but no higher constructors, and a similar scheme is given by
Moeneclaey \cite{moeneclaey2016schema}. 
Dybjer and Moeneclaey extended this scheme by allowing homotopy constructors and
they give semantics in the groupoid model \cite{DBLP:journals/entcs/DybjerM18}.
In the framework of computational higher-dimensional type theory \cite{AngiuliHW17},
Cavallo and Harper defined indexed cubical inductive types and prove canonicity \cite{CavalloH19}.
Altenkirch \etal \ defined quotient inductive-inductive types, which combine the features
of quotient types with inductive-inductive types \cite{forsberg2010inductive,AltenkirchCDKF18}.
Kov\'acs and Kaposi extended this syntax to higher inductive-inductive types \cite{KaposiK18},
which can be used to define not necessarily set-truncated types.
The scheme studied in this paper, is most similar to the one by Dybjer and Moeneclaey \cite{DBLP:journals/entcs/DybjerM18}
with the restriction that each type has a constructor indicating that the type is 1-truncated.
In particular, this means that inductive-inductive types are not considered.
Note that the HITs we study only have the right elimination property with respect to 1-types,
unlike W-suspensions \cite{Sojakova15,sojakovaPhD}.

Higher inductive types have already been used for numerous applications.
One of them is synthetic homotopy theory.
Spaces, such as the real projective spaces, higher spheres, and Eilenberg-MacLane spaces,
can be defined as higher inductive types \cite{licata2013pi,LicataF14,DBLP:conf/lics/BuchholtzR17,hottbook}.
The resulting definitions are strong enough to determine homotopy groups
\cite{licata2013pi,LicataS13}.
In addition, algebraic theories can be modeled as HITs, which allows one
to define finite sets as a higher inductive type \cite{frumin2018finite}.
Other applications of HITs include homotopical patch theory, which provides a way
to model version control systems \cite{AngiuliMLH16}, and modeling data types
such as the integers \cite{BasoldGW17,altenkirchscoccola}.
Besides, quotient inductive-inductive types can be used to define the partiality monad \cite{AltenkirchDK17}.
These types can also be used to define type theory within type theory \cite{AltenkirchK16}
and to prove its normalization \cite{DBLP:journals/lmcs/AltenkirchK17}.
Since the HITs in this paper are 1-truncated, they are able to express term algebras of finitary algebraic theories.
For examples such as real projective spaces and higher spheres, we can only define their 1-truncation.

Several classes of higher inductive types have already been reduced to simpler ones.
Both Van Doorn and Kraus constructed propositional truncations from non-recursive higher inductive types
\cite{Doorn16,Kraus16}.
Van Doorn, Von Raumer, and Buchholtz showed that the groupoid quotient can be constructed using pushouts \cite{DoornRB17}.
By combining their result with ours, we get that all finitary 1-truncated higher inductive types can be constructed from pushouts.
Using the join construction, Rijke constructed several examples of HITs, namely $n$-truncations, the Rezk completion,
and set quotients \cite{rijke2017join}.
Awodey \etal \ gave an impredicative construction of finitary inductive types and some HITs \cite{awodey2018impredicative}. 
Constructions of more general classes of HITs have also been given.
Assuming UIP, Kaposi \etal \ constructed all finitary quotient inductive-inductive types
from a single one \cite{KaposiKA19}, and without UIP, Van der Weide and Geuvers
constructed all finitary set truncated HITs from quotients \cite{van2019construction}.
Note that these two works only concern set truncated HITs while our work concerns 1-truncated HITs.
Furthermore, the HITs considered by Van der Weide and Geuvers are a special case of
the HITs in this paper.

An alternative way to verify the existence of higher inductive types,
is by constructing them directly in a model.
Coquand \etal \ interpreted several HITs in the cubical sets model \cite{BezemCH13,CoquandHM18}.
Note that one can constructively prove univalence in the cubical sets model \cite{CohenCHM16}
and that cubical type theory satisfies homotopy canonicity \cite{DBLP:conf/rta/CoquandHS19}.
Furthermore, cubical type theory has been implemented in Agda with support for higher inductive types \cite{vezzosi2019cubical}.
Lumsdaine and Shulman give a semantical scheme for HITs and show that these can be interpreted
in sufficiently nice model categories \cite{lumsdaine2017semantics}.

Lastly, there are other approaches to 2-dimensional universal algebra.
Blackwell \etal \ study 2-dimensional universal algebra in a syntax-less fashion: they define 2-categories of algebras of a given 2-monad \cite{blackwell1989two}.
Their main result says that these 2-categories support limits and colimits.
On the contrast, our approach is based on a concrete notion of signature, which represents the syntax.
Note that by Corollary \ref{cor:pseudomonadofsig}, each signature gives rise to a pseudomonad and as such, the work by Blackwell \etal \ considers a more general version of 2-dimensional universal algebra.
\\
\emph{\textbf{Formalization.}}
All results in this paper are formalized in Coq \cite{Coq:manual} using UniMath \cite{UniMath}.
The formalization uses the version with \texttt{git} hash \href{https://github.com/UniMath/UniMath/tree/2dadfb61f5ef0d9805cf0eb6b80ef2beb26472d5}{2dadfb61} and can be found here:
\begin{center}
\url{https://github.com/nmvdw/GrpdHITs/tree/extended}
\end{center}
\emph{\textbf{Overview.}}
We start by recalling the groupoid quotient and displayed bicategories in Section \ref{sec:prelims}.
Displayed bicategories are our main tool to construct the bicategory of algebras for a signature.
In Section \ref{sec:signs}, we define signatures and show that each signature gives rise to a bicategory of algebras in both 1-types and groupoids.
The notion of a higher inductive type on a signature is given in Section \ref{sec:induction}.
There, we also prove initial algebra semantics, which says that biinitiality is a sufficient condition for being a HIT.
To construct the desired higher inductive type, we use the groupoid quotient, and in Section \ref{sec:biadj} we lift this to a biadjunction on the level of algebras.
As a consequence, constructing the initial algebra of a signature in groupoids is sufficient to construct the desired higher inductive type.
In Section \ref{sec:existence}, we construct the desired initial algebra and we conclude that each signature has an associated higher inductive type. 
Next we discuss more examples of HITs in Section \ref{sec:examples} and there we also show how to obtain monoidal objects and coherent 2-groups as algebras for certain signatures.
After that we study 2-dimensional universal algebra with our signatures.
More specifically, we construct PIE limits of algebras in Section \ref{sec:finite_limits}, the free algebra for a signature in Section \ref{sec:free_algebra}, and we prove the first isomorphism theorem in Section \ref{sec:isomorphism_theorem}.
The final topic we discuss, is the calculation of fundamental groups.
In Section \ref{sec:fundamental_groups}, we use the way we constructed higher inductive types to determine the fundamental group of the circle, the torus, and the group quotient.
Lastly, we conclude in Section \ref{sec:conclusion}.
\\
\emph{\textbf{Publication History.}}
This paper is an extended version of \cite{nmvdw2020} by the second author.
In Section \ref{sec:signs}, we changed Definition \ref{def:homotep} and in Section \ref{sec:induction}, we added Construction \ref{constr:total_alg}.
Sections \ref{sec:examples} to \ref{sec:fundamental_groups} are new.
\\
\emph{\textbf{Notation.}}
In this paper, we work in dependent type theory and we assume the univalence axiom. In particular, this means that we also have function extensionality.
Let us recall some notation from HoTT which we use throughout this paper.
The identity path is denoted by $\refl{x}$, the concatenation of paths $p : x = y$ and $q : y = z$ is denoted by $p \vcomp q$, and the inverse of a path $p : x = y$ is denoted by $\inverse{p} : y = x$.
Given a type $X$ with points $x, y : X$ and paths $p, q : x = y$, we call a path $s : p = q$ a \emph{2-path}.
A \emph{proposition} is a type of which all inhabitants are equal.
A \emph{set} is a type $X$ such that for all $x, y : X$ the type $x = y$ is a proposition.
\remove{A \emph{1-type} is a type $X$ such that for all $x, y : X$ the type $x = y$ is a set.}
A \emph{homotopy} between $f, g : X \rightarrow Y$ consists of a path
$f(x) = g(x)$ for each $x : A$. By assuming function
extensionality, we have access to a map $\funextsec$ sending a
homotopy between functions $f$ and $g$ to a path $f = g$. Given a
type $A$, we write $\PT{A}$ for its \emph{propositional truncation}
and $\PC : A \to \PT{A}$ for the point constructor. We define the existential quantification
$\exists a : A.B(a)$ as $\PT{\Sum {a : A}. B(a)}$.


\section{Preliminaries}
\label{sec:prelims}
\begin{figure*}
Introduction rules: 

\vspace{2pt}

\begin{center}
\begin{bprooftree}
\AxiomC{$x : G$}
\UnaryInfC{$\gcl(x) : \gquotType(G)$}
\end{bprooftree}
\begin{bprooftree}
\AxiomC{$x, y : G$}
\AxiomC{$f : \mor{G}{x}{y}$}
\BinaryInfC{$\gcleq(f) : \gcl(x) = \gcl(y)$}
\end{bprooftree}
\begin{bprooftree}
\AxiomC{$x : G$}
\UnaryInfC{$\ge(x) : \gcleq(\idgrpd{x}) = \refl{\gcl(x)}$}
\end{bprooftree}
\end{center}

\vspace{5pt}


\begin{center}
\begin{bprooftree}
\AxiomC{$x, y, z : G$}
\AxiomC{$f : \mor{G}{x}{y}$}
\AxiomC{$g : \mor{G}{y}{z}$}
\TrinaryInfC{$\gconcat(f, g) : \gcleq(\compgrpd{f}{g}) = \concat{\gcleq(f)}{\gcleq(g)}$}
\end{bprooftree}
\begin{bprooftree}
\AxiomC{$x, y : \gquotType(G)$}
\AxiomC{$p, q : x = y$}
\AxiomC{$r, s : p = q$}
\TrinaryInfC{$\gtrunc(r, s) : r = s$}
\end{bprooftree}
\end{center}

\vspace{5pt}

Elimination rule:

\vspace{2pt}

\begin{center}
\begin{bprooftree}
\AxiomC{$Y : \gquotType(G) \rightarrow \onetypes$}
\AxiomC{$\gclY : \depprod{(x : G)}{Y(\gcl(x))}$}
\noLine
\BinaryInfC{$\gcleqY : \depprod{(x, y : G) (f : \mor{G}{x}{y})}{\pathover{\gcleq(f)}{\gclY(x)}{\gclY(y)}}$}
\noLine
\UnaryInfC{$\geY : \depprod{(x : G)}{\globeover{\ge(x)}{\gcleqY(\idgrpd{x})}{\refl{\gclY(x)}}}$}
\noLine
\UnaryInfC{$\gconcatY : \depprod{(x, y, z : G) (f : \mor{G}{x}{y}) (g : \mor{G}{y}{z})}{}$}
\noLine
\UnaryInfC{$\qquad\qquad\qquad\qquad\globeover{\gconcat(f, g)}{\gcleqY(\compgrpd{f}{g})}{\concat{\gcleqY(f)}{\gcleqY(g)}}$}
\UnaryInfC{$\gind(\gclY, \gcleqY, \geY, \gconcatY) : \depprod{(x : \gquotType(G))}{Y(x)}$}
\end{bprooftree}
\end{center}

\vspace{5pt}

Computation rules:

For $\gcl$: $\gind(\gclY, \gcleqY, \geY, \gconcatY)(\gcl(x)) \defeq \gclY(x)$

For $\gcleq$: $\apd{(\gind(\gclY, \gcleqY, \geY, \gconcatY))}{(\gcleq(f))} = \gcleqY(f)$
\caption{Introduction, elimination, and computation rules for the groupoid quotient \cite{sojakovaPhD}.}
\label{fig:gquot}
\end{figure*}
		
\subsection{Groupoid Quotient}
Let us start by formally introducing the groupoid quotient \cite{sojakovaPhD}.
The groupoid quotient is a higher dimensional version of the set quotient,
so let us quickly recall the set quotient.
Given a setoid $(X,R)$ (a set $X$ with an equivalence relation $R$ valued in propositions on $X$),
the set quotient gives a type $\quot{X}{R}$, which is $X$ with the points identified according to $R$.
Note that $\quot{X}{R}$ always is a set since $R$ becomes the equality.

Instead of a setoid, the groupoid quotient takes a groupoid as input.
Recall that a groupoid is a category in which every morphism is invertible.
In particular, each groupoid has identity morphisms, denoted by $\idgrpd{x}$, and a composition operation.
The composition of $f$ and $g$ is denoted by $\compgrpd{f}{g}$.
In addition, the type of morphisms from $x$ to $y$ is required to be a set.
We write $\grpd$ for the type of groupoids.

Given $G : \grpd$, the groupoid quotient gives a 1-type $\gquotType(G)$.
In this type, the points are objects of $G$
and these are identified according to the morphisms in $G$.
In addition, the groupoid structure must be preserved.
Informally, we define the groupoid quotient as the following HIT.

\begin{lstlisting}[mathescape=true]
Inductive $\gquotType$ $(G : \grpd)$ :=
| $\gcl$ : $\functions{G}{\gquotType(G)}$
| $\gcleq$ : $\depprod{(x, y : G) (f : \mor{G}{x}{y})}{\gcl(x) = \gcl(y)}$
| $\ge$ : $\depprod{(x : G)}{\gcleq(\idgrpd{x}) = \refl{\gcl(x)}}$
| $\gconcat$ : $\depprod{(x, y, z : G)(f : \mor{G}{x}{y})(g : \mor{G}{y}{z})}{}$
           $\gcleq(\compgrpd{f}{g}) = \concat{\gcleq(f)}{\gcleq(g)}$
| $\gtrunc$ : $\depprod{(x, y : \gquotType(G))(p, q: x = y) (r, s : p = q)}$
           ${r = s}$
\end{lstlisting}

To formally add this type to our theory, we need to provide
introduction, elimination, and computation rules for $\gquotType(G)$.
Formulating the elimination principle requires two preliminary notions.
These are inspired by the work by Licata and Brunerie \cite{licata2015cubical}.
The first of these gives paths in a dependent type over a path in the base.

\begin{definition}
\label{def:path_over}
Given a type $X : \Type$,
a type family $Y : \functions{X}{\Type}$,
points $x_1, x_2 : X$,
a path $p : x_1 = x_2$,
and points $\overline{x_1} : Y(x_1)$ and $\overline{x_2} : Y(x_2)$ over $x_1$ and $x_2$ respectively,
we define the type $\pathover[Y]{p}{\overline{x_1}}{\overline{x_2}}$ of \fat{paths over} $p$ from $\overline{x_1}$ to $\overline{x_2}$ by path induction on $p$
by saying that the paths over the identity path $\refl{x}$ from $\overline{x_1}$ to $\overline{x_2}$ are just paths $\overline{x_1} = \overline{x_2}$.
\end{definition}

Note that the groupoid quotient also has constructors for paths between paths.
This means that we also need a dependent version of 2-paths,
and inspired by the terminology of globular sets, we call these \emph{globes} over a given 2-path.
We define them as follows.

\begin{definition}
\label{def:globe_over}
Let $X$, $Y$, and $x_1, x_2$ be as in Definition \ref{def:path_over}.
Suppose, that we paths $p, q : x_1 = x_2$,
a 2-path $g : p = q$,
and paths $\pover{p} : \pathover{p}{\overline{x_1}}{\overline{x_2}}$ and $\pover{q} : \pathover{q}{\overline{x_1}}{\overline{x_2}}$ over $p$ and $q$ respectively,
we define the type $\globeover{g}{\pover{p}}{\pover{q}}$ of \fat{globes over} $g$ from $\pover{p}$ to $\pover{q}$ by path induction on $g$
by saying that the paths over the identity path $\refl{p}$ are just paths $\pover{p} = \pover{q}$.
\end{definition}

From this point on, we assume that our type theory has the groupoid quotient.
More specifically, we assume the following axiom.

\begin{axiom}
For each groupoid $G$ there is a type $\gquotType(G)$ which satisfies the rules in Figure \ref{fig:gquot}.
\end{axiom}

Note that there are no computation rules for $\gconcat$, $\ge$, and $\gtrunc$,
because such rules follow from the fact that $Y$ is a family of 1-types.

\subsection{Bicategory Theory}
The upcoming construction makes heavy use of notions from bicategory theory \cite{10.1007/BFb0074299,leinster:basic-bicats}
and in particular, the displayed machinery introduced by Ahrens \etal \ \cite{bicatjournal}.
Here we recall some examples of bicategories and the basics of displayed bicategories.

Recall that a bicategory consists of objects, 1-cells between objects, and 2-cells between 1-cells.
The type of 1-cells from $x$ to $y$ is denoted by $x \onecell y$ and the type of 2-cells from $f$ to $g$ is denoted by $f \twocell g$.
Note that the type $f \twocell g$ is required to be a set.
There are identity 1-cells and 2-cells denoted by $\id_1$ and $\id_2$ respectively, composition of 1-cells $f$ and $g$ is denoted by $f \cdot g$,
and the vertical composition of 2-cells $\tc$ and $\tc'$ is denoted by $\tc \vcomp \tc'$.
The left whiskering of a 2-cell $\tc$ with 1-cell $f$ is denoted by $f \whiskerl \tc$ and right whiskering of $\tc$ with a 1-cell $g$ is denoted by $\tc \whiskerr g$. 
Unitality and associativity of vertical composition of 2-cells hold strictly while for 1-cells, these laws only hold up to an invertible 2-cell.

Let us fix some notation before continuing.
Given bicategories $\B_1$ and $\B_2$, we write $\pseudo(\B_1, \B_2)$ for the type of pseudofunctors from $\B_1$ to $\B_2$.
The type of pseudotransformations from $F$ to $G$ is denoted by $\pstrans{F}{G}$
and the type of modifications from $\theta$ to $\theta'$ is denoted by $\modif{\theta}{\theta'}$ \cite{leinster:basic-bicats}.
Next we discuss \emph {biadjunctions} \cite{gurski2012biequivalences,LACK2000179}.

\begin{definition}
Let $\B_1$ and $\B_2$ be bicategories.
A \fat{biadjunction} from $\B_1$ to $\B_2$ consists of
\begin{itemize}
	\item pseudofunctors $L : \pseudo(\B_1, \B_2)$ and $R : \pseudo(\B_2, \B_1)$;
	\item pseudotransformations $\eta : \pstrans{\id(\B_1)}{R \circ L}$ and $\varepsilon : \pstrans{L \circ R}{\id(\B_2)}$;
	\item invertible modifications
	\[
	\tau_1 : \modif{\rho(L)^{-1} \vcomp L \whiskerl \eta \vcomp \alpha(L, R, L) \vcomp \varepsilon \whiskerr L \vcomp \lambda(L)}{\id(L)}
	\]
	\[
	\tau_2 : \modif{\lambda(R)^{-1} \vcomp \eta \whiskerr R \vcomp \alpha(R, L, R)^{-1} \vcomp R \whiskerl \varepsilon \vcomp \rho(R)}{\id(R)}.
	\]
\end{itemize}
The type of biadjunctions from $\B_1$ to $\B_2$ is denoted by $L \dashv R$
where $L : \pseudo(\B_1, \B_2)$ and $R : \pseudo(\B_2, \B_1)$.
If we have $L \dashv R$, we say that $L$ is \fat{left biadjoint} to $R$.
\end{definition}

Before we say when a biadjunction is coherent, let us introduce some notation.
Suppose, that we have a biadjunction $L \dashv R$ where $L : \pseudo(\B_1, \B_2)$.
For each $x : \B_1$ we get a 1-cell $\eta(x) : x \onecell R(L(x))$ and for $x : \B_2$, we get a 1-cell $\varepsilon(x) : L(R(x)) \onecell x$.
Given $x : \B_1$, we get an invertible 2-cell $\widehat{\tau_2(x)} : L(\eta(x)) \cdot \varepsilon(L(x)) \twocell \id_1(L(x))$.
Similarly, we define an invertible 2-cell $\widehat{\tau_1(x)} : \eta(R(x)) \cdot R(\varepsilon(x)) \twocell \id_1(R(x))$ for $x : \B_2$.

\begin{definition}
Given bicategories $\B_1$ and $\B_2$ and a biadjunction $L \dashv R$ with $L : \pseudo(\B_1, \B_2)$, we say that $L \dashv R$ is \fat{coherent} if the following 2-cells are equal to identity 2-cells
\begin{equation*}
\begin{split}
(\identitor{L} \vcomp L(\widehat{\tau_1(x)}^{-1}) \vcomp \compositor{L}^{-1}) \whiskerr \varepsilon(x)
\vcomp \alpha^{-1}
\vcomp L(\eta(R(x))) \whiskerl \varepsilon_1(\varepsilon(x))^{-1}
\vcomp \alpha
\vcomp \widehat{\tau_2(R(x))} \whiskerr \varepsilon(x)
\end{split}
\end{equation*}
\begin{equation*}
\begin{split}
\eta(x) \whiskerl \widehat{\tau_1(L(x))}^{-1}
\vcomp \alpha
\vcomp (\eta_1(\eta(x)))^{-1} \whiskerr R(\varepsilon(L(x)))
\vcomp \alpha^{-1}
\vcomp \eta(x) \whiskerl (\compositor{R} \vcomp R(\widehat{\tau_2(x)}) \vcomp \identitor{R}^{-1})
\end{split}
\end{equation*}
\end{definition}

Beside these standard notions, we use two bicategories: $\onetypes$ and $\grpd$.

\begin{example}
We have
\begin{itemize}
	\item a bicategory $\onetypes$ whose objects are 1-types, 1-cells are functions, and 2-cells are homotopies;
	\item a bicategory $\grpd$ of groupoids
	whose objects are (not necessarily univalent) groupoids, 1-cells are functors, and 2-cells are natural transformations.
\end{itemize}
\end{example}

Next we discuss \emph{displayed bicategories}, which is our main tool to define bicategories of algebras on a signature.
Intuitively, a displayed bicategory $\D$ over $\B$ represents structure and properties to be added to $\B$.
Displayed bicategories generalize displayed categories to the bicategorical setting \cite{AhrensL19}.
Each such $\D$ gives rise to a total bicategory $\total{\D}$.
The full definition can be found in the paper by Ahrens \etal \ \cite{bicatjournal}, and here, we only show a part.

\begin{definition}
Let $\B$ be a bicategory.
A \fat{displayed bicategory} $\D$ over $\B$ consists of
\begin{itemize}
	\item For each $x : \B$ a type $\dob{\D}{x}$ of \fat{objects over $x$};
	\item For each $f : x \onecell y$, $\xx : \dob{\D}{x}$ and $\yy : \dob{\D}{y}$,
	a type $\dmor{\xx}{\yy}{f}$ of \fat{1-cells over $f$};
	\item For each $\tc : f \twocell g$, $\ff : \dmor{\xx}{\yy}{f}$, and $\gg : \dmor{\xx}{\yy}{g}$, a \emph{set} $\dtwo{\ff}{\gg}{\tc}$ of \fat{2-cells over $\tc$}.
\end{itemize}
In addition, there are identity cells and there are composition and whiskering operations.
The composition of displayed 1-cells $f$ and $g$ is denoted by $f \cdot g$, the displayed identity 1-cell is denoted by $\id_1(x)$.
The vertical composition of 2-cells $\tc$ and $\tc'$ is denoted by $\tc \vcomp \tc'$, the left and right whiskering is denoted by $f \whiskerl \tc$ and $\tc \whiskerr f$ respectively,
and the identity 2-cell is denoted by $\id_2(f)$. 
\end{definition}

\begin{definition}
\label{def:totalbicat}
Let $\B$ be a bicategory and let $\D$ be a displayed bicategory over $\B$.
We define the \fat{total bicategory} $\total{D}$ as the bicategory whose objects of $\total{D}$ are just dependent pairs $(x, \xx)$ with $x$ in $\B$ and $\xx$ in $\dob{\D}{x}$.
The 1-cells and 2-cells in $\total{D}$ are defined similarly.
In addition, we define the \fat{projection} $\dproj{D} : \pseudo(\total{D}, \B)$ to be the pseudofunctor which takes the first component of each pair.
\end{definition}

Let us finish this section by defining the displayed bicategories we need in the remainder of this paper.
Motivation and explanation of Examples \ref{ex:DFAlg} and \ref{ex:DCell} is given by Ahrens \etal \ \cite{bicatjournal}.

\begin{example}
\label{ex:DFAlg}
Given a bicategory $\B$ and a pseudofunctor $F : \pseudo(\B, \B)$,
we define a displayed bicategory $\DFAlg(F)$ over $\B$ such that
\begin{itemize}
	\item the objects over $x : \B$ are 1-cells $h_x : F(x) \onecell x$;
	\item the 1-cells over $f : x \onecell y$ from $h_x$ to $h_y$ are invertible 2-cells $\tcC_f : h_x \cdot f \twocell F(f) \cdot h_y$;
	\item the 2-cells over $\tc : f \twocell g$ from $\tcC_f$ to $\tcC_g$ are equalities
	\begin{equation*}\label{eq:twocell_alg}
	h_x \whiskerl \tc \vcomp \tcC_g = \tcC_f \vcomp F(\tc) \whiskerr h_y.
	\end{equation*}
\end{itemize}
\end{example}

\begin{example}
\label{ex:disp_depprod}
Given a bicategory $\B$, a type $I$, and for each $i : I$ a displayed bicategory $\D_i$ over $\B$,
we define a displayed bicategory $\depprod{(i : I)}{\D_i}$ over $\B$ such that
\begin{itemize}
	\item the objects over $x : \B$ are functions $\depprod{(i : I)}{\D_i(x)}$;
	\item the 1-cells over $f : x \onecell y$ from $\overline{x}$ to $\overline{y}$ are functions $\depprod{(i : I)}{\dmor{\overline{x}(i)}{\overline{y}(i)}{f}}$;
	\item the 2-cells over $\tc : f \twocell g$ from $\overline{f}$ to $\overline{g}$ are functions $\depprod{(i : I)}{\dtwo{\overline{f}(i)}{\overline{g}(i)}{\tc}}$.
\end{itemize}
\end{example}

\begin{example}
\label{ex:DCell}
Let $\B$ be a bicategory with a displayed bicategory $\D$ over it.
Now suppose that we have pseudofunctors $S, T : \pseudo(\B, \B)$ and two pseudotransformations $l, r : \pstrans{\dproj{D} \cdot S}{\dproj{D} \cdot T}$.
Then we define a displayed bicategory $\DCell(l,r)$ over $\total{D}$ such that
\begin{itemize}
	\item the objects over $x$ are 2-cells $\tcB_x : l(x) \twocell r(x)$;
	\item the 1-cells over $f : x \onecell y$ from $\tcB_x$ to $\tcB_y$ are equalities
	\[
	(\tcB_x \whiskerr T(\dproj{D}(f))) \vcomp r(f)
	=
	l(f) \vcomp (S(\dproj{D}(f)) \whiskerl \tcB_y);
	\]
	\item the 2-cells over $\tc : f \twocell g$ are inhabitants of the unit type.
\end{itemize}
\end{example}

\begin{example}
\label{ex:fullsub}
Let $\B$ be a bicategory and let $P$ be a family of propositions on the objects of $\B$.
Then we define a displayed bicategory $\FSub(P)$ over $\B$ whose objects over $x$ are proofs of $P(x)$
and whose displayed 1-cells and 2-cells are inhabitants of the unit type.
The total bicategory $\total{\FSub(P)}$ is the \fat{full subbicategory} of $\B$ whose objects satisfy $P$.
\end{example}


\section{Signatures and their Algebras}
\label{sec:signs}
Before we can discuss how to construct 1-truncated higher inductive types,
we need to define signatures for those.
Our notion of signature is similar to the one by Dybjer and Moenclaey \cite{DBLP:journals/entcs/DybjerM18}.
However, instead of defining them externally, we define a type of signatures within type theory
just like what was done for inductive-recursive and inductive-inductive
definitions \cite{Dybjer1999AFA,forsberg2012finite}. 
We also show that each signature $\sign$ gives rise to a bicategory of algebras for $\sign$.

In this section, we study HITs of the following shape
\begin{lstlisting}[mathescape=true]
Inductive $H$ :=
| $c$ : $\functions{\polyAct{P}{H}}{H}$
| $p$ : $\depprod{(j : I) (x : \polyAct{Q}{H})}{l(x) = r(x)}$
| $s$ : $\depprod{(j : J) (x : \polyAct{R}{H}) (r : a_1(x) = a_2(x))}{\homotendpointAct{q_1}{x, r} = \homotendpointAct{q_2}{x, r}}$
| $t$ : $\depprod{(x, y : H) (q_1, q_2 : x = y) (r_1, r_2 : q_1 = q_2)}{r_1 = r_2}$
\end{lstlisting}
To see what the challenges are when defining such HITs, let us take a closer look at the torus.

\begin{lstlisting}[mathescape=true]
Inductive $\torus$ :=
| $\base$ : $\torus$
| $\leftLoop, \rightLoop$ : $\base = \base$
| $\surface$ : $\concat{\leftLoop}{\rightLoop} = \concat{\rightLoop}{\leftLoop}$
\end{lstlisting}

There is a point constructor $\base$, two paths constructors $\leftLoop, \rightLoop : \base = \base$,
and a homotopy constructor $\surface : \concat{\leftLoop}{\rightLoop} = \concat{\rightLoop}{\leftLoop}$.
Note that $\leftLoop$ and $\rightLoop$ refer to $\base$ and that $\surface$ refers to all other constructors.
Hence, the signatures we define must be flexible enough to allow such dependencies.

A similar challenge comes up when defining the bicategory of algebras for a signature.
For the torus, an algebra would consist of a type $X$, a point $b$, paths $p, q : b = b$, and a 2-path $s : \concat{p}{q} = \concat{q}{p}$.
Again there are dependencies: $p$ and $q$ depend on $b$ while $s$ depends on both $p$ and $q$.
To deal with these dependencies, we use displayed bicategories, which allow us to construct the bicategory of algebras in a stratified way.

\subsection{Signatures}\label{sec:signatures}
Now let us define signatures, and to do so, we must specify data which describes the constructors for points, paths, and homotopies.
To specify the point constructors, we use \emph{polynomial codes}.
Given a type $X$ and a polynomial code $P$, we get another type $\polyAct{P}{X}$.
Such a code $P$ describes the input of an operation of the form $\polyAct{P}{X} \rightarrow X$.

\begin{definition}
The type of \fat{codes for polynomials} is inductively generated by the following constructors
\[ 
\constantP{X} : \poly, \quad \idP : \poly, \quad \sumP{P_1}{P_2} : \poly, \quad \prodP{P_1}{P_2} : \poly
\]
where $X$ is a 1-type and $P_1$ and $P_2$ are elements of $\poly$.
\end{definition}

The constructor $\constantP{X}$ represents the constant polynomial returning the type $X$, $\idP$ represents the identity,
and $\sumP{P_1}{P_2}$ and $\prodP{P_1}{P_2}$ represent the sum and product respectively.
Note that we restrict ourselves to finitary polynomials since we do not have a constructor which represents the function space.

The second part of the signature describes the path constructors, which represent universally quantified equations.
To describe them, we must give two \emph{path endpoints}.
These endpoints can refer to the point constructor, which we represent by a polynomial $A$.
In addition, they have a source (the type of the quantified variable) and a target (the type of the term).
The source and the target are represented by polynomials $S$ and $T$ respectively.

\begin{figure*}[t]
\begin{center}
\begin{bprooftree}
\AxiomC{$P : \poly$}
\UnaryInfC{$\idE{A} : \pathendpoint{A}{P}{P}$}
\end{bprooftree}
\begin{bprooftree}
\AxiomC{$P, Q, R : \poly$}
\AxiomC{$e_1 : \pathendpoint{A}{P}{Q}$}
\AxiomC{$e_2 : \pathendpoint{A}{Q}{R}$}
\TrinaryInfC{$\comp{e_1}{e_2} : \pathendpoint{A}{P}{R}$}
\end{bprooftree}
\end{center}

\vspace{5pt}

\begin{center}
\begin{bprooftree}
\AxiomC{$\vphantom{P, Q : \poly}$}
\UnaryInfC{$\constr : \pathendpoint{A}{A}{\idP}$}
\end{bprooftree}
\begin{bprooftree}
\AxiomC{$P, Q : \poly$}
\UnaryInfC{$\inle : \pathendpoint{A}{P}{\sumP{P}{Q}}$}
\end{bprooftree}
\begin{bprooftree}
\AxiomC{$P, Q : \poly$}
\UnaryInfC{$\inre : \pathendpoint{A}{Q}{\sumP{P}{Q}}$}
\end{bprooftree}
\end{center}

\vspace{5pt}

\begin{center}
\begin{bprooftree}
\AxiomC{$P, Q : \poly$}
\UnaryInfC{$\prle : \pathendpoint{A}{\prodP{P}{Q}}{P}$}
\end{bprooftree}
\begin{bprooftree}
\AxiomC{$P, Q : \poly$}
\UnaryInfC{$\prre : \pathendpoint{A}{\prodP{P}{Q}}{Q}$}
\end{bprooftree}
\end{center}

\vspace{5pt}

\begin{center}

\begin{bprooftree}
\AxiomC{$P, Q, R: \poly$}
\AxiomC{$e_1 : \pathendpoint{A}{P}{Q}$}
\AxiomC{$e_2 : \pathendpoint{A}{P}{R}$}
\TrinaryInfC{$\pair{e_1}{e_2} : \pathendpoint{A}{P}{\prodP{Q}{R}}$}
\end{bprooftree}
\end{center}

\vspace{5pt}

\begin{center}
\begin{bprooftree}
\AxiomC{$P : \poly$}
\AxiomC{$X : \onetypes$}
\AxiomC{$x : X$}
\TrinaryInfC{$\Ce(x) : \pathendpoint{A}{P}{\constantP{X}}$}
\end{bprooftree}
\begin{bprooftree}
\AxiomC{$X , Y : \onetypes$}
\AxiomC{$f : X \rightarrow Y$}
\BinaryInfC{$\fmap(f) : \pathendpoint{A}{\constantP{X}}{\constantP{Y}}$}
\end{bprooftree}
\end{center}
\caption{Rules for the path endpoints.}
\label{fig:path_ep}
\end{figure*}

\begin{definition}
Let $A$, $S$, and $T$ be codes for polynomials.
The type $\pathendpoint{A}{S}{T}$ of \fat{path endpoints} with arguments $A$, source  $S$, and target $T$ is inductively generated by the constructors given in Figure \ref{fig:path_ep}.
\end{definition}

Note that the parameter $A$ is only used in the path endpoint $\constr$, which represents the point constructor.
If we have a type $X$ with a function $c : \polyAct{A}{X} \rightarrow X$,
then each endpoint $e$ gives for every $x : \polyAct{S}{X}$ a point $\pathendpointFun{e}(x) : \polyAct{T}{X}$.
Note that $\pathendpointFun{e}(x)$ depends on $c$ while we do not write $c$ in the notation.
Often we write $e(x)$ instead of $\pathendpointFun{e}(x)$.
Hence, two endpoints $l, r : \pathendpoint{A}{S}{T}$ represent the equation
\[
\depprod{(x : \polyAct{S}{X})}{\pathendpointAct{l}{x} = \pathendpointAct{r}{x}}.
\]
Note that a HIT could have arbitrarily many path constructors and we index them by the type $J$.

The last part of the signature describes the homotopy constructors
and these depend on both the point and path constructors.
A homotopy constructor represents an equation of paths, which is universally quantified over both points and paths of
the HIT being defined.
The point argument is represented by a polynomial $R$, and the path argument is represented by a polynomial $T$ and endpoints $a, b : \pathendpoint{A}{R}{T}$.
Lastly, the type of the paths in the equation is described by two endpoints $s, t : \pathendpoint{A}{R}{W}$ with a polynomial $W$.

\begin{figure*}[t]
\begin{center}
\begin{bprooftree}
\AxiomC{$T : \poly$}
\AxiomC{$e : \pathendpoint{A}{R}{T}$}
\BinaryInfC{$\hrefl{e} : \homotendpoint{l}{r}{a}{b}{e}{e}$}
\end{bprooftree}
\begin{bprooftree}
\AxiomC{$T : \poly$}
\AxiomC{$e_1, e_2 : \pathendpoint{A}{R}{T}$}
\AxiomC{$h : \homotendpoint{l}{r}{a}{b}{e_1}{e_2}$}
\TrinaryInfC{$\hinv{h} : \homotendpoint{l}{r}{a}{b}{e_2}{e_1}$}
\end{bprooftree}
\end{center}

\vspace{5pt}

\begin{center}
\begin{bprooftree}
\AxiomC{$T : \poly$}
\AxiomC{$e_1, e_2, e_3 : \pathendpoint{A}{R}{T}$}
\AxiomC{$h_1 : \homotendpoint{l}{r}{a}{b}{e_1}{e_2}$}
\AxiomC{$h_2 : \homotendpoint{l}{r}{a}{b}{e_2}{e_3}$}
\QuaternaryInfC{$\hconcat{h_1}{h_2} : \homotendpoint{l}{r}{a}{b}{e_1}{e_3}$}
\end{bprooftree}
\end{center}

\vspace{5pt}

\begin{center}
\begin{bprooftree}
\AxiomC{$T_1, T_2 : \poly$}
\AxiomC{$e_1, e_2 : \pathendpoint{A}{Q}{T_1}$}
\AxiomC{$e : \pathendpoint{A}{T_1}{T_2}$}
\AxiomC{$h : \homotendpoint{l}{r}{a}{b}{e_1}{e_2}$}
\QuaternaryInfC{$\hap e h : \homotendpoint{l}{r}{a}{b}{\comp{e_1}{e}}{\comp{e_2}{e}}$}
\end{bprooftree}
\end{center}

\vspace{5pt}

\begin{center}
\begin{bprooftree}
\AxiomC{$T_1, T_2, T_3 : \poly$}
\AxiomC{$e_1 : \pathendpoint{A}{R}{T_1}$}
\AxiomC{$e_2 : \pathendpoint{A}{T_1}{T_2}$}
\AxiomC{$e_3 : \pathendpoint{A}{T_2}{T_3}$}
\QuaternaryInfC{$\hassoc{e_1}{e_2}{e_3} : \homotendpoint{l}{r}{a}{b}{\comp{e_1}{(\comp{e_2}{e_3})}}{\comp{(\comp{e_1}{e_2})}{e_3}}$}
\end{bprooftree}
\end{center}

\vspace{5pt}

\begin{center}
\begin{bprooftree}
\AxiomC{$T: \poly$}
\AxiomC{$e : \pathendpoint{A}{R}{T}$}
\BinaryInfC{$\hlunit{e} : \homotendpoint{l}{r}{a}{b}{\comp{\idE{R}}{e}}{e}$}
\end{bprooftree}
\begin{bprooftree}
\AxiomC{$T: \poly$}
\AxiomC{$e : \pathendpoint{A}{R}{T}$}
\BinaryInfC{$\hrunit{e} : \homotendpoint{l}{r}{a}{b}{\comp{e}{\idE{T}}}{e}$}
\end{bprooftree}
\end{center}

\vspace{5pt}

\begin{center}
\begin{bprooftree}
\AxiomC{$T_1, T_2 : \poly$}
\AxiomC{$e_1 : \pathendpoint{A}{R}{T_1}$}
\AxiomC{$e_2 : \pathendpoint{A}{R}{T_2}$}
\TrinaryInfC{$\hpairprle : \homotendpoint{l}{r}{a}{b}{\comp{\pair{e_1}{e_2}}{\prle}}{e_1}, \quad \hpairprre : \homotendpoint{l}{r}{a}{b}{\comp{\pair{e_1}{e_2}}{\prre}}{e_2}$}
\end{bprooftree}
\end{center}

\vspace{5pt}

\begin{center}
\begin{bprooftree}
\def\defaultHypSeparation{\hskip .037in}
\AxiomC{$T_1, T_2 : \poly$}
\AxiomC{$e_1, e_2 : \pathendpoint{A}{R}{T_1}$}
\AxiomC{$e_3, e_4 : \pathendpoint{A}{R}{T_2}$}
%\noLine
\AxiomC{$h_1 : \homotendpoint{l}{r}{a}{b}{e_1}{e_2}$}
\AxiomC{$h_2 : \homotendpoint{l}{r}{a}{b}{e_3}{e_4}$}
\QuinaryInfC{$\hpair{h_1}{h_2} : \homotendpoint{l}{r}{a}{b}{\pair{e_1}{e_3}}{\pair{e_2}{e_4}}$}
\end{bprooftree}
\end{center}

\vspace{5pt}

\begin{center}
\begin{bprooftree}
\AxiomC{$T_1, T_2 , T_3: \poly$}
\AxiomC{$e_1 : \pathendpoint{A}{R}{T_1}$}
\AxiomC{$e_2 : \pathendpoint{A}{T_1}{T_2}$}
\AxiomC{$e_3 : \pathendpoint{A}{T_1}{T_3}$}
\QuaternaryInfC{$\hcomppair : \homotendpoint{l}{r}{a}{b}{\comp{e_1}{\pair{e_2}{e_3}}}{\pair{\comp{e_1}{e_2}}{\comp{e_1}{e_3}}}$}
\end{bprooftree}
\end{center}

\vspace{5pt}

\begin{center}
\begin{bprooftree}
\AxiomC{$T: \poly$}
\AxiomC{$X: \onetypes$}
\AxiomC{$x: X$}
\AxiomC{$e : \pathendpoint{A}{R}{T}$}
\QuaternaryInfC{$\hcompconst e : \homotendpoint{l}{r}{a}{b}{\comp{e}{\Ce x}}{\Ce x}$}
\end{bprooftree}
\end{center}

\vspace{5pt}

\begin{center}
%% \begin{bprooftree}
%% \AxiomC{$T_1, T_2 : \poly$}
%% \AxiomC{$e_1, e_2 : \pathendpoint{A}{R}{T_1}$}
%% \AxiomC{$h : \homotendpoint{l}{r}{a}{b}{e_1}{e_2}$}
%% \TrinaryInfC{$\hinl{h} : \homotendpoint{l}{r}{a}{b}{\comp{e_1}{\inle}}{\comp{e_2}{\inle}}$}
%% \end{bprooftree}
\begin{bprooftree}
\AxiomC{$j : J$}
\AxiomC{$e : \pathendpoint{A}{R}{Q_j}$}
\BinaryInfC{$\hconstr{j}{e} : \homotendpoint{l}{r}{a}{b}{\comp{e}{l(j)}}{\comp{e}{r(j)}}$}
\end{bprooftree}
%% \vspace{5pt}
%% \begin{bprooftree}
%% \AxiomC{$T_1, T_2 : \poly$}
%% \AxiomC{$e_1, e_2 : \pathendpoint{A}{R}{T_2}$}
%% \AxiomC{$h : \homotendpoint{l}{r}{a}{b}{e_1}{e_2}$}
%% \TrinaryInfC{$\hinr{h} : \homotendpoint{l}{r}{a}{b}{\comp{e_1}{\inre}}{\comp{e_2}{\inre}}$}
%% \end{bprooftree}
\begin{bprooftree}
\AxiomC{$\vphantom{e : \pathendpoint{A}{R}{Q_j}}$}
\UnaryInfC{$\harg : \homotendpoint{l}{r}{a}{b}{a}{b}$}
\end{bprooftree}
\end{center}

\caption{Rules for the homotopy endpoints.}
\label{fig:homot_ep}
\end{figure*}

\begin{definition}
\label{def:homotep}
Suppose that we have
\begin{itemize}
	\item A polynomial $A$;
	\item A type $J$ together with for each $j : J$ a polynomial $Q_j$ and endpoints $l_j, r_j : \pathendpoint{A}{Q_j}{\idP}$;
	\item A polynomial $R$;
	\item A polynomial $T$ with endpoints $a, b : \pathendpoint{A}{R}{T}$;
	\item A polynomial $W$ with endpoints $s, t : \pathendpoint{A}{R}{W}$.
\end{itemize}
Then we define the type $\homotendpoint{l}{r}{a}{b}{s}{t}$ of \fat{homotopy endpoint} inductively by the constructors in Figure \ref{fig:homot_ep}.
\end{definition}

There are three homotopy endpoints of particular importance.
The first one is $\constructor{path}$, which represents the path constructor and it makes use of $l_j$ and $r_j$.
The second one is $\harg$, which represents the path argument and it uses $a$ and $b$.
The last one is $\constructor{ap}$ and it corresponds to the action of an endpoint on a homotopy endpoint.

The way we represent path arguments allows us to specify equations with any finite number of path arguments by only two path endpoints.
For example, two path arguments $p : x_1 = y_1$ and $q : x_2 = y_2$ is represented by one path argument of type $(x_1, x_2) = (y_1, y_2)$.

From the grammar in Figure \ref{fig:homot_ep}, we can derive
the following additional homotopy endpoints, which will be employed in forthcoming
examples. 

\vspace{5pt}

\begin{center}
\begin{bprooftree}
\AxiomC{$T_1, T_2 : \poly$}
\AxiomC{$e_1, e_2 : \pathendpoint{A}{R}{T_1}$}
\AxiomC{$e_3, e_4 : \pathendpoint{A}{R}{T_2}$}
\AxiomC{$h : \homotendpoint{l}{r}{a}{b}{\pair{e_1}{e_3}}{\pair{e_2}{e_4}}$}
\QuaternaryInfC{$\hprl{h} \eqdef \hconcat{\hinv{\hpairprle}}{\hconcat{\hap {\prle} h}{\hpairprle}} : \homotendpoint{l}{r}{a}{b}{e_1}{e_2}$}
\alwaysNoLine
\UnaryInfC{$\hprr{h} \eqdef \hconcat{\hinv{\hpairprre}}{\hconcat{\hap {\prre} h}{\hpairprre}} : \homotendpoint{l}{r}{a}{b}{e_3}{e_4}$}
\end{bprooftree}
\end{center}

\vspace{5pt}
Moreover, we have a function $\idtoH$ sending a path between endpoints $e_1 = e_2$ into an homotopy endpoint $\homotendpoint{l}{r}{a}{b}{e_1}{e_2}$, readily definable by path induction.

Given a type $X$ with a function $c : \polyAct{A}{X} \rightarrow X$ and for each $x : Q_j(X)$ a path $l_j(x) = r_j(x)$,
a homotopy endpoint $p : \homotendpoint{l}{r}{a}{b}{s}{t}$ gives rise for each point $x : \polyAct{R}{X}$ and path $w : a(x) = b(x)$
to another path $\homotendpointAct{p}{x, w} : s(x) = t(x)$.
Hence, two homotopy endpoints $p, q : \homotendpoint{l}{r}{a}{b}{s}{t}$ represent the equation
\[
\depprod{(x : \polyAct{R}{X}) (w : a(x) = b(x))}{\homotendpointAct{p}{x, w} = \homotendpointAct{q}{x, w}}
\]
Now let us put this all together and define what signatures for higher inductive types are.

\begin{definition}
\label{def:signature}
A \fat{HIT-signature} $\sign$ consists of
\begin{itemize}
	\item A polynomial $\pointconstr[\sign]$;
	\item A type $\pathlabel[\sign]$ together with for each $j : \pathlabel[\sign]$ a polynomial $\patharg[\sign]_j$ and endpoints $\pathleft[\sign]_j, \pathright[\sign]_j : \pathendpoint{\pointconstr[\sign]}{\patharg[\sign]_j}{\idP}$;
	\item A type $\homotlabel[\sign]$ together with for each $j : \homotlabel[\sign]$ polynomials $\homotpointarg[\sign]_j$ and $\homotpathtarg[\sign]_j$,
	endpoints $\pathargleft[\sign]_j, \pathargright[\sign]_j : \pathendpoint{\pointconstr[\sign]}{\homotpointarg[\sign]_j}{\homotpathtarg[\sign]_j}$
	and $\homotpathleft[\sign]_j, \homotpathright[\sign]_j : \pathendpoint{\pointconstr[\sign]}{\homotpointarg[\sign]_j}{\idP}$,
	and homotopy endpoints $\homotleft[\sign]_j, \homotright[\sign]_j : \homotendpoint{\pathleft[\sign]}{\pathright[\sign]}{\pathargleft[\sign]_j}{\pathargright[\sign]_j}{\homotpathleft[\sign]_j}{\homotpathright[\sign]_j}$.
\end{itemize}
\end{definition}

If $\sign$ is clear from the context, we do not write the superscript.
In the remainder, we show how to interpret the following HIT given a signature $\sign$:

\begin{lstlisting}[mathescape=true]
Inductive $H$ :=
| $c$ : $\functions{\polyAct{\pointconstr}{H}}{H}$
| $p$ : $\depprod{(j : \pathlabel) (x : \polyAct{\patharg_j}{H})}{\pathendpointAct{\pathleft_j}{x} = \pathendpointAct{\pathright_j}{x}}$
| $s$ : $\depprod{(j : \homotlabel) (x : \polyAct{\homotpointarg_j}{H}) (r : \pathargleft_j(x) = \pathargright_j(x))}{\homotendpointAct{\homotleft_j}{x, r} = \homotendpointAct{\homotright_j}{x, r}}$
| $t$ : $\depprod{(x, y : H) (q_1, q_2 : x = y) (r_1, r_2 : q_1 = q_2)}{r_1 = r_2}$
\end{lstlisting}

Next we consider three examples of HITs we can express with these signatures.

\begin{example}
\label{ex:torus}
The torus is described by the signature $\torus$.
\begin{itemize}
	\item Take $\pointconstr[\torus] \eqdef \constantP{\unit}$;
	\item Take $\pathlabel[\torus] \eqdef \bool$ and for both inhabitants we take $\patharg[\torus] \eqdef \constantP{\unit}$ and $\pathleft[\torus] \eqdef \pathright[\torus] \eqdef \constr$;
	\item Take $\homotlabel[\torus] \eqdef \unit$.
	Since there are no arguments for this path constructor, we take $\homotpointarg[\torus] \eqdef \homotpathtarg[\torus] \eqdef \constantP{\unit}$ and $\pathargleft[\torus] \eqdef \pathargright[\torus] \eqdef \Ce(\unitt)$.
	Now for the left-hand side and right-hand side of this equation, we take $\hconcat{\hconstr{\booltrue}{\idENA}}{\hconstr{\boolfalse}{\idENA}}$ and $\hconcat{\hconstr{\boolfalse}{\idENA}}{\hconstr{\booltrue}{\idENA}}$ respectively. 
\end{itemize}
Notice that the usual presentation of the torus does not include the
explicit 1-truncation constructor $t$, since $\torus$ is already provably
1-truncated without its presence. A similar consideration applies to
the circle $\circleS$.
\end{example}

\begin{example}
\label{ex:mod}
We represent the integers modulo 2 as the following HIT:
\begin{lstlisting}[mathescape=true]
Inductive $\ZT$ :=
| $\ZZ$ : $\ZT$
| $\ZS$ : $\ZT \rightarrow \ZT$
| $\ZM$ : $\depprod{(x : \ZT)}{\ZS(\ZS(x)) = x}$
| $\ZC$ : $\depprod{(x : \ZT)}{\ZM(\ZS(x)) = \ap{\ZS}{(\ZM(x))}}$
\end{lstlisting}
Note that all constructors except $\ZZ$ are recursive.
We define a signature $\ZT$.
\begin{itemize}
	\item Take $\pointconstr[\ZT] \eqdef \sumP{\constantP{\unit}}{\idP}$;
	\item Take $\pathlabel[\ZT] \eqdef \constantP{\unit}$ and for its unique inhabitant we take $\patharg[\ZT] \eqdef \idP$ and
	\[
	\pathleft[\ZT] \eqdef \comp{(\comp{\inre}{\constr})}{(\comp{\inre}{\constr})}, \quad
	\pathright[\ZT] \eqdef \idENA;
	\]
	\item Take $\homotlabel[\ZT] \eqdef \constantP{\unit}$.
	Furthermore, we take $\homotpointarg[\ZT] \eqdef \idP$ and $\pathargleft[\ZT] \eqdef \pathargright[\ZT] \eqdef \Ce(\unitt)$.
	The endpoints $\homotpathleft$ and $\homotpathright$ encode $\ZS(\ZS(\ZS(x)))$ and $\ZS(x)$ respectively,
	and for the left-hand side and right-hand side of this equation, we take
	\[
	\hap{\constr}{(\hconcat{\hinv{\hlunitN}}{\hconcat{\hassocN}{\hconcat{\hap{\inre}{(\hconstr{\unit}{\idENA})}}{\hconcat{\hinv{\hassocN}}{\hconcat{\hlunitN}{\hlunitN}}}}})}
	\]
	\[
	\hconcat{\hinv{\hassocN}}{\hconcat{\hinv{\hassocN}}{\hconcat{\hconstr{\unitt}{\comp{\inre}{\constr}}}{\hrunitN}}}.
	\]
	respectively.
	Note that we use $\hassocN$, $\hlunitN$, and $\hrunitN$ to make the equations type check.
	If we would interpret the left-hand side and right-hand side of  the homotopy constructor in 1-types,
	then all occurrences of $\hassocN$, $\hlunitN$, and $\hrunitN$ become the identity path.
	We thus get the right homotopy constructor.
\end{itemize}
\end{example}

\begin{example}
\label{ex:settrunc}
Given a 1-type $A$,  the set truncation of $A$ is defined by the following HIT:
\begin{lstlisting}[mathescape=true]
Inductive $\ST{A}$ :=
| $\SC$ : $A \rightarrow \ST{A}$
| $\Strunc$ : $\depprod{(x, y : \ST{A}) (p, q : x = y)}{p = q}$
\end{lstlisting}
Note that this higher inductive type has a parameter $A$,
so the signature we define depends on a 1-type $A$ as well.
To encode the path arguments of $\Strunc$, we use the fact that giving two paths $p, q : x = y$ is the same as giving a path $r : (x, x) = (y, y)$.
Define a signature $\ST{A}$ such that
\begin{itemize}
	\item $\pointconstr[\ST{A}] \eqdef \constantP{A}$;
	\item $\pathlabel[\ST{A}]$ is the empty type;
	\item $\homotlabel[\ST{A}] \eqdef \unit$.
	In addition, there are two point arguments $\homotpointarg[\ST{A}] \eqdef \prodP{\idP}{\idP}$
	and a path argument with left-hand side $\pair{\prle}{\prle}$ and right-hand side $\pair{\prre}{\prre}$.
	For the left-hand side and right-hand side of the homotopy, we take $\hprl{\harg}$ and $\hprr{\harg}$ respectively.
\end{itemize}
\end{example}


More examples of HIT signatures are discussed in Section \ref{sec:examples}.

\subsection{Algebras in 1-types and groupoids}
With the signatures in place, our next goal is to study the introduction rules of HITs and for that, we define bicategories of algebras for a signature.
Since we ultimately want to construct HITs via the groupoid quotient, we look at both algebras in 1-types and groupoids.

In both cases, we use a stratified approach with displayed bicategories.
Let us illustrate this by briefly describing the construction for 1-types.
On $\onetypes$, we define a displayed bicategory and we denote its total bicategory by $\prealg{\sign}$.
The objects of $\prealg{\sign}$ consist of a 1-type $X$ together with an operation $\polyAct{\pointconstr[\sign]}{X} \rightarrow X$.
Concretely, the objects satisfy the point introduction rules specified by $\sign$.
On top of $\prealg{\sign}$, we define another displayed bicategory whose total bicategory is denoted by $\pathalg{\sign}$.
Objects of $\pathalg{\sign}$ satisfy the introduction rules for both the points and the paths.
Lastly, we take a full subbicategory of $\pathalg{\sign}$ obtaining another bicategory $\alg(\sign)$ whose objects satisfy the introduction rules for the points, paths and homotopies.

To define $\prealg{\sign}$, we use Example \ref{ex:DFAlg}.

\begin{problem}
\label{prob:sem_poly}
Given $P : \poly$, to construct pseudofunctors
\[
\semP{P} : \pseudo(\onetypes, \onetypes), 
\semPG{P} : \pseudo(\grpd, \grpd).
\]
\end{problem}

\begin{construction}{prob:sem_poly}
We only discuss the case for 1-types since the case for groupoids is similar.
Given a polynomial $P$ and a type $X$, we get a type $\polyAct{P}{X}$ by induction.
The verification that this gives rise to a pseudofunctor can be found in the formalization.
\end{construction}

\begin{definition}
\label{def:prealg}
Let $\sign$ be a signature.
Then we define the bicategories $\prealg{\sign}$ and $\prealgG{\sign}$
to be the total bicategories of $\DFAlg(\semP{\pointconstr[\sign]})$
and $\DFAlg(\semPG{\pointconstr[\sign]})$ respectively.
Objects of these bicategories are called \fat{prealgebras} for $\sign$.
\end{definition}

Note that prealgebras only have structure witnessing the introduction rule for the points.
Next we look at the introduction rule for the paths.
In this case, the desired structure is added via Example \ref{ex:DCell} and to apply this construction,
we interpret path endpoints as pseudotransformations.

\begin{problem}
\label{prob:sem_endpoint}
Given $e : \pathendpoint{A}{P}{Q}$, to construct pseudotransformations
\[
\semE{e} : \pstrans{\dproj{\DFAlg(\semP{A})} \cdot \semP{P}}{\dproj{\DFAlg(\semP{A})} \cdot \semP{Q}},
\]
\[
\semEG{e} : \pstrans{\dproj{\DFAlg(\semPG{A})} \cdot \semPG{P}}{\dproj{\DFAlg(\semPG{A})} \cdot \semPG{Q}}.
\]
\end{problem}

\begin{construction}{prob:sem_endpoint}
We only discuss $\semE{e}$ since $\semEG{e}$ is defined similarly.
Given a 1-type $X$ and $c : \polyAct{A}{X} \rightarrow X$, we define the function
$\semE{e} : \polyAct{P}{X} \rightarrow \polyAct{Q}{X}$ by induction.
The verification that this gives rise to a pseudotransformation can be found in the formalization.
\end{construction}

\begin{definition}
\label{def:pathalg}
Let $\sign$ be a signature.
We use Examples \ref{ex:disp_depprod} and \ref{ex:DCell} to define displayed bicategories over $\prealg{\sign}$ and $\prealgG{\sign}$.
\[
\pathalgMD(\sign) \eqdef \depprod{(i : \pathlabel[\sign])}{\DCell(\semE{\pathleft[\sign](i)},\semE{\pathright[\sign](i)})}
\]
\[
\pathalgGD{\sign} \eqdef \depprod{(i : \pathlabel[\sign])}{\DCell(\semEG{\pathleft[\sign](i)},\semEG{\pathright[\sign](i)})}
\]
We define $\pathalgM(\sign)$ and $\pathalgG{\sign}$ to be the total bicategories of $\pathalgMD(\sign)$ and $\pathalgGD{\sign}$ respectively.
Objects of $\pathalgM(\sign)$ and $\pathalgG{\sign}$ are called \fat{path algebras} for $\sign$.
\end{definition}

\begin{problem}
\label{prob:sem_homendpoint}
Suppose that we have a homotopy endpoint $h : \homotendpoint{l}{r}{a}{b}{s}{t}$.
Given a 1-type $X$ with $c : \polyAct{A}{X} \rightarrow X$ and $p : \depprod{(j : J) (x : \polyAct{Q_j}{X})}{l_j(x) = r_j(x)}$,
to construct for each $x : \polyAct{Q}{X}$ and $w : \semE{a}(x) = \semE{b}(x)$
an equality $\semH{h}(x, w) : \semE{s}(x) = \semE{t}(x)$.

In addition, given a groupoid $G$ together with a functor $c : \semPG{A}{(G)} \rightarrow G$ and for each $j : J$ a natural transformation $\semEG{l_j}(G) \twocell \semEG{r_j}(G)$,
to construct for each object $x : \semPG{Q}{(G)}$ and morphism $w : \semEG{a}(G)(x) \rightarrow \semEG{b}(G)(x)$
a morphism $\semHG{h}(x, w) : \semEG{s}(G)(x) \rightarrow \semEG{t}(G)(x)$.
\end{problem}

\begin{construction}{prob:sem_homendpoint}
By induction.
\end{construction}

\begin{definition}
\label{def:bicat_grpd}
Let $\sign$ be a HIT signature.
We define $\algM(\sign)$ to be the full subbicategory of $\pathalgM(\sign)$
in which every object $X$ satisfies
\begin{equation*}
\begin{split}
\prod
(j : \homotlabel[\sign]) (x : \polyAct{\homotpointarg[\sign]_j}{X}) (w : \semE{\pathargleft[\sign]_j}(x) = \semE{\pathargright[\sign]_j}(x)),
\ \semH{\homotleft[\sign]_j}(x, w) = \semH{\homotright[\sign]_j}(x, w)
\end{split}
\end{equation*}
In addition, we define $\algG{\sign}$ to be the full subbicategory of $\pathalgG{\sign} $
in which every object $X$ satisfies
\begin{equation*}
\begin{split}
\prod
(j : \homotlabel[\sign]) (x : \semPG{\homotpointarg[\sign]_j}{X}) (w : \semEG{\pathargleft[\sign]_j}(x) = \semEG{\pathargright[\sign]_j}(x)),\
\semHG{\homotleft[\sign]_j}(x, w) = \semHG{\homotright[\sign]_j}(x, w)
\end{split}
\end{equation*}
Objects of $\algM(\sign)$ and $\algG{\sign}$ are called \fat{algebras} for $\sign$.
\end{definition}

The bicategory $\algM(\sign)$ is constructed by repeatedly using Definition \ref{def:totalbicat}.
By unpacking the definition, we see that an algebra $X : \algM(\sign)$ consists of
\begin{itemize}
	\item A 1-type $X$;
	\item A function $\AlgPoint{X} : \polyAct{\pointconstr}{X} \rightarrow X$;
	\item For each $j : \pathlabel$ and point $x : \polyAct{\patharg_j}{X}$ a path
	$
	\AlgPath{X}{j}(x) : \semE{\pathleft_j}(x) = \semE{\pathright_j}(x);
	$
	\item For each $j : \homotlabel$, $x : \polyAct{\homotpointarg}{X}$ and $w : \semE{a_1}(x) = \semE{a_2}(x)$, a homotopy
	$
	\AlgHomot{X}{j} : \semH{\homotleft}(x, w) = \semH{\homotright}(x, w)
	$
\end{itemize}
Furthermore, given two algebras  $X, Y : \algM(\sign)$, an algebra morphism $f : X \onecell Y$ consists of
\begin{itemize}
	\item a map $f : X \rightarrow Y$;
	\item for each $x : \polyAct{\pointconstr[\sign]}{X}$ a path
	\[
	\AlgMapPoint{f}(x) : f(\AlgPoint{X}(x)) = \AlgPoint{Y}(\semP{\pointconstr[\sign]}(f)(x));
	\]
	\item for each $j : \pathlabel[\sign]$ and $x : \polyAct{\patharg[\sign]}{X}$ a 2-path
	\[
	\AlgMapPath{f}{j}(x) :
	\ap{f}{(\AlgPath{X}{j}(x))} \vcomp \semE{\pathleft[\sign]_j}(\AlgMapPoint{f})(x)
	=
	\semE{\pathleft[\sign]_j}(\AlgMapPoint{f})(x) \vcomp \AlgPath{Y}{j}(\semP{\patharg[\sign]}(f)(x)).
	\]
\end{itemize}
Lastly, given two algebras $X, Y : \algM(\sign)$ and an algebra morphisms $f, g : X \onecell Y$, a 2-cell $\theta : f \twocell g$ in $\algM(\sign)$ consists of
\begin{itemize}
	\item for each $x : X$ a path $\theta : f(x) = g(x)$;
	\item for each $x : \polyAct{\pointconstr[\sign]}{X}$ a path
	\[
	\AlgCellPoint{\theta}(x) : 
	\theta(x) \vcomp \AlgMapPoint{g}(x)
	=
	\AlgMapPoint{f}(x)
	\vcomp
	\ap{\AlgPoint{Y}}{(\semP{\pointconstr[\sign]}(\theta)(x))}
	\]
\end{itemize}


\section{Induction and Biinitiality}
\label{sec:induction}
The algebra structure only represents the introduction rule
and the next step is to define the elimination and computation rules for higher inductive types.
Before we can formulate these principles, we need to define dependent actions of polynomials, path endpoints, and homotopy endpoints.
All of these constructions are done by induction and details can be found in the literature \cite{DBLP:journals/entcs/DybjerM18,hermida1998structural,van2019construction}.

\begin{problem}
\label{prob:poly_dact}
Given a type $X$, a type family $Y$ on $X$, and a polynomial $P$,
to construct a type family $\polyDact{P}{Y}$ on $\polyAct{P}{X}$.
\end{problem}

\begin{problem}
\label{prob:poly_dmap}
Given a type $X$, a type family $Y$ on $X$, a polynomial $P$, and a map $f : \depprod{(x : X)}{Y(x)}$,
to construct a map
$
\polyDmap{P}{f} : \depprod{(x : \polyAct{P}{X})}{\polyDact{P}{Y}(x)}.
$
\end{problem}

\begin{problem}
\label{prob:poly_pr}
Given a type $X$, a type family $Y$ on $X$, and a polynomial $P$, to construct a map
$
\polyoplax P : \polyAct{P}{\Sum (x : X). Y(x)} \rightarrow \Sum (x : \polyAct{P}{X}). \polyDact{P}{Y}(x)
$
such that, for all $z :  \polyAct{P}{\Sum (x : X). Y(x)}$, we have $\projl(\polyoplax{P}(z)) = \polyAct{P}{\projl}(z)$.
\end{problem}

\begin{problem}
\label{prob:endpoint_dact}
Given a type $X$, a type family $Y$ on $X$, an endpoint $e : \pathendpoint{A}{P}{Q}$, and a map $c : \polyAct{A}{X} \rightarrow X$,
to construct for each $x : \polyAct{P}{X}$ and $y : \polyDact{P}{Y}(x)$ an inhabitant $\pathendpointDact{e}{y} : \polyDact{Q}{Y}(\pathendpointAct{e}{x})$.
\end{problem}

\begin{problem}
\label{prob:endpoint_dact_natural}
Suppose, that we have polynomials $A, P, Q$, a type $X$ with a map $c_X : \polyAct{A}{X} \rightarrow X$,
and a type family $Y$ on $X$ with a map $c_Y : \depprod{(x : X)}{\polyDact{A}{Y}(x) \rightarrow Y(c_X(x))}$
and a map $f : \depprod{(x : X)}{Y(x)}$.
Given an endpoint $e : \pathendpoint{A}{P}{Q}$,
to construct an equality
\[
\pathendpointDnat{e}{f} : \polyDmap{Q}{f}(\semE{e}(x)) = \pathendpointDact{e}{\polyDmap{P}{f}(x)}.
\]
\end{problem}

\begin{problem}
\label{prob:homot_dact}
Let $\sign$ be a signature.
Let $X$ be a type with a function $c_X : \polyAct{\pointconstr}{X} \rightarrow X$
and with for each $j : \pathlabel$ and $x : \polyAct{\patharg}{X}$ a path $p_X(j, x) : \semE{\pathleft}(x) = \semE{\pathright}(x)$.
In addition, suppose that $Y$ is a type family on $X$,
that we have a function $c_Y : \depprod{(x : \polyAct{\pointconstr}{X})}{\polyDact{\pointconstr}{Y}(x) \rightarrow Y(c_X(x))}$,
and that for all $j  : \pathlabel$ and points $x : \polyAct{\patharg}{X}$ and $\xx : \polyDact{\patharg}{Y}(x)$,
we have a path $p_Y : \pathover{p_X(j, x)}{\pathendpointDact{\pathleft}{\xx}}{\pathendpointDact{\pathright}{\xx}}$.
Furthermore, let $x : \homotpointarg$ and $\xx : \polyDact{\homotpointarg}{Y}(x)$ be points
and let $w : \semE{\pathargleft}(x) = \semE{\pathargright}(x)$
and $\disp{w} : \pathover{p}{\pathendpointDact{\pathargleft}{\xx}}{\pathendpointDact{\pathargright}{\xx}}$
be paths.
Then for each homotopy endpoint $h : \homotendpoint{l}{r}{a}{b}{s}{t}$, 
to construct a path
\[
\homotendpointDact{h}{\xx, \disp{w}} : \pathover{\homotendpointAct{h}{x, w}}{\pathendpointDact{\homotpathleft}{\xx}}{\pathendpointDact{\homotpathright}{\xx}}.
\]
\end{problem}

With these notions in place, we define \emph{displayed algebras}.
These represent the input of the elimination rule.
Furthermore, we show that each displayed algebra gives rise to a total algebra and a projection.

\begin{definition}\label{def:disp_alg}
Given a signature $\sign$ and an algebra $X$ for $\sign$,
a \fat{displayed algebra} $Y$ over $X$ consists of
\begin{itemize}
	\item A family $Y$ of 1-types over $X$;
	\item For each $x : \polyAct{\pointconstr}{X}$ a map $\DispAlgPoint{Y} : \polyDact{\pointconstr}{Y}(x) \rightarrow Y(\AlgPoint{X}(x))$;
	\item For each $j : \pathlabel$, $x : \polyAct{\patharg_j}{X}$, and $\pover{x} : \polyDact{\patharg_j}{Y}(x)$, a path
	$
	\DispAlgPath{Y}{j} : \pathover{\AlgPath{X}{j}{x}}{\pathendpointDact{l_j}{\pover{x}}}{\pathendpointDact{r_j}{\pover{x}}}
	$;
	\item For each $j : \homotlabel$, points $x : \polyAct{\homotpointarg}{X}$ and $\pover{x} : \polyDact{\homotpointarg}{Y}(x)$,
	and paths $w : \pathargleft(x) = \pathargright(x)$ and $\pover{w} : \pathover{w}{\pathendpointDact{\pathargleft}{\pover{x}}}{\pathendpointDact{\pathargright}{\pover{x}}}$,
	a globe
	$
	\DispAlgHomot{Y}{j} : \globeover{\AlgHomot{X}{j}{(x, w)}}{\homotendpointDact{\homotleft}{\xx}}{\homotendpointDact{\homotright}{\xx}}
	$
	over $\AlgHomot{X}{j}{(x, w)}$.
\end{itemize}
\end{definition}

\begin{definition}\label{def:total_alg}
Let $X$ be an algebra for a signature $\sign$ and let $Y$ be a
displayed algebra over $X$. The \fat{total algebra} of $Y$ is the
algebra $\TotalAlg Y$ with carrier $\Sum (x : X). Y(x)$ and function
$\AlgPoint{\TotalAlg Y}$ acting on an element $z
: \polyAct{\pointconstr}{\Sum (x : X). Y(x)}$ as follows:
\[
\AlgPoint{\TotalAlg Y}(z) \eqdef
(\AlgPoint{X}(\projl(\polyoplax{\pointconstr}(z)))
,
\DispAlgPoint{Y}(\projr(\polyoplax{\pointconstr}(z)))).
\]
We refer to the formalization for details on the construction of the
required paths and homotopies. The first projection $\TotalAlg
Y \rightarrow X$ is a morphism of algebras.
\end{definition}

\begin{remark}
The type family of a displayed algebra is required to be 1-truncated.
This means that the HITs we construct, can only be eliminated into
1-types, and as a consequence, these HITs only have the right
elimination principle with respect to 1-types.
\end{remark}

The output of the elimination rule and the computation rules are given by a \emph{section} to be defined in Definition \ref{def:section} below.
One might expect that, just like for the groupoid quotient, the computation rules for the paths
are given as globes over some 2-path in the base (Definition \ref{def:globe_over}).
However, this is not the case.

This is because there is a slight discrepancy between the rules for the groupoid quotient and the HITs we discuss,
namely for the former the computation rules for the points are definitional equalities
while for the latter, these rules only hold propositionally.
This affects how we need to formulate the computation rules for the paths.

Let us illustrate this via the torus (Example \ref{ex:torus}).
The input for the elimination rule consists, among others, of a type family $Y$, a point $b : Y(\base)$,
and a path $p_l : \pathover{\leftLoop}{b}{b}$.
The elimination rule gives a map $f : \depprod{(x : \torus)}{Y(x)}$.
By the point computation rule, we have a propositional equality between $f(\base)$ and $b$.
Now the computation rule for $\leftLoop$ ought to equate $\apd{f}{\leftLoop}$ and $p_l$.
However, such an equation does not type check since $\apd{f}{\leftLoop}$ has type $\pathover{\leftLoop}{f(b)}{f(b)}$ while $p_l$ has type $\pathover{\leftLoop}{b}{b}$.
In conclusion, we cannot formulate the computation rules the same way as we did for the grorupoid quotient.

Our solution to this problem is to define a type of \emph{squares} over a given 2-path similarly to Definition \ref{def:globe_over}.

\begin{definition}
Let $X$ be a type and let $Y$ be a type family on $X$.
Suppose that we are given points $x_1, x_2 : X$ and $\overline{x_1}, \overline{x_1}' : Y(x_1)$ and $\overline{x_2}, \overline{x_2}' : Y(x_2)$, paths $p, q : x_1 = x_2$ together with paths $\pover{p} : \pathover{p}{\overline{x_1}}{\overline{x_2}}$ and $\pover{q} : \pathover{q}{\overline{x_1}'}{\overline{x_2}'}$ over $p$ and $q$ respectively.
If we also have two paths $h_1 : \overline{x_1} = \overline{x_1}'$ and $h_2 : \overline{x_2} = \overline{x_2}'$ and a 2-path $g : p = q$,
then we define the type of \fat{squares} over $g$ from $\pover{p}$ to $\pover{q}$ with sides $h_1$ and $h_2$ by path induction.
\end{definition}

\begin{definition}
\label{def:section}
Let $X$ be an algebra for a given signature $\sign$ and let $Y$ be a displayed algebra over $X$.
Then a \fat{section} of $Y$ consists of
\begin{itemize}
	\item A map $f : \depprod{(x : X)}{Y(x)}$;
	\item For all $x : \polyAct{\pointconstr}{X}$, an equality $f(\AlgPoint{X}(x)) = \DispAlgPoint{Y}(\polyDmap{\pointconstr}{f}(x))$;
	\item For all $j : \pathlabel$ and $x : \polyAct{\patharg}{X}$, a square from
	$\apd{f}{(\AlgPath{X}{j}(x))}$
	to
	$\DispAlgPath{Y}{j}(\polyDmap{\patharg}{f}(x))$
	with sides
	$\pathendpointDnat{\pathleft}{f}(x)$
	and
	$\pathendpointDnat{\pathright}{f}(x)$.
\end{itemize}
\end{definition}

\begin{definition}
Let $\sign$ be a signature and let $X$ be an algebra for $\sign$.
Then we say that $X$ is a \fat{1-truncated higher inductive type} for $\sign$ if each displayed algebra $Y$ over $X$ has a section.
\end{definition}

Often we just say that $X$ is a HIT for $\sign$ instead of saying that $X$ is 1-truncated HIT.
With this in place, we can check whether our rules for higher inductive types
agree for the usual examples with the rules given in the literature \cite{hottbook}.
We illustrate this with the torus (Example \ref{ex:torus}) and the set truncation (Example \ref{ex:settrunc}).
In the next example, we write $p \vcomp q$ for the concatenation of dependent paths.

\begin{example}[Example \ref{ex:torus} cont'd]
Recall the signature $\torus$ for the torus.
Let $X$ be a HIT for $\torus$.
Since $X$ is an algebra, we have a point $\base : X$, two paths $\leftLoop, \rightLoop : \base = \base$,
and a 2-path $\surface : \concat{\leftLoop}{\rightLoop} = \concat{\rightLoop}{\leftLoop}$.
This corresponds precisely to the usual introduction rules of the torus.

A family $Y$ of 1-types on $X$ together with a point $b : Y(\base)$,
two paths $l : \pathover{\leftLoop}{b}{b}$ and $r : \pathover{\rightLoop}{b}{b}$
and a globe $h : \globeover{\surface}{l \vcomp r}{r \vcomp l}$
over $\surface$ gives rise to a displayed algebra over $X$.
This corresponds to the usual input of the elimination rule of the torus.
If we have a section $s$ of $Y$, then in particular, we get a map $f_s : \depprod{(x : X)}{Y(x)}$.
We also get a path $p_s : f(\base) = b$, a square from $\apd{f}{\leftLoop}$ to $l$ and one from $\apd{f}{\rightLoop}$ to $r$.
Both squares have sides $p_s$ and $p_s$.
These are the computation rules for the points and paths of the torus.
Note that since we are looking at $1$-truncated HITs, this only gives the 1-truncation of the torus.
\end{example}

\begin{example}[Example \ref{ex:settrunc} cont'd]
Let $A$ be a 1-type and recall the signature $\ST{A}$.
Now let $X$ be a HIT on $\ST{A}$.
Note that an algebra for $\ST{A}$ consists of a type $Z$ together with a map $A \rightarrow Z$
and a proof that $Z$ is a set.
This means in particular, that we have a map $\SC : A \rightarrow X$ and a proof $\Strunc$
that $X$ is a set.

A family $Y$ of sets on $X$ together with a map $i : \depprod{(a  : A)}{Y(\SC(A))}$
give rise to a displayed algebra over $X$.
A section $s$ of that displayed algebra consists of a map $f_s : \depprod{(x : X)}{Y(x)}$
such that $f_s(\SC(a)) = i(a)$ for all $a : A$.
This corresponds to the usual elimination and computation rules for the
set truncation.
\end{example}

To verify that an algebra satisfies the elimination rule, we use \emph{initial algebra semantics} \cite{hermida1998structural}.
However, this technique is usually applied in a categorical setting and it uses initial objects in categories.
Since we work in a bicategorical setting, we need to use the corresponding notion in bicategory theory: \emph{biinitiality}.

\begin{definition}
Let $\B$ be a bicategory and let $x$ be an object in $\B$.
Then we say $x$ is \fat{biinitial} if
\begin{itemize}
	\item For each object $y$ there is a 1-cell $x \onecell y$;
	\item Given 1-cells $f, g : x \onecell y$, there is a 2-cell $f \twocell g$;
	\item Given 2-cells $\tc, \tc' : f \twocell g$, there is an equality $\tc = \tc'$.
\end{itemize}
\end{definition}

Briefly, an object $x$ is biinitial if for each $y$ there is 1-cell from $x$ to $y$, which is unique up to a unique 2-cell.
Now we can formulate initial algebra semantics for our signatures.

\begin{proposition}
\label{thm:initial_alg_sem}
Let $\sign$ be a signature and let $X$ be an algebra for $\sign$.
Then
\begin{itemize}
	\item If $X$ is a 1-truncated HIT for $\sign$, then $X$ is biinitial in $\algM(\sign)$.
	\item If $X$ is biinitial, then $X$ is a 1-truncated HIT for $\sign$.
\end{itemize}
\end{proposition}

One consequence of initial algebra semantics, is that HITs are unique up to equality if the univalence axiom holds.
This result is a consequence of the fact that the bicategory of algebras is \emph{univalent}.
Recall that a bicategory is univalent if
equality between objects $X$ and $Y$ is equivalent to adjoint equivalences between $X$ and $Y$
and equality of 1-cells $f$ and $g$ is equivalent to invertible 2-cells between $f$ and $g$
\cite{bicatjournal}.
Using the methods employed by Ahrens \etal \ one can show that the bicategory of algebras
is univalent.
Since biinitial objects are unique up to adjoint equivalence, one can conclude that HITs are unique up
to equality.

\begin{proposition}
Let $\sign$ be a signature and let $H_1$ and $H_2$ are HITs for $\sign$.
Denote the underlying algebras of $H_1$ and $H_2$ by $X_1$ and $X_2$.
Then $X_1 = X_2$.
\end{proposition}


\section{Lifting the Groupoid Quotient}
\label{sec:biadj}
To construct higher inductive types, we use Proposition \ref{thm:initial_alg_sem}, which says that binitial objects satisfy the induction principle.
We use the groupoid quotient to acquire the desired algebra.
More specifically, we construct a pseudofunctor from $\algG{\sign}$ to $\algM(\sign)$,
which is the groupoid quotient on the carrier.
We do that in such a way that the obtained pseudofunctor preserves biinitiality,
so that we obtain the HIT by constructing a biinitial object in $\algG{\sign}$.

One class of pseudofunctors which preserve biinitial objects is given by left biadjoints.
More precisely, suppose that we have bicategories $\B$ and $\C$, a left biadjoint pseudofunctor $L : \pseudo(\B, \C)$, and an object $x : \B$.
Then the object $L(x)$ is biinitial if $x$ is.

Instead of directly lifting the groupoid quotient to the level of algebras,
we first show that the groupoid quotient is a left biadjoint and then we lift that biadjunction to the level of algebras.
To do so, we use the fact we defined the bicategory of algebras via displayed bicategories.
This way we can define the biadjunction on each part of the structure separately.

More specifically, we define the notion of \emph{displayed biadjunction} between two displayed bicategories over a biadjunction in the base,
and we show that each displayed biadjunction gives rise to a total biadjunction between the total bicategories.
Defining displayed biadjunctions requires defining displayed analogues of pseudofunctors, pseudotransformations, and invertible modification,
which were defined by Ahrens \etal \ \cite{bicatjournal}.
For this, we make use of \emph{displayed invertible 2-cells} \cite{bicatjournal}.

\begin{definition}
Let $\D_1$ and $\D_2$ be displayed bicategories over $\B_1$ and $\B_2$ respectively and let $F : \pseudo(\B_1, \B_2)$ be a pseudofunctor.
Then a \fat{displayed pseudofunctor} $\FF$ from $\D_1$ to $\D_2$ over $F$ consist of
\begin{itemize}
	\item For each $x : \B_1$ a map $\FF_0 : \D_1(x) \rightarrow \D_2(F(x))$;
	\item For all 1-cells $f : x \onecell y$, objects $\xx : \D_1(x)$ and $\yy : \D_1(y)$,
	and displayed 1-cells $\ff : \dmor{\xx}{\yy}{f}$, a displayed 1-cell $\FF_1(\ff) :  \dmor{\FF_0(\xx)}{\FF_0(\yy)}{F(f)}$;
	\item For all 2-cells $\tc : f \twocell g$, displayed 1-cells $\ff : \dmor{\xx}{\yy}{f}$ and $\gg : \dmor{\xx}{\yy}{g}$,
	and displayed 2-cells $\dtc : \dtwo{\ff}{\gg}{\tc}$, a displayed 2-cell $\FF_2(\dtc) : \dtwo{\FF_1(\ff)}{\FF_2(\gg)}{F(\tc)}$;
	\item For each $x : \B$ and $\xx : \D(x)$, a displayed invertible 2-cell
	$\identitor{\FF}(\xx) : \dmor{\id_1(\FF_0(\xx))}{\FF_1(\id_1(\xx))}{\identitor{F}(x)}$;
	\item For all $\ff : \dmor{\xx}{\yy}{f}$ and $\gg : \dmor{\yy}{\zz}{g}$. a displayed invertible 2-cell
	$\compositor{\FF}(\ff, \gg) : \dmor{\FF_1(\ff) \cdot \FF_1(\gg)}{\FF_1(\ff \cdot \gg)}{\compositor{F}(f, g)}$.
\end{itemize}
Here $\identitor{F}$ and $\compositor{F}$ denote the identitor and compositor of $F$.
In addition, several coherencies, which can be found in the formalization, need to be hold.
We denote the type of displayed pseudofunctors from $\D_1$ to $\D_2$ over $F$ by $\disppsfun{\D_1}{\D_2}{F}$.
\end{definition}

\begin{definition}
Let $\D_1$ and $\D_2$ be displayed bicategories over $\B_1$ and $\B_2$ respectively.
Suppose that we have displayed pseudofunctors $\FF : \disppsfun{\D_1}{\D_2}{F}$ and $\GG : \disppsfun{\D_1}{\D_2}{G}$
and a pseudotransformation $\theta : \pstrans{F}{G}$.
Then a \fat{displayed pseudotransformation} $\thetatheta$ from $\FF$ to $\GG$ over $\theta$ consists of
\begin{itemize}
	\item For all objects $x : \B$ and $\xx : \D_1(x)$ a displayed 1-cell $\thetatheta_0(x) : \dmor{\FF_0(\xx)}{\GG_0(\xx)}{\theta(x)}$;
	\item For all 1-cells $f : x \onecell y$ and $\ff : \dmor{\xx}{\yy}{f}$ a displayed invertible 2-cell
	$\thetatheta_1(f) : \dtwo{\thetatheta_0(\xx) \cdot \FF_1(\ff)}{\GG_1(\ff) \cdot \thetatheta_0(\yy)}{\theta_1(f)}$.
\end{itemize}
Here $\theta_1$ denotes the family of invertible 2-cells corresponding to the naturality squares of $\theta$.
Again several coherencies must be satisfied and the precise formulation can be found in the formalization.
The type of displayed pseudotransformatons from $\FF$ to $\GG$ over $\theta$ is denoted by $\disppstrans{\FF}{\GG}{\theta}$.
\end{definition}

\begin{definition}
Suppose that we have displayed bicategories $\D_1$ and  $\D_2$ over $\B_1$ and $\B_2$, displayed pseudofunctors $\FF$ and $\GG$ from $\D_1$ to $\D_2$ over $F$ and $G$ respectively, and displayed pseudotransformations $\thetatheta$ and $\thetatheta'$ from $\FF$ to $\GG$ over $\theta$ and $\theta'$ respectively.
In addition, let $m$ be an invertible modification from $\theta$ to $\theta'$.
Then a \fat{displayed invertible modification} $\mm$ from $\thetatheta$ to $\thetatheta'$ over $m$ consists of a displayed invertible 2-cell
$\mm_2(\xx) :  \dtwo{\thetatheta(\xx)}{\thetatheta'(\xx)}{m(x)}$ for each $x : \B_1$ and $\xx : \D_1(x)$,
In addition, a coherency must be satisfied, which can be found in the formalization.
The type of displayed invertible modifications from $\thetatheta$ to $\thetatheta'$ over $m$ is denoted by $\dispmodif{\thetatheta}{\thetatheta'}{m}$.
\end{definition}

Each of these gadgets has a total version.

\begin{problem}
\label{prob:total}
We have
\begin{enumerate}
	\item Given a displayed pseudofunctor $\FF : \disppsfun{\D_1}{\D_2}{F}$, to construct a pseudofunctor $\total{\FF} : \pseudo(\total{\D_1}, \total{\D_2})$;
	\item Given a displayed pseudotransformation $\thetatheta : \disppstrans{\FF}{\GG}{\theta}$, to construct a pseudotransformation $\total{\thetatheta} : \pstrans{\total{\FF}}{\total{\GG}}$;
	\item Given a displayed invertible modification $\mm : \xymatrix@C=1em{\thetatheta \ar@3[r]^-{m} & \thetatheta',}$ to construct an invertible modificaton $\total{\mm} : \modif{\total{\thetatheta}}{\total{\thetatheta'}}$.
\end{enumerate}
\end{problem}

\begin{construction}{prob:total}
\label{constr:biadj}
By pairing.
\end{construction}

Before we can define displayed biadjunctions, we need several operations on the displayed gadgets we introduced.

\begin{example}
We have the following
\begin{itemize}
	\item We have $\id(\D) : \disppsfun{\D}{\D}{\id(\B)}$
	where $\id(\B)$ is the identity pseudofunctor;
	\item Given $\FF : \disppsfun{\D_1}{\D_2}{F}$ and $\GG : \disppsfun{\D_2}{\D_3}{G}$,
	we have $\FF \cdot \GG : \disppsfun{\D_1}{\D_3}{F \cdot G}$
	where $F \cdot G$ is the composition of pseudofunctors;
	\item Given $\FF : \disppsfun{\D_1}{\D_2}{F}$,
	we have $\id_1(\FF) : \disppstrans{\FF}{\FF}{\id_1(F)}$
	where $\id_1(F)$ is the identity pseudotransformation on $F$;
	\item Given $\thetatheta : \disppstrans{\FF}{\GG}{\theta}$ and $\thetatheta' : \disppstrans{\GG}{\HH}{\theta'}$,
	we have $\thetatheta \vcomp \thetatheta' : \disppstrans{\FF}{\HH}{\theta \vcomp \theta'}$
	where $\theta \vcomp \theta'$ is the composition of pseudotrasformations;
	\item Given $\FF : \disppsfun{\D_1}{\D_2}{F}$, $\GG : \disppsfun{\D_2}{\D_3}{G}$, $\HH : \disppsfun{\D_2}{\D_3}{H}$, and $\thetatheta : \disppstrans{\GG}{\HH}{\theta}$,
	we have $\FF \whiskerl \thetatheta : \disppsfun{\FF \cdot \GG}{\FF \cdot \HH}{F \whiskerl \theta}$;
	\item Given $\FF : \disppsfun{\D_1}{\D_2}{F}$, $\GG : \disppsfun{\D_1}{\D_2}{G}$, $\HH : \disppsfun{\D_2}{\D_3}{H}$, and $\thetatheta : \disppstrans{\FF}{\GG}{\theta}$,
	we have $\thetatheta \whiskerr \HH : \disppsfun{\FF \cdot \HH}{\GG \cdot \HH}{\theta \whiskerr H}$;
	\item Given $\FF : \disppsfun{\D_1}{\D_2}{F}$,
	we have
	\[
	\lunitor{\FF} : \disppstrans{\id \cdot \FF}{\FF}{\lunitor{F}}, \quad
	\runitor{\FF} : \disppstrans{\FF \cdot \id}{\FF}{\runitor{\FF}},
 	\]
%% 	\item Given $\FF : \disppsfun{\D_1}{\D_2}{F}$,
%% 	we have
	\[
	\linvunitor{\FF} : \disppstrans{\FF}{\id \cdot \FF}{\linvunitor{F}}, \quad
	\rinvunitor{\FF} : \disppstrans{\FF}{\FF \cdot \id}{\rinvunitor{\FF}};
	\]
	\item Given $\FF : \disppsfun{\D_1}{\D_2}{F}$, $\GG : \disppsfun{\D_2}{\D_3}{G}$, and $\HH : \disppsfun{\D_3}{\D_4}{H}$,
	we have
	\[
	\lassoc{\FF}{\GG}{\HH} : \disppstrans{(\FF \cdot \GG) \cdot \HH}{\FF \cdot (\GG \cdot \HH)}{\lassoc{F}{G}{H}},
	\]
	\[
	\rassoc{\FF}{\GG}{\HH} : \disppstrans{\FF \cdot (\GG \cdot \HH)}{(\FF \cdot \GG) \cdot \HH}{\rassoc{F}{G}{H}}.
	\]
\end{itemize}
\end{example}

\begin{definition}
Suppose we have bicategories $\B_1$ and $\B_2$ and a biadjunction $L \dashv R$ from $\B_1$ to $\B_2$.
We write $\eta$ and $\epsilon$ for the unit and counit of $L \dashv R$ respectively,
and we write $\tau_1$ and $\tau_r$ for the left and right triangle respectively.
Suppose, that we also have displayed bicategories $\D_1$ and $\D_2$ over $\B_1$ and $\B_2$ respectively
and a displayed pseudofunctor $\LL : \disppsfun{\D_1}{\D_2}{L}$.
Then we say $\LL$ is a \fat{displayed left biadjoint pseudofunctor} if we have
\begin{itemize}
	\item A displayed pseudofunctor
	$
	\RR : \disppsfun{\D_2}{\D_1}{R}
	$;
	\item Displayed pseudotransformations
	\[
	\etaeta : \disppstrans{\id}{\LL \cdot \RR}{\eta}, \quad
	\epseps : \disppstrans{\RR \cdot \LL}{\id}{\epsilon};
	\]
	\item Displayed invertible modifications
	\[
	\tautaul : \dispmodif{\rho^{-1} \vcomp \LL \whiskerl \etaeta \vcomp \alpha \vcomp \epseps \whiskerr \LL \vcomp \lambda}{\id_1(\LL),}{\tau_1}
	\]
	\[
	\tautaur : \dispmodif{\lambda^{-1} \vcomp \etaeta \whiskerr \RR \vcomp \alpha^{-1} \vcomp \RR \whiskerl \epseps \vcomp \rho}{\id_1(\RR).}{\tau_2}
	\]
\end{itemize}
\end{definition}

From Construction \ref{constr:biadj}, we get

\begin{proposition}
\label{prop:total_biadj}
Given a displayed left biadjoint pseudofunctor $\LL$,
then $\total{\LL}$ is a left biadjoint pseudofunctor.
\end{proposition}

\begin{figure}
\[
\xymatrix
{
	\algebra{\sign} \ar@/^/[rr]_-{\top}^-{\algpgrpd} \ar[d] & & \algG{\sign} \ar@/^/[ll]^-{\alggquot} \ar[d] \\
	\pathalg{\sign} \ar@/^/[rr]_-{\top}^-{\pathpgrpd} \ar[d] & & \pathalgG{\sign} \ar@/^/[ll]^-{\pathgquot} \ar[d] \\
	\prealg{\sign} \ar@/^/[rr]_-{\top}^-{\prepgrpd} \ar[d] & & \prealgG{\sign} \ar@/^/[ll]^-{\pregquot} \ar[d]\\
	\onetypes \ar@/^/[rr]_-{\top}^-{\pgrpd} & & \grpd \ar@/^/[ll]^-{\gquot}
}
\]
\caption{The biadjunction}
\label{fig:biadj}
\end{figure}

Now let us use the introduced notions to construct the biadjunction on the level of algebras.
Our approach is summarized in Figure \ref{fig:biadj}.
We start by showing that the groupoid quotient gives rise to a biadjunction.

\begin{problem}
\label{prob:gquit_biadj}
To construct $\gquot \dashv \pgrpd$ where $\gquot : \pseudo(\grpd, \onetypes)$.
\end{problem}

\begin{construction}{prob:gquit_biadj}
\label{constr:gquit_biadj}
We only show how the involved pseudofunctors are defined.
The pseudofunctor $\gquot$ is the groupoid quotient
while $\pgrpd$ sends a 1-type $X$ to the groupoid whose objects are points of $X$ and morphisms from $x$ to $y$ are paths $x = y$.
\end{construction}

Next we lift this biadjunction to the level of algebras using the displayed machinery introduced in this section.

\begin{problem}
\label{prob:alg_biadj}
Given a signature $\sign$, to construct a biadjunction $\alggquot \dashv \algpgrpd$ where $\alggquot : \pseudo(\algG{\sign}, \algebra{\sign})$.
\end{problem}

\begin{construction}{prob:alg_biadj}
\label{constr:alg_biadj}
We only give a very brief outline of the construction.

We start by constructing a displayed biadjunction
from $\DFAlg(\semPG{\pointconstr[\sign]})$
to $\DFAlg(\semP{\pointconstr[\sign]})$
over the biadjunction from
Construction \ref{constr:gquit_biadj}.
To do so, we first need to lift the pseudofunctors, and for that, we generalize the approach of Hermida and Jacobs
to the bicategorical setting \cite[Theorem 2.14]{hermida1998structural}.
This requires us to construct two pseudotransformations.
\[
p_1 : \pstrans{\semP{P} \cdot \gquot}{\gquot \cdot \semPG{P}},
\]
\[
p_2 : \pstrans{\semPG{P} \cdot \pgrpd}{\pgrpd \cdot \semP{P}}.
\]
We denote the total biadjunction of the resulting displayed biadjunction by $\pregquot \dashv \prepgrpd$.

Next we lift the biadjunction to the level of path algebras
and for that, we construct a displayed biadjunction between 
$\DCell(\semEG{\pathleft[\sign](i)},\semEG{\pathright[\sign](i)})$
and $\DCell(\semE{\pathleft[\sign](i)},\semE{\pathright[\sign](i)}$
for all $j : \pathlabel$.
Denote the resulting total biadjunction by $\pathgquot \dashv \pathpgrpd$.

To finish the proof, we need to construct one more displayed biadjunction.
For that, we only need to show that if $G : \pathalgG{\sign}$ is an algebra, then $\pathgquot(G)$ also is an algebra,
and if $X : \pathalg{\sign}$ is an algebra, then so is $\pathpgrpd(X)$.
\end{construction}

The next proposition concludes this section.

\begin{proposition}
\label{prop:biinitial_in_grpd}
If $G$ is an biinitial object in $\algG{\sign}$,
then $\alggquot(G)$ is a biinitial object in $\algebra{\sign}$.
\end{proposition}


\section{HIT Existence}
\label{sec:existence}
From Theorem \ref{thm:initial_alg_sem} we know that initiality implies the induction principle.
Hence, it suffices to construct a biinitial object in the bicategory of algebras in 1-types.
By Proposition \ref{prop:biinitial_in_grpd}, it suffices to construct a biinitial object in $\algG{\sign}$.
To do so, we adapt the semantics by Dybjer and Moeneclaey to our setting \cite{DBLP:journals/entcs/DybjerM18}.

\begin{problem}
\label{prob:initial_grpd_alg}
Given a signature $\sign$, to construct a biinitial object  $\constfont{G}$ in $\algG{\sign}$.
\end{problem}

\begin{construction}{prob:initial_grpd_alg}
\label{constr:initial_grpd_alg}
We only discuss how the carrier $G$ of $\constfont{G}$ is defined.
\begin{itemize}
	\item Note  that each polynomial $P$ gives rise to a container $\hat{P}$.
	Note that each container induces a W-type \cite{abbott2003categories},
	and we define the type of objects of $G$ to be the W-type induced by $\hat{\pointconstr}$.
	Denote this type by $\initob$ and let $\constfont{c}^{\initob} : \pointconstr(\initob) \to \initob$ its algebra map.
	\item The morphisms of $G$ are constructed as a set quotient.
	First, we define an inductive type $\initmorgen{P}{x}{y}$, for each polynomial $P :\poly$ and elements $x, y : P (\initob)$. Its constructors are given in Figure \ref{fig:initmor}.
        When $P$ is $\idP$ and $x,y : \initob$, we write $\initmor{x}{y}$ instead of $\initmorgen{\idP}{x}{y}$. Afterwards, for each $x, y : \initob$ and $f, g : \initmor{x}{y}$, we define a type $\initeq{f}{g}$.
	Both of these are defined as an inductive type and for the constructors, we refer the reader to the formalization.
	Basically, the constructors for these types are chosen in such a way that the groupoid being defined here, has the desired structure.
	This means we add constructors witnessing the path constructors, identity, composition, and all other laws.
	We use the same idea to define $\initeq{f}{g}$.
	
	Note that the input of the quotient is an equivalence relation, which is valued in propositions.
	For this reason, we define $\initeqprop{f}{g}$ to be the propositional truncation of $\initeq{f}{g}$.
	All in all, we define the morphisms from $x$ to $y$ to be the set quotient of $\initmor{x}{y}$ by $\approx_p$.
	\qedhere	
\end{itemize}
\end{construction}

\begin{figure*}[t]
\begin{center}
\begin{bprooftree}
\AxiomC{$P : \poly$}
\AxiomC{$x : P(\initob)$}
\BinaryInfC{$\initmorid{x} : \initmorgen{P}{x}{x}$}
\end{bprooftree}
\begin{bprooftree}
\AxiomC{$P : \poly$}
\AxiomC{$x,y : P(\initob)$}
\AxiomC{$f : \initmorgen{P}{x}{y}$}
\TrinaryInfC{$\initmorinv{f} : \initmorgen{P}{y}{x}$}
\end{bprooftree}
\end{center}

\vspace{5pt}

\begin{center}
\begin{bprooftree}
\AxiomC{$P : \poly$}
\AxiomC{$x,y,z : P(\initob)$}
\AxiomC{$f : \initmorgen{P}{x}{y}$}
\AxiomC{$g : \initmorgen{P}{y}{z}$}
\QuaternaryInfC{$\initmorcomp{f}{g} : \initmorgen{P}{x}{z}$}
\end{bprooftree}
\end{center}

\vspace{5pt}

\begin{center}
\begin{bprooftree}
\AxiomC{$P,Q:\poly$}
\AxiomC{$x,y : P(\initob)$}
\AxiomC{$f : \initmorgen{P}{x}{y}$}
\TrinaryInfC{$\initmorinl{f} : \initmorgen{P+Q}{\inl(x)}{\inl(y)}$}
\end{bprooftree}
\begin{bprooftree}
\AxiomC{$P,Q:\poly$}
\AxiomC{$x,y : Q(\initob)$}
\AxiomC{$f : \initmorgen{Q}{x}{y}$}
\TrinaryInfC{$\initmorinr{f} : \initmorgen{P+Q}{\inr(x)}{\inr(y)}$}
\end{bprooftree}
\end{center}

\vspace{5pt}

\begin{center}
\begin{bprooftree}
\AxiomC{$P,Q : \poly$}
\AxiomC{$x,y : P(\initob)$}
\AxiomC{$w,z : Q(\initob)$}
\AxiomC{$f : \initmorgen{P}{x}{y}$}
\AxiomC{$g : \initmorgen{Q}{w}{z}$}
\QuinaryInfC{$\initmorpair{f}{g} : \initmorgen{P\times Q}{\pair{x}{w}}{\pair{y}{z}}$}
\end{bprooftree}
\end{center}

\vspace{5pt}

\begin{center}
\begin{bprooftree}
\AxiomC{$j : \pathlabel$}
\AxiomC{$x : \patharg_j(\initob)$}
\BinaryInfC{$\initmorpath{j,x} : \initmorgen{\idP}{\semE{\pathleft_j}(x)}{\semE{\pathright_j}(x)}$}
\end{bprooftree}
\begin{bprooftree}
\AxiomC{$x,y : \pointconstr(\initob)$}
\AxiomC{$f : \initmorgen{\pointconstr}{x}{y}$}
\BinaryInfC{$\initmorap{f} : \initmorgen{\idP}{\constfont{c}^{\initob}(x)}{\constfont{c}^{\initob}(y)}$}
\end{bprooftree}
\end{center}
\caption{Rules for the type $\initmorgen{P}{x}{y}$.}
\label{fig:initmor}
\end{figure*}

\begin{figure*}[t]
\begin{center}
\begin{bprooftree}
\AxiomC{$f : \initmorgen{P}{x}{y}$}
\UnaryInfC{$\initeqgen{P}{f}{f}$}
\end{bprooftree}
\begin{bprooftree}
\AxiomC{$f,g : \initmorgen{P}{x}{y}$}
\AxiomC{$\initeqgen{P}{f}{g}$}
\BinaryInfC{$\initeqgen{P}{g}{f}$}
\end{bprooftree}
\begin{bprooftree}
\AxiomC{$f,g,h : \initmorgen{P}{x}{y}$}
\AxiomC{$\initeqgen{P}{f}{g}$}
\AxiomC{$\initeqgen{P}{g}{h}$}
\TrinaryInfC{$\initeqgen{P}{f}{h}$}
\end{bprooftree}
\end{center}

\vspace{5pt}

\begin{center}
\begin{bprooftree}
\AxiomC{$\initeqgen{\idP}{\initmorap{\initmorid{x}}}{\initmorid{\constfont{c}^{\initob}(x)}}$}
\end{bprooftree}
\begin{bprooftree}
\AxiomC{$f : \initmorgen{\pointconstr}{x}{y}$}
\AxiomC{$g : \initmorgen{\pointconstr}{y}{z}$}
\BinaryInfC{$\initeqgen{\idP}{\initmorap{\initmorcomp{f}{g}}}{\initmorcomp{\initmorap{f}}{\initmorap{g}}}$}
\end{bprooftree}
\end{center}

\vspace{5pt}

\begin{center}
\begin{bprooftree}
\AxiomC{$\initeqgen{P+Q}{\initmorinl{\initmorid{x}}}{\initmorid{\inl{x}}}$}
\end{bprooftree}
\begin{bprooftree}
\AxiomC{$\initeqgen{P+Q}{\initmorinr{\initmorid{x}}}{\initmorid{\inr{x}}}$}
\end{bprooftree}
\end{center}

\vspace{5pt}

\begin{center}
\begin{bprooftree}
\AxiomC{$\initeqgen{P}{f}{g}$}
\UnaryInfC{$\initeqgen{P+Q}{\initmorinl{f}}{\initmorinl{g}}$}
\end{bprooftree}
\begin{bprooftree}
\AxiomC{$f : \initmorgen{P}{x}{y}$}
\AxiomC{$g : \initmorgen{P}{y}{z}$}
\BinaryInfC{$\initeqgen{P + Q}{\initmorinl{\initmorcomp{f}{g}}}{\initmorcomp{\initmorinl{f}}{\initmorinl{g}}}$}
\end{bprooftree}
\end{center}

\vspace{5pt}

\begin{center}
\begin{bprooftree}
\AxiomC{$\initeqgen{Q}{f}{g}$}
\UnaryInfC{$\initeqgen{P+Q}{\initmorinr{f}}{\initmorinr{g}}$}
\end{bprooftree}
\begin{bprooftree}
\AxiomC{$f : \initmorgen{Q}{x}{y}$}
\AxiomC{$g : \initmorgen{Q}{y}{z}$}
\BinaryInfC{$\initeqgen{P + Q}{\initmorinr{\initmorcomp{f}{g}}}{\initmorcomp{\initmorinr{f}}{\initmorinr{g}}}$}
\end{bprooftree}
\end{center}


\caption{Rules for the type $\initeqgen{P}{f}{g}$.}
\label{fig:initmor}
\end{figure*}


\begin{problem}
\label{prob:hit_exist}
Each signature has a HIT.
\end{problem}

\begin{construction}{prob:hit_exist}
\label{constr:hit_exist}
By Propositions \ref{thm:initial_alg_sem} and \ref{prop:biinitial_in_grpd}, it suffices to find a biinitial object in $\algG{\sign}$.
The desired object is given in Construction \ref{constr:initial_grpd_alg}.
\end{construction}


%\section{Calculating Fundamental Groups}

\section{Conclusion and Further Work}
\label{sec:conclusion}
We showed how to construct finitary 1-truncated higher inductive types using the propositional truncation, quotient, and the groupoid quotient.
This reduces the existence of a general class of HITs to simpler ones.
We needed the types to be 1-truncated, so that we could use the framework of bicategory theory,
and the HITs we studied had to be finitary to guarantee that the groupoid quotient
commutes with the involved operations \cite{DBLP:journals/mscs/ChapmanUV19}.
On the way, we also proved that HITs are unique and we studied universal algebra with our signatures.
We showed that the bicategory of algebras has finite limits and we proved the first isomorphism theorem for these algebras.
Lastly, we used the way we constructed HITs, to calculate fundamental groups.

There are numerous ways to improve on these results.
First of all, we only constructed finite limits of algebras while it should also be possible to construct finite colimits of algebras.
The scheme studied in this paper is not flexible enough to support these colimits since we do not have a path endpoint that represents the action of a polynomial on the point constructor.
Hence, if we want to internally construct these colimits, then we need to define a more permissive signature for higher inductive types.

Secondly, it should be possible to modify our approach to obtain HITs in directed type theory (DTT) \cite{north2019towards}.
In the model of DTT provided by North, types are interpreted as categories and higher inductive types in DTT could be interpreted as initial algebras.
We constructed such algebras in the bicategory of groupoids and in a similar way, one should be able to construct the desired algebras in the bicategory of categories.

Lastly, our construction only considers a rather simple scheme of HITs.
In particular, we restrict ourselves to the 1-truncated case. % \cite{KaposiK18}.
Since untruncated types correspond to $\infty$-groupoids,
generalizing the methods used in this paper to the untruncated case,
requires formalizing notions from $\infty$-category theory in type theory
\cite{DBLP:conf/csl/AltenkirchR12,DBLP:journals/pacmpl/CapriottiK18,FinsterM17}.
This also requires finding an $\infty$-dimensional generalization of the groupoid quotient.
An alternative approach to deal with untruncated HITs, pointed out by Ali Caglayan,
would be using wild categories \cite{DBLP:conf/tlca/HirschowitzHT15,KrausRaumer}.
We would also like to extend our scheme to incorporate both indexed HITs and higher inductive-inductive types \cite{CavalloH19,KaposiK18}.



\begin{acks}
The author thanks Herman Geuvers, Dan Frumin, Niccol\`o Veltri, Benedikt Ahrens, and Ali Caglayan for helpful comments and discussions.
The author also thanks the anonymous reviewers for their helpful comments and suggestions.
\end{acks}


\bibliography{literature}

\end{document}
