In this section we present some additional examples, complementing the
ones introduced in Section \ref{sec:signs}. Remember that our higher
inductive types are all 1-truncated, and we always omit the 1-truncation
constructor from their syntax.

\subsection{Coequalizer}
\label{sec:coequalizer}

The coequalizer is a particular kind of finite colimit in a
category. It admits a bicategorical generalization, which is typically
called coinserter. Here we refrain from using the latter terminology
and refer to this colimit in bicategories again as coequalizer. This
choice of terminology is motivated by our interest for this
construction in the bicategory of 1-types, in which the coinserter
captures precisely the notion of homotopy coequalizer \cite[Chapter 6]{hottbook}.

\begin{definition}
Let $\B$ be a bicategory. Let $A$ and $B$ be objects of $\B$ and let
$f,g : A \onecell B$. The \emph{coequalizer} of $f$ and $g$ is an
object $Q$ together with a 1-cell $q : B \onecell Q$ and a 2-cell
$\theta : f \cdot q\twocell g \cdot q$.

The triple $(Q,q,\theta)$ must satisfy the following universal
property. Assume given:
\begin{itemize}
\item an object $Q'$,
\item a 1-cell $q' : B \onecell Q'$, and
\item a 2-cell $\theta' : f \cdot q' \twocell g \cdot q'$.
\end{itemize}
Then there exists a 1-cell $h : Q \onecell Q'$ together with a 2-cell
$\phi : q \cdot h \twocell q'$ with a path $\theta \whiskerr h \vcomp
g \whiskerl \phi = \lassoc{f}{q}{h} \vcomp f \whiskerl \phi \vcomp
\theta'$. The pair $(h,\phi)$ is ``unique'' in the sense that,
%% \footnote{Notice that this path would type-check also without the
%%   left associator $\lassoc{}{}{}$, since function composition is
%%   stricly associative. But the presence of the left associator is
%%   necessary if we consider algebras for the coequalizer in
%%   bicategories more general than the one of 1-types and groupoids.}.
given another 1-cell $h' : Q \onecell Q'$ and another 2-cell $\phi' :
q \cdot h' \twocell q'$ with a path $\theta \whiskerr h' \vcomp g
\whiskerl \phi' = \lassoc{f}{q}{h'} \vcomp f \whiskerl \phi' \vcomp
\theta'$, then there exists a unique invertible 2-cell $\tau : h
\twocell h'$ such that $q \whiskerl \tau \vcomp \phi' = \phi$.
\end{definition}

This section is dedicated to the construction of the coequalizer in
the bicategory of 1-types.  Given two 1-types $A$ and $B$ and two
functions $f , g: A \to B$, the coequalizer of $f$ and $g$ is
given by the following HIT:
\begin{lstlisting}[mathescape=true]
Inductive $\coequalizer f g$ :=
| $\coequalizerbase$ : $B \rightarrow \coequalizer f g$
| $\coequalizerglue$ : $\depprod{(x : A)}{\coequalizerbase(f(x)) = \coequalizerbase(g(x))}$
\end{lstlisting}

Here are all the ingredients needed to specify the signature $\coequalizer f g$ for the coequalizer:
%is constructed as follows:
\begin{itemize}
\item $\pointconstr[\coequalizer f g] = \constantP{B}$;
\item $\pathlabel[\coequalizer f g] = \unit$, and for its unique inhabitant take $\patharg[\coequalizer f g] = \constantP{A}$ and endpoints
  \[
  \pathleft[\coequalizer f g] = \comp{\fmap(f)}{\constr}, \quad
  \pathright[\coequalizer f g] = \comp{\fmap(g)}{\constr}; 
  \]
\item $\homotlabel[\coequalizer f g]$ is the empty type.
\end{itemize}

%% The coequalizer satisfies the following universal property. Assume
%% given a 1-type $C$, a function $h : B \rightarrow C$ and a homotopy
%% $p$ between $f \cdot h$ and $g \cdot h$. Then there exists a function
%% $h' : \coequalizer f g \rightarrow C$ together with a homotopy $p'$
%% between $\coequalizerbase \cdot h'$ and $h$, and a path
%% $\coequalizerglue \whiskerr h' \vcomp g \whiskerl p' = \lassoc{}{}{}
%% \vcomp f \whiskerl p' \vcomp p$\footnote{Notice that this path would
%%   type-check also without the left associator $\lassoc{}{}{}$, since
%%   function composition is stricly associative. But the presence of the
%%   left associator is necessary if we consider algebras for the
%%   coequalizer in bicategories more general than the one of 1-types and
%%   groupoids.}.  Given another function $h'' : \coequalizer f g
%% \rightarrow C$ with homotopy $p''$ between $\coequalizerbase \cdot
%% h''$ and $h$ such that $\coequalizerglue \whiskerr h'' \vcomp g
%% \whiskerl p'' = \lassoc{}{}{} \vcomp f \whiskerl p'' \vcomp p$, then
%% there exists a unique homotopy $\tau$ between $h'$ and $h''$ such that
%% $\coequalizerbase \whiskerl \tau \vcomp p'' = p'$.



\subsection{Coequifier}
\label{sec:coequifier}

The coequifier is a finite colimit in a bicategory, corresponding to a
higher version of the coequalizer.

\begin{definition}
Let $\B$ be a bicategory. Let $A$ and $B$ be objects of $\B$, let $f,
g : A \onecell B$ and $\beta,\gamma : f \twocell
g$. The \emph{coequifier} of $\beta$ and $\gamma$ is an object $Q$
together with a 1-cell $q : B \onecell Q$ and a path
$\beta \whiskerr q = \gamma \whiskerr q$.

The pair $(Q,q)$ must satisfy the following universal
property. Assume given:
\begin{itemize}
\item an object $Q'$,
\item a 1-cell $q' : B \onecell Q'$, and
\item a path $\beta \whiskerr q' = \gamma \whiskerr q'$.
\end{itemize}
Then there exists a 1-cell $h : Q \onecell Q'$ together with a 2-cell
$\phi : q \cdot h \twocell q'$. The pair $(h,\phi)$ is ``unique'' in the sense that,
given another 1-cell $h' : Q \onecell Q'$ and another 2-cell $\phi' :
q \cdot h' \twocell q'$, then there exists a unique invertible 2-cell $\tau : h
\twocell h'$ such that $q \whiskerl \tau \vcomp \phi' = \phi$.
\end{definition}


In the bicategory of 1-types, the coequifier is defined as a HIT.  In
the definition below, $A$ and $B$ are 1-types, $f , g: A \to B$ are
functions and $\beta$ and $\gamma$ are homotopies between $f$ and $g$.
\begin{lstlisting}[mathescape=true]
Inductive $\coequifier \beta \gamma$ :=
| $\coequifierbase$ : $B \rightarrow \coequifier f g$
| $\coequifierglue$ : $\depprod{(x : A)}{\ap {\coequifierbase}{(\beta(x))} = \ap {\coequifierbase}{(\gamma(x))}}$
\end{lstlisting}

The signature $\coequifier \beta \gamma$ for the coequifier of $\beta$
and $\gamma$ is given as follows:
\begin{itemize}
\item $\pointconstr[\coequifier f g] = \constantP{B}$;
\item $\pathlabel[\coequifier f g]$ is the empty type; 
\item $\homotlabel[\coequifier f g] = \unit$, and take 
  $\homotpointarg[\coequifier f g] = \constantP{A}$.
  The homotopy constructor $\coequifierglue$ does not have path arguments, so we take  
  $\homotpathtarg[\coequifier f g] = \constantP{\unit}$ and
  $\pathargleft[\coequifier f g] = \pathargright[\coequifier f g] = \Ce(\unitt)$.
  The endpoints $\homotpathleft[\coequifier f g]$ and $\homotpathright[\coequifier f g]$ are:
  \[
  \homotpathleft[\coequifier f g] = \comp{\fmap(f)}{\constr}, \quad
  \homotpathright[\coequifier f g] = \comp{\fmap(g)}{\constr},
  \]
  while the left and right homotopy endpoints are:
  \[
  \hap{\constr}{(\idtoH(\ap{\fmap}{(\funextsec(\beta))}))}, \quad
  \hap{\constr}{(\idtoH(\ap{\fmap}{(\funextsec(\gamma))}))}.
  \]
\end{itemize}
In the construction of the homotopy endopoint we used the function
$\idtoH$ introduced in Section~\ref{sec:signatures}, which embeds
paths between endpoints into homotopy endpoints. Notice also the
difference between $\constructor{ap}$ and $\constfont{ap}$: the first
is an homotopy endpoint constructor, the second indicates the
application of a function to a path.


\subsection{Group Quotient}
\label{sec:group_quotient}

Now we introduce a particular instance of the groupoid quotient, that
we call \emph{group quotient}. Start with a group $G$, with $\id$ as
unit and $\cdot$ as multiplication. Define a groupoid $\widehat{G}$
with only one object and with $G$ as the only homset. The group
quotient of $G$ is the groupoid quotient of $\widehat{G}$,
corresponding to the following HIT:
\begin{lstlisting}[mathescape=true]
Inductive $\groupquot G$ :=
| $\groupquotbase$ : $\groupquot G$
| $\groupquotloop$ : $G \rightarrow \groupquotbase = \groupquotbase$
| $\groupquotloope$ : $\groupquotloop(\id)= \refl \groupquotbase$
| $\groupquotloopm$ : $\depprod{(x,y : G)}{\groupquotloop(x \cdot y)= \concat{\groupquotloop(x)}{\groupquotloop(y)}}$
\end{lstlisting}


The signature $\groupquot G$ for the group quotient is defined as follows:
\begin{itemize}
\item $\pointconstr = \constantP{\unit}$;
\item $\pathlabel = \unit$, and for its unique inhabitant take $\patharg = \constantP{G}$ and both endpoints $\pathleft$ and $\pathright$ equal to $\comp{\Ce(\unitt)}{\constr}$.
\item $\homotlabel = \bool$, where $\bool$ is the type of
booleans with inhabitants $\booltrue$ and $\boolfalse$. This means
that there are two homotopy constructors: $\groupquotloope$, with
associated label $\booltrue$, and $\groupquotloopm$, with associated
label $\boolfalse$.
\begin{itemize}
\item The constructor $\groupquotloope$ does not have
point arguments, so we take $\homotpointarg_{\booltrue}
= \constantP{\unit}$. It also does not have path arguments, therefore
$\homotpathtarg_{\booltrue} = \constantP{\unit}$ and
$\pathargleft_{\booltrue} = \pathargright_{\booltrue} = \Ce(\unitt)$.
The endpoints $\homotpathleft_{\booltrue}$ and
$\homotpathright_{\booltrue}$ are both equal to
$\comp{\Ce(\unitt)}{\constr}$. The left homotopy endpoint is
\[
\hconcat{\hap{\constr}{\hinv{(\hcompconst{\Ce(\id)})}}}{
\hconcat{\hinv{\hassocN}}{
\hconcat{\hconstr{}{\Ce(\id)}}{
\hconcat{\hassocN}{\hap{\constr}{(\hcompconst{\Ce(\id)})}}
}}},
\]
while the right homotopy endpoint is $\hrefl{\constr}$.
\item The constructor $\groupquotloopm$ has two point
arguments of type $G$, so we take $\homotpointarg_{\boolfalse}
= \constantP{G \times G}$. It does not have path arguments, therefore
$\homotpathtarg_{\boolfalse} = \constantP{\unit}$ and
$\pathargleft_{\boolfalse} = \pathargright_{\boolfalse}
= \Ce(\unitt)$.  The endpoints $\homotpathleft_{\boolfalse}$ and
$\homotpathright_{\boolfalse}$ are both equal to
$\comp{\Ce(\unitt)}{\constr}$. The left homotopy endpoint is
%%\begin{align*}
%%\hconcat{\hap{\constr}{\hinv{(\hcompconst{\fmap(\Lam {x,y}. x \cdot y)})}}}{
%%& \hinv{\hassocN}} \\
%%\hconcat{}{& \hconstr{}{\fmap(\Lam {x,y}. x \cdot y)}} \\
%%\hconcat{}{& \hassocN} \\
%%\hconcat{}{& \hap{\constr}{(\hcompconst{\fmap(\Lam {x,y}. x \cdot y)})}},
%%\end{align*}
%%while the right endpoint is
%%\begin{align*}
%%\hconcat{\hap{\constr}{\hinv{(\hcompconst{\fmap(\projl)})}}}{
%%& \hinv{\hassocN}} \\
%%\hconcat{}{& \hconstr{}{\fmap(\projl)}} \\
%%\hconcat{}{& \hassocN} \\
%%\hconcat{}{& \hap{\constr}{(\hcompconst{\fmap(\projl)})}} \\
%%\hconcat{}{& \hap{\constr}{\hinv{(\hcompconst{\fmap(\projr)})}}} \\
%%\hconcat{}{& \hinv{\hassocN}} \\
%%\hconcat{}{& \hconstr{}{\fmap(\projr)}} \\
%%\hconcat{}{& \hassocN} \\
%%\hconcat{}{& \hap{\constr}{(\hcompconst{\fmap(\projr)})}}
%%\end{align*}
\begin{align*}
& \hconcat{\hap{\constr}{\hinv{(\hcompconst{\fmap(\Lam {x,y}. x \cdot y)})}}}{\hinv{\hassocN}} \\
& \quad \hconcat{}{\hconstr{}{\fmap(\Lam {x,y}. x \cdot y)}} \\
& \quad \hconcat{}{\hconcat{\hassocN}{\hap{\constr}{(\hcompconst{\fmap(\Lam {x,y}. x \cdot y)})}}},
\end{align*}
while the right endpoint is
\begin{align*}
& \hconcat{\hap{\constr}{\hinv{(\hcompconst{\fmap(\projl)})}}}{
\hinv{\hassocN}} \\
& \quad \hconcat{}{\hconstr{}{\fmap(\projl)}} \\
& \quad \hconcat{}{\hconcat{\hassocN}{\hap{\constr}{(\hcompconst{\fmap(\projl)})}}} \\
& \quad \hconcat{}{\hconcat{\hap{\constr}{\hinv{(\hcompconst{\fmap(\projr)})}}}{\hinv{\hassocN}}} \\
& \quad \hconcat{}{\hconstr{}{\fmap(\projr)}} \\
& \quad \hconcat{}{\hconcat{\hassocN}}{\hap{\constr}{(\hcompconst{\fmap(\projr)})}}
\end{align*}
\end{itemize}
\end{itemize}

The signature for the groupoid quotient is obtainable as a slight
generalization of the signature for the group quotient. We do not show
the more general construction here, since this is not conceptually
more enlightening than the (already quite complicated) signature for
the group quotient.

\subsection{Monoidal Object}
\label{sec:monoidal_object}


Another important algebraic example constructed as a HIT is what we
refer to as the \emph{monoidal object}:
\begin{lstlisting}[mathescape=true]
Inductive $\monobj$ :=
| $\monobjunit$ : $\monobj$
| $\monobjtensor$ : $\monobj \rightarrow \monobj \rightarrow \monobj$
| $\monobjlambda$ : $\depprod{(x : \monobj)}{\monobjtensor(\monobjunit,x) = x}$
| $\monobjrho$ : $\depprod{(x : \monobj)}{\monobjtensor(x,\monobjunit) = x}$
| $\monobjalpha$ : $\depprod{(x ,y,z: \monobj)}{\monobjtensor(x,(\monobjtensor(y,z)))=\monobjtensor(\monobjtensor(x,y),z)}$
| $\monobjtr$ : $\depprod{(x,y : \monobj)}{\ap{(\Lam z. \monobjtensor(x,z))}{(\monobjlambda(y))}} = \concat{\monobjalpha(x,\monobjunit,y)}{\ap{(\Lam z. \monobjtensor(z,y))}{(\monobjrho(x))}}$
| $\monobjpent$ : $\depprod{(w,x,y,z : \monobj)}{}$
    $\concat{\monobjalpha(w,x,\monobjtensor(y,z))}{\monobjalpha(\monobjtensor(w,x),y,z)}=\concat{\ap{(\Lam v. \monobjtensor(w,v))}{(\monobjalpha(x,y,z))}}{\concat{\monobjalpha(w,\monobjtensor(x,y),z)}{\ap{(\Lam v. \monobjtensor(v,z))}{(\monobjalpha(w,x,y))}}}$
\end{lstlisting}
We do not show the signature associated to this HIT here. We redirect
the interested reader to our formalization for the complete
definition.

In the constructors of $\monobj$ we can recognize the data of a
monoidal category. The point constructors $\monobjunit$ and
$\monobjtensor$ correspond to unit object and tensor. The path
constructors $\monobjlambda$, $\monobjrho$ and $\monobjalpha$ are left
unitor, right unitor and associator, while the homotopy constructors
$\monobjtr$ and $\monobjpent$ are the two coherence laws of
monoidal categories. And in fact, algebras in groupoids of the
monoidal object signature are precisely
\emph{monoidal groupoids}, the groupoid variant of monoidal
categories, and $\monobj$ is a presentation of the initial
monoidal groupoid.

%%The importance of this example stems from the fact that algebras in
%%groupoids of the monoidal object signaturecorrespond precisely
%%to
%%\emph{monoidal groupoids}, the groupoid variant of monoidal
%%categories. $\monobj$ is therefore a presentation of the initial
%%monoidal groupoid.

We also formalized the \emph{coherent group} \cite{baez2004groups} as
a HIT, that we name $\cohgroup$.  Its definition includes all the
constructors of the monoidal object, plus a new point constructor
$\cohgroupinv$ : $\cohgroup \rightarrow \cohgroup$, two new path
constructors
\[
\cohgrouplinv : \depprod{(x : \cohgroup)}{\cohgrouptensor(\cohgroupinv(x),x) = \cohgroupunit},
\quad\quad
\cohgrouprinv : \depprod{(x : \cohgroup)}{\cohgroupunit = \cohgrouptensor(x,\cohgroupinv(x))},
\]
and two new homotopy constructors.

%% \subsection{Coherent Group}
%% \label{sec:coeherent_group}
%% 
%% The coherent group is defined by the following HIT:
%% \begin{lstlisting}[mathescape=true]
%% Inductive $\cohgroup$ :=
%% | $\cohgroupunit$ : $\cohgroup$
%% | $\cohgrouptensor$ : $\cohgroup \rightarrow \cohgroup \rightarrow \cohgroup$
%% | $\cohgroupinv$ : $\cohgroup \rightarrow \cohgroup$
%% | $\cohgrouplambda$ : $\depprod{(x : \cohgroup)}{\cohgrouptensor(\cohgroupunit,x) = x}$
%% | $\cohgrouprho$ : $\depprod{(x : \cohgroup)}{\cohgrouptensor(x,\cohgroupunit) = x}$
%% | $\cohgroupalpha$ : $\depprod{(x ,y,z: \cohgroup)}{\cohgrouptensor(x,(\cohgrouptensor(y,z)))=\cohgrouptensor(\cohgrouptensor(x,y),z)}$
%% | $\cohgrouplinv$ : $\depprod{(x : \cohgroup)}{\cohgrouptensor(\cohgroupinv(x),x) = \cohgroupunit}$
%% | $\cohgrouprinv$ : $\depprod{(x : \cohgroup)}{\cohgroupunit = \cohgrouptensor(x,\cohgroupinv(x))}$
%% | $\cohgrouptr$ : $\depprod{(x,y : \cohgroup)}{\ap{(\Lam z. \cohgrouptensor(x,z))}{(\cohgrouplambda(y))}} = \concat{\cohgroupalpha(x,\cohgroupunit,y)}{\ap{(\Lam z. \cohgrouptensor(z,y))}{(\cohgrouprho(x))}}$
%% | $\cohgrouppent$ : $\depprod{(w,x,y,z : \cohgroup)}{}$
%%     $\concat{\cohgroupalpha(w,x,\cohgrouptensor(y,z))}{\cohgroupalpha(\cohgrouptensor(w,x),y,z)}=\concat{\ap{(\Lam v. \cohgrouptensor(w,v))}{(\cohgroupalpha(x,y,z))}}{\concat{\cohgroupalpha(w,\cohgrouptensor(x,y),z)}{\ap{(\Lam v. \cohgrouptensor(v,z))}{(\cohgroupalpha(w,x,y))}}}$
%% | $\cohgroupinvadjtrl$ : $\depprod{(x : \cohgroup)}{\concat{\ap{(\Lam z. \cohgrouptensor(z,x))}{(\cohgrouprinv(x))}}{\concat{\inverse{(\cohgroupalpha(x,\cohgroupinv(x),x))}}{\ap{(\Lam z. \cohgrouptensor(x,z))}{(\cohgrouplinv(x))}}}=\concat{\cohgrouplambda(x)}{\inverse{(\cohgrouprho(x))}}}$
%% | $\cohgroupinvadjtrr$ : $\depprod{(x : \cohgroup)}{\concat{\ap{(\Lam z. \cohgrouptensor(\cohgroupinv(x),z))}{(\cohgrouprinv(x))}}{\concat{\cohgroupalpha(\cohgroupinv(x),x,\cohgroupinv(x))}{\ap{(\Lam z. \cohgrouptensor(z,\cohgroupinv(x)))}{(\cohgrouplinv(x))}}}=\concat{\cohgrouprho(\cohgroupinv(x))}{\inverse{(\cohgrouplambda(\cohgroupinv(x)))}}}$
%% \end{lstlisting}
%% 
%% 
%% 
%% 
%% 
%% Signature for the coherent group
