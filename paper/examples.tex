In this section we present some additional examples, complementing the
ones introduced in Section \ref{sec:signs}.

\subsection{Coequalizer}
\label{sec:coequalizer}

The coqualizer of two morphisms is a basic categorical construction,
that admits the following bicategorical generalization.

\begin{definition}
Let $\B$ be a bicategory. Let $A$ and $B$ be objects in $B$ and let
$f,g : A \onecell B$. The \emph{coequalizer} of $f$ and $g$ is an
object $Q$ together with a 1-cell $q : B \onecell Q$ and an invertible
2-cell $\theta : f \cdot q\twocell g \cdot q$.

The triple $(Q,q,\theta)$ must satisfy the following universal
property. Assume given an object $C$, a 1-cell $h : B \onecell C$ and
an invertible 2-cell $\phi : f \cdot h \twocell g \cdot h$. Then there
exists a 1-cell $h' : Q \onecell C$ together with an invertible 2-cell
$\phi' : q \cdot h' \twocell h$ such that $\theta \whiskerr h' \vcomp g
\whiskerl \phi' = \lassoc{f}{q}{h'} \vcomp f \whiskerl \phi' \vcomp
\phi$
%% \footnote{Notice that this path would type-check also without the
%%   left associator $\lassoc{}{}{}$, since function composition is
%%   stricly associative. But the presence of the left associator is
%%   necessary if we consider algebras for the coequalizer in
%%   bicategories more general than the one of 1-types and groupoids.}.
Given another 1-cell $h'' : Q \onecell C$ and an invertible 2-cell $\phi'' :
q \cdot h'' \twocell h$ such that $\theta \whiskerr h'' \vcomp g
\whiskerl \phi'' = \lassoc{f}{q}{h''} \vcomp f \whiskerl \phi'' \vcomp
\phi$, then there exists a unique invertible 2-cell $\tau : h'
\twocell h''$ such that $q \whiskerl \tau \vcomp \phi'' = \phi'$.
\end{definition}

This section is dedicated to the construction of the coequalizer in
the bicategory of 1-types.  Given two 1-types $A$ and $B$ and two
functions $f , g: A \to B$, the coequalizer of $f$ and $g$ is
given by the following HIT:
\begin{lstlisting}[mathescape=true]
Inductive $\coequalizer f g$ :=
| $\coequalizerbase$ : $B \rightarrow \coequalizer f g$
| $\coequalizerglue$ : $\depprod{(x : A)}{\coequalizerbase(f(x)) = \coequalizerbase(g(x))}$
\end{lstlisting}

Here are all the ingredients needed to specify the signature of the coequalizer:
%is constructed as follows:
\begin{itemize}
\item $\pointconstr[\coequalizer f g] = \constantP{B}$;
\item $\pathlabel[\coequalizer f g] = \unit$, and for its unique inhabitant take $\patharg[\coequalizer f g] = \constantP{A}$ and endpoints
  \[
  \pathleft[\coequalizer f g] = \comp{\fmap(f)}{\constr}, \quad
  \pathright[\coequalizer f g] = \comp{\fmap(g)}{\constr}; 
  \]
\item $\homotlabel[\coequalizer f g]$ is the empty type.
\end{itemize}

%% The coequalizer satisfies the following universal property. Assume
%% given a 1-type $C$, a function $h : B \rightarrow C$ and a homotopy
%% $p$ between $f \cdot h$ and $g \cdot h$. Then there exists a function
%% $h' : \coequalizer f g \rightarrow C$ together with a homotopy $p'$
%% between $\coequalizerbase \cdot h'$ and $h$, and a path
%% $\coequalizerglue \whiskerr h' \vcomp g \whiskerl p' = \lassoc{}{}{}
%% \vcomp f \whiskerl p' \vcomp p$\footnote{Notice that this path would
%%   type-check also without the left associator $\lassoc{}{}{}$, since
%%   function composition is stricly associative. But the presence of the
%%   left associator is necessary if we consider algebras for the
%%   coequalizer in bicategories more general than the one of 1-types and
%%   groupoids.}.  Given another function $h'' : \coequalizer f g
%% \rightarrow C$ with homotopy $p''$ between $\coequalizerbase \cdot
%% h''$ and $h$ such that $\coequalizerglue \whiskerr h'' \vcomp g
%% \whiskerl p'' = \lassoc{}{}{} \vcomp f \whiskerl p'' \vcomp p$, then
%% there exists a unique homotopy $\tau$ between $h'$ and $h''$ such that
%% $\coequalizerbase \whiskerl \tau \vcomp p'' = p'$.



\subsection{Coequifier}
\label{sec:coequifier}

Assume two 1-types $A$ and $B$, two functions $f , g: A \to B$ and two
homotopies $p$ and $q$ between $f$ and $g$. The \emph{coequifier} of $p$ and
$q$ is given by the following HIT:
\begin{lstlisting}[mathescape=true]
Inductive $\coequifier p q$ :=
| $\coequifierbase$ : $B \rightarrow \coequifier f g$
| $\coequifierglue$ : $\depprod{(x : A)}{\ap {\coequifierbase}{(p(x))} = \ap {\coequifierbase}{(q(x))}}$
\end{lstlisting}

The signature of the coequifier of $p$ and $q$ is given as follows:
\begin{itemize}
\item $\pointconstr[\coequifier f g] = \constantP{B}$;
\item $\pathlabel[\coequifier f g]$ is the empty type; 
\item $\homotlabel[\coequifier f g] = \unit$, and take 
  $\homotpointarg[\coequifier f g] = \constantP{A}$.
  The homotopy constructor $\coequifierglue$ does not have path arguments, so we take  
  $\homotpathtarg[\coequifier f g] = \constantP{\unit}$ and
  $\pathargleft[\coequifier f g] = \pathargright[\coequifier f g] = \Ce(\unitt)$.
  The endpoints $\homotpathleft[\coequifier f g]$ and $\homotpathright[\coequifier f g]$ are:
  \[
  \homotpathleft[\coequifier f g] = \comp{\fmap(f)}{\constr}, \quad
  \homotpathright[\coequifier f g] = \comp{\fmap(g)}{\constr},
  \]
  while the left and right homotopy endpoints are:
  \[
  \hap{\constr}{(\idtoH(\ap{\fmap}{(\funextsec(p))}))}, \quad
  \hap{\constr}{(\idtoH(\ap{\fmap}{(\funextsec(q))}))}.
  \]
\end{itemize}


\subsection{Group Quotient}
\label{sec:group_quotient}

Let $G$ be a group with unit $e$ and multiplication $*$. The group
quotient of $G$ is the following HIT:
\begin{lstlisting}[mathescape=true]
Inductive $\groupquot G$ :=
| $\groupquotbase$ : $\groupquot G$
| $\groupquotloop$ : $G \rightarrow \groupquotbase = \groupquotbase$
| $\groupquotloope$ : $\groupquotloop(e)= \refl \groupquotbase$
| $\groupquotloopm$ : $\depprod{(x,y : G)}{\groupquotloop(x * y)= \concat{\groupquotloop(x)}{\groupquotloop(y)}}$
\end{lstlisting}


Signature for the group quotient

\subsection{Monoidal Object}
\label{sec:monoidal_object}


An important algebraic example is given by the \emph{monoidal object},
which can be presented as the following HIT:
\begin{lstlisting}[mathescape=true]
Inductive $\monobj$ :=
| $\monobjunit$ : $\monobj$
| $\monobjtensor$ : $\monobj \rightarrow \monobj \rightarrow \monobj$
| $\monobjlambda$ : $\depprod{(x : \monobj)}{\monobjtensor(\monobjunit,x) = x}$
| $\monobjrho$ : $\depprod{(x : \monobj)}{\monobjtensor(x,\monobjunit) = x}$
| $\monobjalpha$ : $\depprod{(x ,y,z: \monobj)}{\monobjtensor(x,(\monobjtensor(y,z)))=\monobjtensor(\monobjtensor(x,y),z)}$
| $\monobjtr$ : $\depprod{(x,y : \monobj)}{\ap{(\Lam z. \monobjtensor(x,z))}{(\monobjlambda(y))}} = \concat{\monobjalpha(x,\monobjunit,y)}{\ap{(\Lam z. \monobjtensor(z,y))}{(\monobjrho(x))}}$
| $\monobjpent$ : $\depprod{(w,x,y,z : \monobj)}{}$
    $\concat{\monobjalpha(w,x,\monobjtensor(y,z))}{\monobjalpha(\monobjtensor(w,x),y,z)}=\concat{\ap{(\Lam v. \monobjtensor(w,v))}{(\monobjalpha(x,y,z))}}{\concat{\monobjalpha(w,\monobjtensor(x,y),z)}{\ap{(\Lam v. \monobjtensor(v,z))}{(\monobjalpha(w,x,y))}}}$
\end{lstlisting}
We do not show the signature associated to this HIT here. We redirect
the interested reader to our formalization for the complete
definition.

Our interest in these example stems from the algebras in groupoids of
the monoidal object signature, which correspond precisely to
\emph{monoidal groupoids}, the groupoid variant of monoidal
categories. $\monobj$ is therefore a presentation of the initial
monoidal groupoid.

We also formalized the \emph{coherent group} \cite{baez2004groups} as
a HIT, that we name $\cohgroup$.  Its definition includes all the
constructors of the monoidal object, plus a new point constructor
$\cohgroupinv$ : $\cohgroup \rightarrow \cohgroup$, two new path
constructors
\[
\cohgrouplinv : \depprod{(x : \cohgroup)}{\cohgrouptensor(\cohgroupinv(x),x) = \cohgroupunit},
\quad\quad
\cohgrouprinv : \depprod{(x : \cohgroup)}{\cohgroupunit = \cohgrouptensor(x,\cohgroupinv(x))},
\]
and two new homotopy constructors.

%% \subsection{Coherent Group}
%% \label{sec:coeherent_group}
%% 
%% The coherent group is defined by the following HIT:
%% \begin{lstlisting}[mathescape=true]
%% Inductive $\cohgroup$ :=
%% | $\cohgroupunit$ : $\cohgroup$
%% | $\cohgrouptensor$ : $\cohgroup \rightarrow \cohgroup \rightarrow \cohgroup$
%% | $\cohgroupinv$ : $\cohgroup \rightarrow \cohgroup$
%% | $\cohgrouplambda$ : $\depprod{(x : \cohgroup)}{\cohgrouptensor(\cohgroupunit,x) = x}$
%% | $\cohgrouprho$ : $\depprod{(x : \cohgroup)}{\cohgrouptensor(x,\cohgroupunit) = x}$
%% | $\cohgroupalpha$ : $\depprod{(x ,y,z: \cohgroup)}{\cohgrouptensor(x,(\cohgrouptensor(y,z)))=\cohgrouptensor(\cohgrouptensor(x,y),z)}$
%% | $\cohgrouplinv$ : $\depprod{(x : \cohgroup)}{\cohgrouptensor(\cohgroupinv(x),x) = \cohgroupunit}$
%% | $\cohgrouprinv$ : $\depprod{(x : \cohgroup)}{\cohgroupunit = \cohgrouptensor(x,\cohgroupinv(x))}$
%% | $\cohgrouptr$ : $\depprod{(x,y : \cohgroup)}{\ap{(\Lam z. \cohgrouptensor(x,z))}{(\cohgrouplambda(y))}} = \concat{\cohgroupalpha(x,\cohgroupunit,y)}{\ap{(\Lam z. \cohgrouptensor(z,y))}{(\cohgrouprho(x))}}$
%% | $\cohgrouppent$ : $\depprod{(w,x,y,z : \cohgroup)}{}$
%%     $\concat{\cohgroupalpha(w,x,\cohgrouptensor(y,z))}{\cohgroupalpha(\cohgrouptensor(w,x),y,z)}=\concat{\ap{(\Lam v. \cohgrouptensor(w,v))}{(\cohgroupalpha(x,y,z))}}{\concat{\cohgroupalpha(w,\cohgrouptensor(x,y),z)}{\ap{(\Lam v. \cohgrouptensor(v,z))}{(\cohgroupalpha(w,x,y))}}}$
%% | $\cohgroupinvadjtrl$ : $\depprod{(x : \cohgroup)}{\concat{\ap{(\Lam z. \cohgrouptensor(z,x))}{(\cohgrouprinv(x))}}{\concat{\inverse{(\cohgroupalpha(x,\cohgroupinv(x),x))}}{\ap{(\Lam z. \cohgrouptensor(x,z))}{(\cohgrouplinv(x))}}}=\concat{\cohgrouplambda(x)}{\inverse{(\cohgrouprho(x))}}}$
%% | $\cohgroupinvadjtrr$ : $\depprod{(x : \cohgroup)}{\concat{\ap{(\Lam z. \cohgrouptensor(\cohgroupinv(x),z))}{(\cohgrouprinv(x))}}{\concat{\cohgroupalpha(\cohgroupinv(x),x,\cohgroupinv(x))}{\ap{(\Lam z. \cohgrouptensor(z,\cohgroupinv(x)))}{(\cohgrouplinv(x))}}}=\concat{\cohgrouprho(\cohgroupinv(x))}{\inverse{(\cohgrouplambda(\cohgroupinv(x)))}}}$
%% \end{lstlisting}
%% 
%% 
%% 
%% 
%% 
%% Signature for the coherent group
