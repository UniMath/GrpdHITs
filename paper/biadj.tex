To construct higher inductive types, we use Proposition \ref{thm:initial_alg_sem}, which says that binitial objects satisfy the induction principle.
We use the groupoid quotient to acquire the desired algebra.
More specifically, we construct a pseudofunctor from $\algG{\sign}$ to $\algM(\sign)$,
which is the groupoid quotient on the carrier.
We do that in such a way that the obtained pseudofunctor preserves biinitiality,
so that we obtain the HIT by constructing a biinitial object in $\algG{\sign}$.

One class of pseudofunctors which preserve biinitial objects, are left biadjoints.
Let us state that more precisely, so suppose that we have bicategories $\B$ and $\C$, a left biadjoint pseudofunctor $L : \pseudo(\B, \C)$, and an object $x : \B$.
Then the object $L(x)$ is biinitial if $x$ is.

Instead of just lifting the groupoid quotient to the level of algebras,
we first show that the groupoid quotient is a left biadjoint and then we lift that biadjunction to the level of algebras.
To do so, we use the fact we defined the bicategory of algebras via displayed bicategories.
This way we can define the biadjunction on each part of the structure separately.

More specifically, we define the notion of \emph{displayed biadjunction} between two displayed bicategories over a biadjunction in the base,
and we show that each displayed biadjunction gives rise to a total biadjunction between the total bicategories.
Defining displayed biadjunctions requires defining displayed analogues of pseudofunctors, pseudotransformations, and invertible modification,
which were defined by Ahrens \etal \ \cite{bicatjournal}.
For this, we make use of \emph{displayed invertible 2-cells} \cite{bicatjournal}.

\begin{definition}
Let $\D_1$ and $\D_2$ be displayed bicategories over $\B_1$ and $\B_2$ respectively and let $F : \pseudo(\B_1, \B_2)$ be a pseudofunctor.
Then a \fat{displayed pseudofunctor} $\FF$ from $\D_1$ to $\D_2$ over $F$ consist of
\begin{itemize}
	\item For each $x : \B_1$ a map $\FF_0 : \D_1(x) \rightarrow \D_2(F(x))$;
	\item For all 1-cells $f : x \onecell y$, objects $\xx : \D_1(x)$ and $\yy : \D_1(y)$,
	and displayed 1-cells $\ff : \dmor{\xx}{\yy}{f}$, a displayed 1-cell $\FF_1(\ff) :  \dmor{\FF_0(\xx)}{\FF_0(\yy)}{F(f)}$;
	\item For all 2-cells $\tc : f \twocell g$, displayed 1-cells $\ff : \dmor{\xx}{\yy}{f}$ and $\gg : \dmor{\xx}{\yy}{g}$,
	and displayed 2-cells $\dtc : \dtwo{\ff}{\gg}{\tc}$, a displayed 2-cell $\FF_2(\dtc) : \dtwo{\FF_1(\ff)}{\FF_2(\gg)}{F(\tc)}$;
	\item For each $x : \B$ and $\xx : \D(x)$, a displayed invertible 2-cell
	$\identitor{\FF}(\xx) : \dmor{\id_1(\FF_0(\xx))}{\FF_1(\id_1(\xx))}{\identitor{F}(x)}$;
	\item For all $\ff : \dmor{\xx}{\yy}{f}$ and $\gg : \dmor{\yy}{\zz}{g}$. a displayed invertible 2-cell
	$\compositor{\FF}(\ff, \gg) : \dmor{\FF_1(\ff) \cdot \FF_1(\gg)}{\FF_1(\ff \cdot \gg)}{\compositor{F}(f, g)}$.
\end{itemize}
Here $\identitor{F}$ and $\compositor{F}$ denote the identitor and compositor of $F$.
In addition, several coherencies, which can be found in the formalization, need to be hold.
We denote the type of displayed pseudofunctors from $\D_1$ to $\D_2$ over $F$ by $\disppsfun{\D_1}{\D_2}{F}$.
\end{definition}

\begin{definition}
Let $\D_1$ and $\D_2$ be displayed bicategories over $\B_1$ and $\B_2$ respectively.
Suppose that we have displayed pseudofunctors $\FF : \disppsfun{\D_1}{\D_2}{F}$ and $\GG : \disppsfun{\D_1}{\D_2}{G}$
and a pseudotransformation $\theta : \pstrans{F}{G}$.
Then a \fat{displayed pseudotransformation} $\thetatheta$ from $\FF$ to $\GG$ over $\theta$ consists of
\begin{itemize}
	\item For all objects $x : \B$ and $\xx : \D_1(x)$ a displayed 1-cell $\thetatheta_1(x) : \dmor{\FF_0(\xx)}{\GG_0(\xx)}{\theta(x)}$;
	\item For all 1-cells $f : x \onecell y$ and $\ff : \dmor{\xx}{\yy}{f}$ a displayed invertible 2-cell
	$\thetatheta_2(f) : \dtwo{\thetatheta_1(\xx) \cdot \FF_1(\ff)}{\GG_1(\ff) \cdot \thetatheta_1(\yy)}{\theta(f)}$.
\end{itemize}
Again several coherencies must be satisfied and the precise formulation can be found in the formalization.
The type of displayed pseudotransformatons from $\FF$ to $\GG$ over $\theta$ is denoted by $\disppstrans{\FF}{\GG}{\theta}$.
\end{definition}

\begin{definition}
Suppose that we have displayed bicategories $\D_1$ and  $\D_2$ over $\B_1$ and $\B_2$, displayed pseudofunctors $\FF$ and $\GG$ from $\D_1$ to $\D_2$ over $F$ and $G$ respectively, and displayed pseudotransformations $\thetatheta$ and $\thetatheta'$ from $\FF$ to $\GG$ over $\theta$ and $\theta'$ respectively.
In addition, let $m$ be an invertible modification from $\theta$ to $\theta'$.
Then a \fat{displayed invertible modification} $\mm$ from $\thetatheta$ to $\thetatheta'$ over $m$ consists of a displayed invertible 2-cell
$\mm_2(\xx) :  \dtwo{\thetatheta(\xx)}{\thetatheta'(\xx)}{m(x)}$ for each $x : \B_1$ and $\xx : \D_1(x)$,
In addition, a coherency must be satisfied, which can be found in the formalization.
The type of displayed invertible modifications from $\thetatheta$ to $\thetatheta'$ over $m$ is denoted by $\dispmodif{\thetatheta}{\thetatheta'}{m}$.
\end{definition}

Each of these gadgets has a total version.

\begin{problem}
\label{prob:total}
We have
\begin{enumerate}
	\item Given a displayed pseudofunctor $\FF : \disppsfun{\D_1}{\D_2}{F}$, to construct a pseudofunctor $\total{\FF} : \pseudo(\total{\D_1}, \total{\D_2})$;
	\item Given a displayed pseudotransformation $\thetatheta : \disppstrans{\FF}{\GG}{\theta}$, to construct a pseudotransformation $\total{\thetatheta} : \pstrans{\total{\FF}}{\total{\GG}}$;
	\item Given a displayed invertible modification $\mm : \xymatrix@C=1em{\thetatheta \ar@3[r]^-{m} & \thetatheta',}$ to construct an invertible modificaton $\total{\mm} : \modif{\total{\thetatheta}}{\total{\thetatheta'}}$.
\end{enumerate}
\end{problem}

\begin{construction}{prob:total}
\label{constr:biadj}
By pairing.
\end{construction}

Before we can define displayed biadjunctions, we need several operations on the displayed gadgets we introduced.

\begin{example}
We have the following
\begin{itemize}
	\item We have $\id(\D) : \disppsfun{\D}{\D}{\id(\B)}$
	where $\id(\B)$ is the identity;
	\item Given $\FF : \disppsfun{\D_1}{\D_2}{F}$ and $\GG : \disppsfun{\D_2}{\D_3}{G}$,
	we have $\FF \cdot \GG : \disppsfun{\D_1}{\D_3}{F \cdot G}$
	where $F \cdot G$ is the composition;
	\item Given $\FF : \disppsfun{\D_1}{\D_2}{F}$,
	we have $\id_1(\FF) : \disppstrans{\FF}{\FF}{\id_1(F)}$
	where $\id_1(F)$ is the identity transformation on $F$;
	\item Given $\thetatheta : \disppstrans{\FF}{\GG}{\theta}$ and $\thetatheta' : \disppstrans{\GG}{\HH}{\theta'}$,
	we have $\thetatheta \vcomp \thetatheta' : \disppstrans{\FF}{\HH}{\theta \vcomp \theta'}$
	where $\theta \vcomp \theta'$ is the composition;
	\item Given $\FF : \disppsfun{\D_1}{\D_2}{F}$, $\GG : \disppsfun{\D_2}{\D_3}{G}$, $\HH : \disppsfun{\D_2}{\D_3}{H}$, and $\thetatheta : \disppstrans{\GG}{\HH}{\theta}$,
	we have $\FF \whiskerl \thetatheta : \disppsfun{\FF \cdot \GG}{\FF \cdot \HH}{F \whiskerl \theta}$;
	\item Given $\FF : \disppsfun{\D_1}{\D_2}{F}$, $\GG : \disppsfun{\D_1}{\D_2}{G}$, $\HH : \disppsfun{\D_2}{\D_3}{H}$, and $\thetatheta : \disppstrans{\FF}{\GG}{\theta}$,
	we have $\thetatheta \whiskerr \HH : \disppsfun{\FF \cdot \HH}{\GG \cdot \HH}{\theta \whiskerr H}$;
	\item Given $\FF : \disppsfun{\D_1}{\D_2}{F}$,
	we have
	\[
	\lunitor{\FF} : \disppstrans{\id \cdot \FF}{\FF}{\lunitor{F}}, \quad
	\runitor{\FF} : \disppstrans{\FF \cdot \id}{\FF}{\runitor{\FF}};
	\]
	\item Given $\FF : \disppsfun{\D_1}{\D_2}{F}$,
	we have
	\[
	\linvunitor{\FF} : \disppstrans{\FF}{\id \cdot \FF}{\linvunitor{F}}, \quad
	\rinvunitor{\FF} : \disppstrans{\FF}{\FF \cdot \id}{\rinvunitor{\FF}};
	\]
	\item Given $\FF : \disppsfun{\D_1}{\D_2}{F}$, $\GG : \disppsfun{\D_2}{\D_3}{G}$, and $\HH : \disppsfun{\D_3}{\D_4}{H}$,
	we have
	\[
	\lassoc{\FF}{\GG}{\HH} : \disppstrans{(\FF \cdot \GG) \cdot \HH}{\FF \cdot (\GG \cdot \HH)}{\lassoc{F}{G}{H}},
	\]
	\[
	\rassoc{\FF}{\GG}{\HH} : \disppstrans{\FF \cdot (\GG \cdot \HH)}{(\FF \cdot \GG) \cdot \HH}{\rassoc{F}{G}{H}}.
	\]
\end{itemize}
\end{example}

\begin{definition}
Suppose we have bicategories $\B_1$ and $\B_2$ and a biadjunction $L \dashv R$ from $\B_1$ to $\B_2$.
We write $\eta$ and $\epsilon$ for the unit and counit of $L \dashv R$ respectively,
and we write $\tau_1$ and $\tau_r$ for the left and right triangle respectively.
Suppose, that we also have displayed bicategories $\D_1$ and $\D_2$ over $\B_1$ and $\B_2$ respectively
and a displayed pseudofunctor $\LL : \disppsfun{\D_1}{\D_2}{L}$.
Then we say $\LL$ is a \fat{displayed left biadjoint pseudofunctor} if we have
\begin{itemize}
	\item A displayed pseudofunctors
	$
	\RR : \disppsfun{\D_2}{\D_1}{R}
	$;
	\item Displayed pseudotransformations
	\[
	\etaeta : \disppstrans{\id}{\LL \cdot \RR}{\eta}, \quad
	\epseps : \disppstrans{\RR \cdot \LL}{\id}{\epsilon};
	\]
	\item Displayed invertible modifications
	\[
	\tautaul : \dispmodif{\rho^{-1} \vcomp \LL \whiskerl \etaeta \vcomp \alpha \vcomp \epseps \whiskerr \LL \vcomp \lambda}{\id_1(\LL),}{\tau_1}
	\]
	\[
	\tautaur : \dispmodif{\lambda^{-1} \vcomp \etaeta \whiskerr \RR \vcomp \alpha^{-1} \vcomp \RR \whiskerl \epseps \vcomp \rho}{\id_1(\RR).}{\tau_2}
	\]
\end{itemize}
\end{definition}

From Construction \ref{constr:biadj}, we get

\begin{proposition}
\label{prop:total_biadj}
Given a displayed left biadjoint pseudofunctor $\LL$,
then $\total{\LL}$ is a left biadjoint pseudofunctor.
\end{proposition}

\begin{figure}
\[
\xymatrix
{
	\algebra{\sign} \ar@/^/[rr]_-{\top}^-{\algpgrpd} \ar[d] & & \algG{\sign} \ar@/^/[ll]^-{\alggquot} \ar[d] \\
	\pathalg{\sign} \ar@/^/[rr]_-{\top}^-{\pathpgrpd} \ar[d] & & \pathalgG{\sign} \ar@/^/[ll]^-{\pathgquot} \ar[d] \\
	\prealg{\sign} \ar@/^/[rr]_-{\top}^-{\prepgrpd} \ar[d] & & \prealgG{\sign} \ar@/^/[ll]^-{\pregquot} \ar[d]\\
	\onetypes \ar@/^/[rr]_-{\top}^-{\pgrpd} & & \grpd \ar@/^/[ll]^-{\gquot}
}
\]
\caption{The biadjunction}
\label{fig:biadj}
\end{figure}

Now let us use the introduced notions to construct the biadjunction on the level of algebras.
Our approach is summarized in Figure \ref{fig:biadj}.
We start by showing that the groupoid quotient gives rise to a biadjunction.

\begin{problem}
\label{prob:gquit_biadj}
To construct $\gquot \dashv \pgrpd$ with a pseudofunctor $\gquot : \pseudo(\grpd, \onetypes)$.
\end{problem}

\begin{construction}{prob:gquit_biadj}
\label{constr:gquit_biadj}
We only show how the involved pseudofunctors are defined.
The pseudofunctor $\gquot$ is the groupoid quotient
while $\pgrpd$ sends a 1-type $X$ to the groupoid whose objects are points of $X$ and morphisms from $x$ to $y$ are paths $x = y$.
\end{construction}

Next we lift it to the level of algebras using the displayed machinery introduced in this section.

\begin{problem}
\label{prob:alg_biadj}
Given a signature $\sign$, to construct a biadjunction $\alggquot \dashv \algpgrpd$ where
\[
\alggquot : \pseudo(\algG{\sign}, \algebra{\sign}).
\]
\end{problem}

\begin{construction}{prob:alg_biadj}
\label{constr:alg_biadj}
We only give a very brief outline of the construction.

We start by constructing a displayed biadjunction
from $\DFAlg(\semPG{\pointconstr[\sign]})$
to $\DFAlg(\semP{\pointconstr[\sign]})$
over the biadjunction from
Construction \ref{constr:gquit_biadj}.
To do so, we first need to lift the pseudofunctors, and for that, we generalize the approach of Hermida and Jacobs
to the bicategorical setting \cite[Theorem 2.14]{hermida1998structural}.
This requires us to construct two pseudotransformations.
\[
p_1 : \pstrans{\semP{P} \cdot \gquot}{\gquot \cdot \semPG{P}},
\]
\[
p_2 : \pstrans{\semPG{P} \cdot \pgrpd}{\pgrpd \cdot \semP{P}}.
\]
We denote the total biadjunction of the resulting displayed biadjunction by $\pregquot \dashv \prepgrpd$.

Next we lift the biadjunction to the level of path algebras
and for that, we construct a displayed biadjunction between 
$\DCell(\semEG{\pathleft[\sign](i)},\semEG{\pathright[\sign](i)})$
and $\DCell(\semE{\pathleft[\sign](i)},\semE{\pathright[\sign](i)}$
for all $j : \pathlabel$.
Denote the resulting total biadjunction by $\pathgquot \dashv \pathpgrpd$.

To finish the proof, we need to construct one more displayed biadjunction.
For that, we only need to show that if $G : \pathalgG{\sign}$ is an algebra, then $\pathgquot(G)$ also is an algebra,
and if $X : \pathalg{\sign}$ is an algebra, then so is $\pathpgrpd(X)$.
\end{construction}

The next proposition concludes this section.

\begin{proposition}
\label{prop:biinitial_in_grpd}
If $G$ is an biinitial object in $\algG{\sign}$,
then $\alggquot(G)$ is a biinitial object in $\algebra{\sign}$.
\end{proposition}