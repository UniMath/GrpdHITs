We showed how to construct finitary 1-truncated higher inductive types using the propositional truncation, quotient, and the groupoid quotient.
This reduces the existence of a general class of HITs to simpler ones.
We needed the types to be 1-truncated, so that we could use the framework of bicategory theory,
and the HITs we studied had to be finitary to guarantee that the groupoid quotient
commutes with the involved operations \cite{DBLP:journals/mscs/ChapmanUV19}.
On the way, we also proved that HITs are unique and we studied universal algebra with our signatures.
We showed that the bicategory of algebras has finite limits and we proved the first isomorphism theorem for these algebras.
Lastly, we used the way we constructed HITs to calculate fundamental groups.

There are numerous ways to improve on these results.
First of all, we only constructed finite limits of algebras while it should also be possible to construct finite colimits of algebras.
The scheme studied in this paper is not flexible enough to support these colimits since we do not have a path endpoint that represents the action of a polynomial on the point constructor.
Hence, if we want to internally construct these colimits, then we need to define a more permissive signature for higher inductive types.

Secondly, it should be possible to modify our approach to obtain HITs in directed type theory (DTT) \cite{north2019towards}.
In the model of DTT provided by North, types are interpreted as categories and higher inductive types in DTT could be interpreted as initial algebras.
We constructed such algebras in the bicategory of groupoids and in a similar way, one should be able to construct the desired algebras in the bicategory of categories.

Lastly, our construction only considers a rather simple scheme of HITs.
In particular, we restrict ourselves to the 1-truncated case. % \cite{KaposiK18}.
Since untruncated types correspond to $\infty$-groupoids,
generalizing the methods used in this paper to the untruncated case,
requires formalizing notions from $\infty$-category theory in type theory
\cite{DBLP:conf/csl/AltenkirchR12,DBLP:journals/pacmpl/CapriottiK18,FinsterM17}.
This also requires finding an $\infty$-dimensional generalization of the groupoid quotient.
An alternative approach to deal with untruncated HITs, pointed out by Ali Caglayan,
would be using wild categories \cite{DBLP:conf/tlca/HirschowitzHT15,KrausRaumer}.
Note that generalizing Construction \ref{constr:initial_grpd_alg} would also pose a challenge, because the set quotient cannot be used.
Instead all $n$-morphisms must be freely generated by the $n$-path constructors.
We would also like to extend our scheme to incorporate both indexed HITs and higher inductive-inductive types \cite{CavalloH19,KaposiK18}.
